\appendixchapter{Message Class/Field Properties}
\label{cha:msg-properties}

This appendix lists the properties that can be used to customize C++ code
generated from message descriptions.

%
% Note: the following has been generated with the following command:
% opp_msgtool -h latexdoc
%
\begin{description}
\item[allowReplace] \textit{(type: bool, use: field)} \\
    Specifies whether the setter method of an owned pointer field is allowed to
    delete the previously set object.

\item[argType] \textit{(type: string, use: field, class)} \\
    Field setter C++ argument type. When specified on a class, it determines the
    default for fields of that type.

\item[beforeChange] \textit{(type: string, use: class)} \\
    Method to be called before mutator code (in setters, non-const getters,
    operator=, etc.).

\item[byValue] \textit{(type: bool, use: field, class)} \\
    If true: Causes the value to be passed by value (instead of by reference) in
    setters/getters. When specified on a class, it determines the default for
    fields of that type.

\item[castFunction] \textit{(type: bool, use: class)} \\
    If false: Do not specialize the fromAnyPtr<T>(any\_ptr) function for this
    class. Useful for preventing compile errors if the function already exists,
    e.g. in hand-written form, or generated for another type (think aliased
    typedefs).

\item[clone] \textit{(type: string, use: field, class)} \\
    For owned pointer fields: Code to duplicate (one array element of) the field
    value. When specified on a class, it determines the default for fields of
    that type.

\item[cppType] \textit{(type: string, use: field, class)} \\
    Member C++ datatype. When specified on a class, it determines the default
    for fields of that type.

\item[custom] \textit{(type: bool, use: field)} \\
    If true: Do not generate any data or code for the field, only add it to the
    descriptor. Indicates that the field's implementation will be added to the
    class via targeted cplusplus blocks.

\item[customize] \textit{(type: bool, use: class)} \\
    If true: Customize the class via inheritance. Generates base class
    <name>\_Base.

\item[defaultValue] \textit{(type: string, use: class)} \\
    Default value for fields of this type.

\item[descriptor] \textit{(type: string, use: class)} \\
    A 'true'/'false' value specifies whether to generate descriptor class;
    special value 'readonly' requests generating a read-only descriptor (but
    specifying @editable/@replaceable/@resizable on individual fields overrides
    that).

\item[editable] \textit{(type: bool, use: field, class)} \\
    Specifies whether field value (or value of fields that are instances of this
    type) can be set via the class descriptor's setFieldValueFromString() and
    setFieldValue() methods.

\item[enum] \textit{(type: string, use: field)} \\
    For integer fields: Values are from the given enum.

\item[eraser] \textit{(type: string, use: field)} \\
    Name of the eraser method. (This method erases element from a dynamic
    array.)

\item[eventlog] \textit{(type: string, use: field)} \\
    When @eventlog(skip) is given, eventlog recording will skip this field when
    serializing objects

\item[existingClass] \textit{(type: bool, use: class)} \\
    If true: This is a type is already defined in C++, i.e. it does not need to
    be generated.

\item[fieldNameSuffix] \textit{(type: string, use: class)} \\
    Suffix to append to the names of data members.

\item[fromString] \textit{(type: string, use: field, class)} \\
    Affects descriptor class: Code to convert string to field value. When
    specified on a class, it determines the default for fields of that type.

\item[fromValue] \textit{(type: string, use: field, class)} \\
    Affects descriptor class: Code to convert cValue to field value. When
    specified on a class, it determines the default for fields of that type.

\item[getter] \textit{(type: string, use: field)} \\
    Name of the (const) getter method.

\item[getterConversion] \textit{(type: string, use: field, class)} \\
    Code to convert field data type to return type in getters. When specified on
    a class, it determines the default for fields of that type.

\item[getterForUpdate] \textit{(type: string, use: field)} \\
    Name of the non-const getter method.

\item[group] \textit{(type: string, use: field)} \\
    Used for grouping of fields in Qtenv inspectors

\item[hint] \textit{(type: string, use: field)} \\
    Short description of the field, displayed in Qtenv inspectors as tooltip

\item[icon] \textit{(type: string, use: class)} \\
    Icon for objects of this class in Qtenv inspectors

\item[implements] \textit{(type: stringlist, use: class)} \\
    Names of additional base classes.

\item[inserter] \textit{(type: string, use: field)} \\
    Name of the inserter method. (This method inserts element into a dynamic
    array.)

\item[label] \textit{(type: string, use: field)} \\
    When specified, this string will be displayed as field name in Qtenv
    inspectors

\item[nopack] \textit{(type: bool, use: field)} \\
    If true: Ignore this field in parsimPack/parsimUnpack methods.

\item[omitGetVerb] \textit{(type: bool, use: class)} \\
    If true: Drop the 'get' verb from the names of getter methods.

\item[opaque] \textit{(type: bool, use: field, class)} \\
    If true: Treat the field as atomic (non-compound) type, i.e. having no
    descriptor class. When specified on a class, it determines the default for
    fields of that type.

\item[overrideGetter] \textit{(type: bool, use: field)} \\
    If true: Add the 'override' keyword to the declaration of the getter method.

\item[overrideSetter] \textit{(type: bool, use: field)} \\
    If true: Add the 'override' keyword to the declaration of the setter method.

\item[owned] \textit{(type: bool, use: field)} \\
    For pointers and pointer arrays: Whether allocated memory is owned by the
    object (needs to be duplicated in dup(), and deleted in destructor). If
    field type is also a cOwnedObject, take()/drop() calls are also generated.

\item[packetData] \textit{(type: string, use: class, field)} \\
    Denotes packet data in frameworks such as INET; used in Qtenv inspectors

\item[polymorphic] \textit{(type: bool, use: class)} \\
    Specifies whether this type is polymorphic, i.e. has any virtual member
    function.

\item[primitive] \textit{(type: bool, use: field, class)} \\
    Shortcut for @opaque @byValue @editable @subclassable(false)
    @supportsPtr(false).

\item[property] \textit{(type: any, use: file)} \\
    Property for declaring properties.

\item[readonly] \textit{(type: bool, use: field)} \\
    If true: Equivalent to @editable(false) @replaceable(false)
    @resizable(false).

\item[replaceable] \textit{(type: bool, use: field)} \\
    If true: Field is a pointer whose value can be set via the class
    descriptor's setFieldStructValuePointer() and setFieldValue() methods.

\item[resizable] \textit{(type: bool, use: field)} \\
    If true: Field is a variable-size array whose size can be set via the class
    descriptor's setFieldArraySize() method.

\item[returnType] \textit{(type: string, use: field, class)} \\
    Field getter C++ return type. When specified on a class, it determines the
    default for fields of that type.

\item[setter] \textit{(type: string, use: field)} \\
    Name of the setter method.

\item[sizeGetter] \textit{(type: string, use: field)} \\
    Name of the method that returns the array size.

\item[sizeSetter] \textit{(type: string, use: field)} \\
    Name of the method that sets size of dynamic array.

\item[sizeType] \textit{(type: string, use: field)} \\
    C++ type to use for array sizes and indices.

\item[str] \textit{(type: string, use: class)} \\
    Expression to be returned from the generated str() method.

\item[subclassable] \textit{(type: bool, use: class)} \\
    Specifies whether this type can be subclassed (e.g. C++ primitive types and
    final classes cannot).

\item[supportsPtr] \textit{(type: bool, use: field, class)} \\
    Specifies whether this type supports creating a pointer (or pointer array)
    from it.

\item[toString] \textit{(type: string, use: field, class)} \\
    Affects descriptor class: Code to convert field value to string. When
    specified on a class, it determines the default for fields of that type.

\item[toValue] \textit{(type: string, use: field, class)} \\
    Affects descriptor class: Code to convert field value to cValue. When
    specified on a class, it determines the default for fields of that type.

\end{description}
%%% Local Variables:
%%% mode: latex
%%% TeX-master: "usman"
%%% End:
