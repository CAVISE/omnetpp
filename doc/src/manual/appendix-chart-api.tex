\appendixchapter{Python API for Chart Scripts}
\label{cha:chart-api}

This chapter describes the API of the Python modules available
for chart scripts. These modules are available in the Analysis Tool
in the IDE, in \fprog{opp\_charttool}, and may also be used
in standalone Python scripts.

Some conventional import aliases appear in code fragments throughout this
chapter, such as \ttt{np} for NumPy and \ttt{pd} for Pandas.

\section{Modules}
\label{cha:chart-api:modules}

% --------
% The rest of this file is the output of the tools/extract_chart_api.py
% script (tools/chartapi.txt), pasted here manually, do not edit here!
% --------

% @BEGINFILE tools/chartapi.txt
%
% generated with extract_chart_api.py
%

\subsection{Module omnetpp.scave.results}
\label{cha:chart-api:omnetpp.scave.results}

\par Provides access to simulation results loaded from OMNeT++ result files (.sca,
.vec). The results are returned as Pandas \ttt{DataFrame}'s of various formats.

\par The module can be used in several ways, depending on the environment it is run
in, and on whether the set of result files to query are specified in a stateful
or a stateless way:

\begin{enumerate}
  \item 

\par Inside a chart script in the Analysis Tool in the Simulation IDE. In that
mode, the set of result files to take as input are defined on the "Inputs"
page of the editor. The \ttt{get\_results()}, \ttt{get\_scalars()} and similar methods
are invoked with a filter string as first argument to select the appropriate
subset of results from the result files. Note that this mode of operation is
stateful: The state is set up appropriately by the IDE before the chart
script is run.

\par A similar thing happens when charts in an analysis (.anf) file are run from
within \ttt{opp\_charttool}: the tool sets up the module state before running the
chart script, so that the getter methods invoked with a filter string will
return result from the set of result files saved as "inputs" in the anf file.
  \item 

\par Standalone stateful mode. In order to use \ttt{get\_results()}, \ttt{get\_scalars()}
and similar methods with a filter string, the module needs to be configured
via the \ttt{set\_inputs()}/\ttt{add\_inputs()} functions to tell it the set of result
files to use as input for the queries. (Doing so is analogous to filling in
the "Inputs" page in the IDE).
  \item 

\par Stateless mode. It is possible to load the result files (in whole or a subset
of results in them) into memory as a "raw" \ttt{DataFrame} using
\ttt{read\_result\_files()}, and then use \ttt{get\_scalars()}, \ttt{get\_vectors()} and
other getter functions with the dataframe as their first argument to produce
\ttt{DataFrame}'s of other formats. Note that when going this route, a filter
string can be specified to \ttt{read\_result\_files()} but not to the getter
methods. However, Pandas already provides several ways for filtering the rows
of a dataframe, for example by indexing with logical operators on columns, or
using the \ttt{df.query()}, \ttt{df.pipe()} or \ttt{df.apply()} methods.

\end{enumerate}


\par \textbf{Filter expressions}

\par The \ttt{filter\_or\_dataframe} parameters in all functions must contain either a
filter string, or a "raw" dataframe produced by \ttt{read\_result\_files()}. When it
contains a filter string, the function operates on the set of result files
configured earlier (see stateful mode above).

\par Filter strings of all functions have the same syntax. It is always evaluated
independently on every loaded result item or metadata entry, and its value
determines whether the given item or piece of metadata is included in the
returned \ttt{DataFrame}.

\par A filter expression is composed of terms that can be combined with the \ttt{AND},
\ttt{OR}, \ttt{NOT} operators, and parentheses. A term filters for the value of some
property of the item, and has the form \ttt{<property> ={\textasciitilde} <pattern>}, or simply
\ttt{<pattern>}. The latter is equivalent to \ttt{name ={\textasciitilde} <pattern>}.

\par The following properties are available:

\begin{itemize}
  \item \ttt{name}: Name of the result or item.
  \item \ttt{module}: Full path of the result's module.
  \item \ttt{type}: Type of the item. Value is one of: \ttt{scalar}, \ttt{vector}, \ttt{parameter},
\ttt{histogram}, \ttt{statistics}.
  \item \ttt{isfield}: \ttt{true} is the item is a synthetic scalar that represents a field of
statistic or a vector, \ttt{false} if not.
  \item \ttt{file}: File name of the result or item.
  \item \ttt{run}: Unique run ID of the run that contains the result or item.
  \item \ttt{runattr:<name>}: Run attribute of the run that contains the result or item.
Example: \ttt{runattr:measurement}.
  \item \ttt{attr:<name>}: Attribute of the result. Example: \ttt{attr:unit}.
  \item \ttt{itervar:<name>}: Iteration variable of the run that contains the result or
item. Example: \ttt{itervar:numHosts}.
  \item \ttt{config:<key>}: Configuration key of the run that contains the result or item.
Example: \ttt{config:sim-time-limit}, \ttt{config:**.sendIaTime}.

\end{itemize}


\par Patterns may contain the following wildcards:

\begin{itemize}
  \item \ttt{?} matches any character except '.'
  \item \ttt{*} matches zero or more characters except '.'
  \item \ttt{**} matches zero or more characters (any character)
  \item \ttt{\{a-z\}} matches a character in range a-z
  \item \ttt{\{{\textasciicircum}a-z\}} matches a character not in range a-z
  \item \ttt{\{32..255\}} any number (i.e. sequence of digits) in range 32..255 (e.g. \ttt{99})
  \item \ttt{[32..255]} any number in square brackets in range 32..255 (e.g. \ttt{[99]})
  \item \ttt{{\textbackslash}} takes away the special meaning of the subsequent character

\end{itemize}


\par Patterns only need to be surrounded with quotes if they contain whitespace or
other characters that would cause ambiguity in parsing the expression.

\par Example: \ttt{module ={\textasciitilde} "**.host*" AND (name ={\textasciitilde} "pkSent*" OR name ={\textasciitilde} "pkRecvd*")}

\par \textbf{The "raw" dataframe format}

\par This dataframe format is a central one, because the content of "raw" dataframes
correspond exactly to the content result files, i.e. it is possible to convert
between result files and the "raw" dataframe format without data loss. The "raw"
dataframe format also corresponds in a one-to-one manner to the "CSV-R" export
format of the Simulation IDE and \ttt{opp\_scavetool}.

\par The outputs of the \ttt{get\_results()} and \ttt{read\_result\_files()} functions are in
this format, and the dataframes that can be passed as input into certain query
functions (\ttt{get\_scalars()}, \ttt{get\_vectors()}, \ttt{get\_runs()}, etc.) are also
expected in the same format.

\par Columns of the \ttt{DataFrame}:

\begin{itemize}
  \item \ttt{runID} \textit{(string)}:  Identifies the simulation run
  \item \ttt{type} \textit{(string)}: Row type, one of the following: scalar, vector, statistics,
histogram, runattr, itervar, param, attr
  \item \ttt{module} \textit{(string)}: Hierarchical name (a.k.a. full path) of the module that
recorded the result item
  \item \ttt{name} \textit{(string)}: Name of the result item (scalar, statistic, histogram or
vector)
  \item \ttt{attrname} \textit{(string)}: Name of the run attribute or result item attribute (in
the latter case, the module and name columns identify the result item the
attribute belongs to)
  \item \ttt{attrvalue} \textit{(string)}: Value of run and result item attributes, iteration
variables, saved ini param settings (runattr, attr, itervar, param)
  \item \ttt{value} \textit{(double or string)}: Output scalar or parameter value
  \item \ttt{count}, \ttt{sumweights}, \ttt{mean}, \ttt{min}, \ttt{max}, \ttt{stddev} \textit{(double)}: Fields of the
statistics or histogram
  \item \ttt{binedges}, \ttt{binvalues} \textit{(np.array)}: Histogram bin edges and bin values.
\ttt{len(binedges)==len(binvalues)+1}
  \item \ttt{underflows}, \ttt{overflows} \textit{(double)}: Sum of weights (or counts) of underflown
and overflown samples of histograms
  \item \ttt{vectime}, \ttt{vecvalue} \textit{(np.array)}: Output vector time and value arrays

\end{itemize}


\par \textbf{Requesting metadata columns}

\par Several query functions have the \ttt{include\_attrs}, \ttt{include\_runattrs},
\ttt{include\_itervars}, \ttt{include\_param\_assignments}, and \ttt{include\_config\_entries}
boolean options. When such an option is turned on, it will add extra columns
into the returned \ttt{DataFrame}, one for each result attribute, run attribute,
iteration variable, etc. When there is a name clash among items of different
types, the column name for the second one will be modified by adding its type
after an underscore (\ttt{\_runattr}, \ttt{\_itervar}, \ttt{\_config}, \ttt{\_param}).

\begin{itemize}
  \item \ttt{include\_attrs}: Adds the attributes of the result in question
  \item \ttt{include\_runattrs}: Adds the run attributes of the (result's) run
  \item \ttt{include\_itervars}: Adds the iteration variables of the (result's) run
  \item \ttt{include\_config\_entries}: Adds all configuration entries of the (result's)
 run, including parameter parameter assignments and per-object configuration
 options. If this option is turned on, \ttt{include\_param\_assignments} has no
 effect.
  \item \ttt{include\_param\_assignments}: Adds the configuration entries that set module or
 channel parameters. This is a subset of the entries added by
 \ttt{include\_config\_entries}.

\end{itemize}


\par Note that values in metadata columns are generally strings (with missing values
represented as \ttt{None} or \ttt{nan}). The Pandas \ttt{to\_numeric()} function or
\ttt{utils.to\_numeric()} can be used to convert values to \ttt{float} or \ttt{int} where
needed.

\subsubsection{get\_serial()}
\label{cha:chart-api:omnetpp.scave.results:get-serial}

\begin{flushleft}
\ttt{get\_serial()}
\end{flushleft}

\par Returns an integer that is incremented every time the set of loaded results
change, typically as a result of the IDE loading, reloading or unloading
a scalar or vector result file. The serial can be used for invalidating
cached intermediate results when their input changes.

\subsubsection{set\_inputs()}
\label{cha:chart-api:omnetpp.scave.results:set-inputs}

\begin{flushleft}
\ttt{set\_inputs(filenames)}
\end{flushleft}

\par Specifies the set of simulation result files (.vec, .sca) to use as input
for the query functions. The argument may be a single string, or a list of
strings. Each string is interpreted as a file or directory path, and may
also contain wildcards. In addition to \ttt{?} and \ttt{*}, \ttt{**} (which is able to
match several directory levels) is also accepted as wildcard. If a path
corresponds to a directory, it is interpreted as \ttt{[ "<dir>/**/*.sca", "<dir>/**/*.vec" ]}, that is, all result files will be loaded from that
directory and recursively all its subdirectories.

\par Examples: \ttt{set\_inputs("results/")}, \ttt{set\_inputs("results/**.sca")},
\ttt{set\_inputs(["config1/*.sca", *config2/*.sca"])}.

\subsubsection{add\_inputs()}
\label{cha:chart-api:omnetpp.scave.results:add-inputs}

\begin{flushleft}
\ttt{add\_inputs(filenames)}
\end{flushleft}

\par Appends to the set of simulation result files (.vec, .sca) to use as input
for the query functions.  The argument may be a single string, or a list of
strings. Each string is interpreted as a file or directory path, and may
also contain wildcards (\ttt{?}, \ttt{*}, \ttt{**}). See \ttt{set\_inputs()} for more details.

\subsubsection{read\_result\_files()}
\label{cha:chart-api:omnetpp.scave.results:read-result-files}

\begin{flushleft}
\ttt{read\_result\_files(filenames, filter\_expression=None, include\_fields\_as\_scalars=False, vector\_start\_time=-inf, vector\_end\_time=inf)}
\end{flushleft}

\par Loads the simulation result files specified in the first argument
\ttt{filenames}, and returns the filtered set of results and metadata as a
Pandas \ttt{DataFrame}.

\par The \ttt{filenames} argument specifies the set of simulation result files (.vec,
.sca) to load.  The argument may be a single string, or a list of strings.
Each string is interpreted as a file or directory path, and may also contain
wildcards (\ttt{?}, \ttt{*}, \ttt{**}). See \ttt{set\_inputs()} for more details on this
format.

\par It is possible to limit the set of results to return by specifying a filter
expression, and vector start/end times.

\par Parameters:

\begin{itemize}
  \item \ttt{filenames} \textit{(string, or list of strings)}: Specifies the result files to
load.
  \item \ttt{filter\_expression} \textit{(string)}: The filter expression to select the desired
items to load. Example: \ttt{module ={\textasciitilde} "*host*" AND name ={\textasciitilde} "numPacket*"}
  \item \ttt{include\_fields\_as\_scalars} \textit{(bool)}: Optional. If \ttt{True}, the fields of
statistics and histograms (\ttt{:min}, \ttt{:mean}, etc.) are also returned as
synthetic scalars.
  \item \ttt{vector\_start\_time}, \ttt{vector\_end\_time} \textit{(double)}: Optional time limits to
trim the data of vector type results. The unit is seconds, the interval is
left-closed, right-open.

\end{itemize}


\par Returns: a \ttt{DataFrame} in the "raw" format (see the corresponding section of
the module documentation for details).

\subsubsection{get\_results()}
\label{cha:chart-api:omnetpp.scave.results:get-results}

\begin{flushleft}
\ttt{get\_results(filter\_or\_dataframe="", row\_types=None, omit\_unused\_columns=True, include\_fields\_as\_scalars=False, start\_time=-inf, end\_time=inf)}
\end{flushleft}

\par Returns a filtered set of results and metadata in a Pandas \ttt{DataFrame}. The
items can be any type, even mixed together in a single \ttt{DataFrame}. They are
selected from the complete set of data referenced by the analysis file
(\ttt{.anf}), including only those for which the given \ttt{filter\_or\_dataframe}
evaluates to \ttt{True}.

\par Parameters:

\begin{itemize}
  \item \ttt{filter\_or\_dataframe} \textit{(string or dataframe)}: The filter expression to
select the desired items from the inputs, or a dataframe in the "raw"
format. Example: \ttt{module ={\textasciitilde} "*host*" AND name ={\textasciitilde} "numPacket*"}
  \item \ttt{row\_types}: Optional. When given, filters the returned rows by type.
Should be a unique list, containing any number of these strings:
\ttt{"runattr"}, \ttt{"itervar"}, \ttt{"config"}, \ttt{"scalar"}, \ttt{"vector"},
\ttt{"statistic"}, \ttt{"histogram"}, \ttt{"param"}, \ttt{"attr"}
  \item \ttt{omit\_unused\_columns} \textit{(bool)}: Optional. If \ttt{True}, all columns that would
only contain \ttt{None} are removed from the returned DataFrame
  \item \ttt{include\_fields\_as\_scalars} \textit{(bool)}: Optional. If \ttt{True}, the fields of
statistics and histograms (\ttt{:min}, \ttt{:mean}, etc.) are also returned as
synthetic scalars.
  \item \ttt{start\_time}, \ttt{end\_time} \textit{(double)}: Optional time limits to trim the data
of vector type results. The unit is seconds, both the \ttt{vectime} and
\ttt{vecvalue} arrays will be affected, the interval is left-closed,
right-open.

\end{itemize}


\par Returns: a \ttt{DataFrame} in the "raw" format (see the corresponding section of
the module documentation for details).

\subsubsection{get\_runs()}
\label{cha:chart-api:omnetpp.scave.results:get-runs}

\begin{flushleft}
\ttt{get\_runs(filter\_or\_dataframe="", include\_runattrs=False, include\_itervars=False, include\_param\_assignments=False, include\_config\_entries=False)}
\end{flushleft}

\par Returns a filtered list of runs, identified by their run ID.

\par Parameters:

\begin{itemize}
  \item \ttt{filter\_or\_dataframe} \textit{(string or dataframe)}: The filter expression to
select the desired run from the inputs, or a dataframe in the "raw" format
(e.g. one returned by \ttt{read\_result\_files()}). Example: \ttt{runattr:network ={\textasciitilde} "Aloha" AND config:Aloha.slotTime ={\textasciitilde} 0}
  \item \ttt{include\_runattrs}, \ttt{include\_itervars}, \ttt{include\_param\_assignments},
\ttt{include\_config\_entries} \textit{(bool)}: Optional. When set to \ttt{True}, additional
pieces of metadata about the run is appended to the result, pivoted into
columns. See the "Metadata columns" section of the module documentation
for details.

\end{itemize}


\par Columns of the returned DataFrame:

\begin{itemize}
  \item \ttt{runID} \textit{(string)}: Identifies the simulation run
  \item Additional metadata items (run attributes, iteration variables, etc.), as
requested

\end{itemize}

\subsubsection{get\_runattrs()}
\label{cha:chart-api:omnetpp.scave.results:get-runattrs}

\begin{flushleft}
\ttt{get\_runattrs(filter\_or\_dataframe="", include\_runattrs=False, include\_itervars=False, include\_param\_assignments=False, include\_config\_entries=False)}
\end{flushleft}

\par Returns a filtered list of run attributes.

\par The set of run attributes is fixed: \ttt{configname}, \ttt{datetime}, \ttt{experiment},
\ttt{inifile}, \ttt{iterationvars}, \ttt{iterationvarsf}, \ttt{measurement}, \ttt{network},
\ttt{processid}, \ttt{repetition}, \ttt{replication}, \ttt{resultdir}, \ttt{runnumber},
\ttt{seedset}.

\par Parameters:

\begin{itemize}
  \item \ttt{filter\_or\_dataframe} \textit{(string or dataframe)}: The filter expression to
select the desired run attributes from the inputs, or a dataframe in the
"raw" format. Example: \ttt{name ={\textasciitilde} *date* AND config:Aloha.slotTime ={\textasciitilde} 0}
  \item \ttt{include\_runattrs}, \ttt{include\_itervars}, \ttt{include\_param\_assignments},
\ttt{include\_config\_entries} \textit{(bool)}: Optional. When set to \ttt{True}, additional
pieces of metadata about the run is appended to the result, pivoted into
columns. See the "Metadata columns" section of the module documentation
for details.

\end{itemize}


\par Columns of the returned DataFrame:

\begin{itemize}
  \item \ttt{runID} \textit{(string)}: Identifies the simulation run
  \item \ttt{name} \textit{(string)}: The name of the run attribute
  \item \ttt{value} \textit{(string)}: The value of the run attribute
  \item Additional metadata items (run attributes, iteration variables, etc.)

\end{itemize}

\subsubsection{get\_itervars()}
\label{cha:chart-api:omnetpp.scave.results:get-itervars}

\begin{flushleft}
\ttt{get\_itervars(filter\_or\_dataframe="", include\_runattrs=False, include\_itervars=False, include\_param\_assignments=False, include\_config\_entries=False)}
\end{flushleft}

\par Returns a filtered list of iteration variables.

\par Parameters:

\begin{itemize}
  \item \ttt{filter\_or\_dataframe} \textit{(string or dataframe)}: The filter expression to
select the desired iteration variables from the inputs, or a dataframe in
the "raw" format. Example: \ttt{name ={\textasciitilde} iaMean AND config:Aloha.slotTime ={\textasciitilde} 0}
  \item \ttt{include\_runattrs}, \ttt{include\_itervars}, \ttt{include\_param\_assignments},
\ttt{include\_config\_entries} \textit{(bool)}: Optional. When set to \ttt{True}, additional
pieces of metadata about the run is appended to the result, pivoted into
columns. See the "Metadata columns" section of the module documentation
for details.

\end{itemize}


\par Columns of the returned DataFrame:

\begin{itemize}
  \item \ttt{runID} \textit{(string)}: Identifies the simulation run
  \item \ttt{name} \textit{(string)}: The name of the iteration variable
  \item \ttt{value} \textit{(string)}: The value of the iteration variable.
  \item Additional metadata items (run attributes, iteration variables, etc.), as
requested

\end{itemize}

\subsubsection{get\_scalars()}
\label{cha:chart-api:omnetpp.scave.results:get-scalars}

\begin{flushleft}
\ttt{get\_scalars(filter\_or\_dataframe="", include\_attrs=False, include\_fields=False, include\_runattrs=False, include\_itervars=False, include\_param\_assignments=False, include\_config\_entries=False)}
\end{flushleft}

\par Returns a filtered list of scalar results.

\par Parameters:

\begin{itemize}
  \item \ttt{filter\_or\_dataframe} \textit{(string)}: The filter expression to select the
desired scalars, or a dataframe in the "raw" format. Example: \ttt{name ={\textasciitilde} "channelUtilization*" AND runattr:replication ={\textasciitilde} "\#0"}
  \item \ttt{include\_attrs} \textit{(bool)}: Optional. When set to \ttt{True}, result attributes
(like \ttt{unit} or \ttt{source} for example) are appended to the DataFrame,
pivoted into columns.
  \item \ttt{include\_fields} \textit{(bool)}: Optional. If \ttt{True}, the fields of statistics and
histograms (\ttt{:min}, \ttt{:mean}, etc.) are also returned as synthetic scalars.
  \item \ttt{include\_runattrs}, \ttt{include\_itervars}, \ttt{include\_param\_assignments},
\ttt{include\_config\_entries} \textit{(bool)}: Optional. When set to \ttt{True}, additional
pieces of metadata about the run is appended to the DataFrame, pivoted
into columns. See the "Metadata columns" section of the module
documentation for details.

\end{itemize}


\par Columns of the returned DataFrame:

\begin{itemize}
  \item \ttt{runID} \textit{(string)}: Identifies the simulation run
  \item \ttt{module} \textit{(string)}: Hierarchical name (a.k.a. full path) of the module that
recorded the result item
  \item \ttt{name} \textit{(string)}: The name of the scalar
  \item \ttt{value} \textit{(double)}: The value of the scalar
  \item Additional metadata items (result attributes, run attributes, iteration
variables, etc.), as requested

\end{itemize}

\subsubsection{get\_parameters()}
\label{cha:chart-api:omnetpp.scave.results:get-parameters}

\begin{flushleft}
\ttt{get\_parameters(filter\_or\_dataframe="", include\_attrs=False, include\_runattrs=False, include\_itervars=False, include\_param\_assignments=False, include\_config\_entries=False)}
\end{flushleft}

\par Returns a filtered list of parameters - actually computed values of
individual \ttt{cPar} instances in the fully built network.

\par Parameters are considered "pseudo-results", similar to scalars - except
their values are strings. Even though they act mostly as input to the actual
simulation run, the actually assigned value of individual \ttt{cPar} instances
is valuable information, as it is the result of the network setup process.
For example, even if a parameter is set up as an expression like \ttt{normal(3, 0.4)} from \ttt{omnetpp.ini}, the returned DataFrame will contain the single
concrete value picked for every instance of the parameter.

\par Parameters:

\begin{itemize}
  \item \ttt{filter\_or\_dataframe} \textit{(string)}: The filter expression to select the
desired parameters, or a dataframe in the "raw" format. Example: \ttt{name ={\textasciitilde} "x" AND module ={\textasciitilde} Aloha.server}
  \item \ttt{include\_attrs} \textit{(bool)}: Optional. When set to \ttt{True}, result attributes
(like \ttt{unit} for example) are appended to the DataFrame, pivoted into
columns.
  \item \ttt{include\_runattrs}, \ttt{include\_itervars}, \ttt{include\_param\_assignments},
\ttt{include\_config\_entries} \textit{(bool)}: Optional. When set to \ttt{True}, additional
pieces of metadata about the run is appended to the DataFrame, pivoted
into columns. See the "Metadata columns" section of the module
documentation for details.

\end{itemize}


\par Columns of the returned DataFrame:

\begin{itemize}
  \item \ttt{runID} \textit{(string)}: Identifies the simulation run
  \item \ttt{module} \textit{(string)}: Hierarchical name (a.k.a. full path) of the module that
recorded the result item
  \item \ttt{name} \textit{(string)}: The name of the parameter
  \item \ttt{value} \textit{(string)}: The value of the parameter.
  \item Additional metadata items (result attributes, run attributes, iteration
variables, etc.), as requested

\end{itemize}

\subsubsection{get\_vectors()}
\label{cha:chart-api:omnetpp.scave.results:get-vectors}

\begin{flushleft}
\ttt{get\_vectors(filter\_or\_dataframe="", include\_attrs=False, include\_runattrs=False, include\_itervars=False, include\_param\_assignments=False, include\_config\_entries=False, start\_time=-inf, end\_time=inf)}
\end{flushleft}

\par Returns a filtered list of vector results.

\par Parameters:

\begin{itemize}
  \item \ttt{filter\_or\_dataframe} \textit{(string)}: The filter expression to select the
desired vectors, or a dataframe in the "raw" format. Example: \ttt{name ={\textasciitilde} "radioState*" AND runattr:replication ={\textasciitilde} "\#0"}
  \item \ttt{include\_attrs} \textit{(bool)}: Optional. When set to \ttt{True}, result attributes
(like \ttt{unit} or \ttt{source} for example) are appended to the DataFrame,
pivoted into columns.
  \item \ttt{include\_runattrs}, \ttt{include\_itervars}, \ttt{include\_param\_assignments},
\ttt{include\_config\_entries} \textit{(bool)}: Optional. When set to \ttt{True}, additional
pieces of metadata about the run is appended to the DataFrame, pivoted
into columns. See the "Metadata columns" section of the module
documentation for details.
  \item \ttt{start\_time}, \ttt{end\_time} \textit{(double)}: Optional time limits to trim the data
of vector type results. The unit is seconds, both the \ttt{vectime} and
\ttt{vecvalue} arrays will be affected, the interval is left-closed,
right-open.

\end{itemize}


\par Columns of the returned DataFrame:

\begin{itemize}
  \item \ttt{runID} \textit{(string)}: Identifies the simulation run
  \item \ttt{module} \textit{(string)}: Hierarchical name (a.k.a. full path) of the module that
recorded the result item
  \item \ttt{name} \textit{(string)}: The name of the vector
  \item \ttt{vectime}, \ttt{vecvalue} \textit{(np.array)}: The simulation times and the
corresponding values in the vector
  \item Additional metadata items (result attributes, run attributes, iteration
variables, etc.), as requested

\end{itemize}

\subsubsection{get\_statistics()}
\label{cha:chart-api:omnetpp.scave.results:get-statistics}

\begin{flushleft}
\ttt{get\_statistics(filter\_or\_dataframe="", include\_attrs=False, include\_runattrs=False, include\_itervars=False, include\_param\_assignments=False, include\_config\_entries=False)}
\end{flushleft}

\par Returns a filtered list of statistics results.

\par Parameters:

\begin{itemize}
  \item \ttt{filter\_or\_dataframe} \textit{(string)}: The filter expression to select the
desired statistics, or a dataframe in the "raw" format. Example: \ttt{name ={\textasciitilde} "collisionLength:stat" AND itervar:iaMean ={\textasciitilde} "5"}
  \item \ttt{include\_attrs} \textit{(bool)}: Optional. When set to \ttt{True}, result attributes
(like \ttt{unit} or \ttt{source} for example) are appended to the DataFrame,
pivoted into columns.
  \item \ttt{include\_runattrs}, \ttt{include\_itervars}, \ttt{include\_param\_assignments},
\ttt{include\_config\_entries} \textit{(bool)}: Optional. When set to \ttt{True}, additional
pieces of metadata about the run is appended to the DataFrame, pivoted
into columns. See the "Metadata columns" section of the module
documentation for details.

\end{itemize}


\par Columns of the returned DataFrame:

\begin{itemize}
  \item \ttt{runID} \textit{(string)}: Identifies the simulation run
  \item \ttt{module} \textit{(string)}: Hierarchical name (a.k.a. full path) of the module that
recorded the result item
  \item \ttt{name} \textit{(string)}: The name of the vector
  \item \ttt{count}, \ttt{sumweights}, \ttt{mean}, \ttt{stddev}, \ttt{min}, \ttt{max} \textit{(double)}: The
characteristic mathematical properties of the statistics result
  \item Additional metadata items (result attributes, run attributes, iteration
variables, etc.), as requested

\end{itemize}

\subsubsection{get\_histograms()}
\label{cha:chart-api:omnetpp.scave.results:get-histograms}

\begin{flushleft}
\ttt{get\_histograms(filter\_or\_dataframe="", include\_attrs=False, include\_runattrs=False, include\_itervars=False, include\_param\_assignments=False, include\_config\_entries=False)}
\end{flushleft}

\par Returns a filtered list of histogram results.

\par Parameters:

\begin{itemize}
  \item \ttt{filter\_or\_dataframe} \textit{(string)}: The filter expression to select the
desired histograms, or a dataframe in the "raw" format. Example: \ttt{name ={\textasciitilde} "collisionMultiplicity:histogram" AND itervar:iaMean ={\textasciitilde} "2"}
  \item \ttt{include\_attrs} \textit{(bool)}: Optional. When set to \ttt{True}, result attributes
(like \ttt{unit} or \ttt{source} for example) are appended to the DataFrame,
pivoted into columns.
  \item \ttt{include\_runattrs}, \ttt{include\_itervars}, \ttt{include\_param\_assignments},
\ttt{include\_config\_entries} \textit{(bool)}: Optional. When set to \ttt{True}, additional
pieces of metadata about the run is appended to the DataFrame, pivoted
into columns. See the "Metadata columns" section of the module
documentation for details.

\end{itemize}


\par Columns of the returned DataFrame:

\begin{itemize}
  \item \ttt{runID} \textit{(string)}: Identifies the simulation run
  \item \ttt{module} \textit{(string)}: Hierarchical name (a.k.a. full path) of the module that
recorded the result item
  \item \ttt{name} \textit{(string)}: The name of the vector
  \item \ttt{count}, \ttt{sumweights}, \ttt{mean}, \ttt{stddev}, \ttt{min}, \ttt{max} \textit{(double)}: The
characteristic mathematical properties of the histogram
  \item \ttt{binedges}, \ttt{binvalues} \textit{(np.array)}: The histogram edge locations and the
weighted sum of the collected samples in each bin. \ttt{len(binedges) == len(binvalues) + 1}
  \item \ttt{underflows}, \ttt{overflows} \textit{(double)}: The weighted sum of the samples that
fell outside of the histogram bin range in the two directions
  \item Additional metadata items (result attributes, run attributes, iteration
variables, etc.), as requested

\end{itemize}

\subsubsection{get\_config\_entries()}
\label{cha:chart-api:omnetpp.scave.results:get-config-entries}

\begin{flushleft}
\ttt{get\_config\_entries(filter\_or\_dataframe, include\_runattrs=False, include\_itervars=False, include\_param\_assignments=False, include\_config\_entries=False)}
\end{flushleft}

\par Returns a filtered list of config entries. That is: parameter assignment
patterns; and global and per-object config options.

\par Parameters:

\begin{itemize}
  \item \ttt{filter\_or\_dataframe} \textit{(string)}: The filter expression to select the
desired config entries, or a dataframe in the "raw" format. Example: \ttt{name ={\textasciitilde} sim-time-limit AND itervar:numHosts ={\textasciitilde} 10}
  \item \ttt{include\_runattrs}, \ttt{include\_itervars}, \ttt{include\_param\_assignments},
\ttt{include\_config\_entries} \textit{(bool)}: Optional. When set to \ttt{True}, additional
pieces of metadata about the run is appended to the result, pivoted into
columns. See the "Metadata columns" section of the module documentation
for details.

\end{itemize}


\par Columns of the returned DataFrame:

\begin{itemize}
  \item \ttt{runID} \textit{(string)}: Identifies the simulation run
  \item \ttt{name} \textit{(string)}: The name of the config entry
  \item \ttt{value} \textit{(string)}: The value of the config entry
  \item Additional metadata items (run attributes, iteration variables, etc.), as
requested

\end{itemize}

\subsubsection{get\_param\_assignments()}
\label{cha:chart-api:omnetpp.scave.results:get-param-assignments}

\begin{flushleft}
\ttt{get\_param\_assignments(filter\_or\_dataframe, include\_runattrs=False, include\_itervars=False, include\_param\_assignments=False, include\_config\_entries=False)}
\end{flushleft}

\par Returns a filtered list of parameter assignment patterns. The result is a
subset of what \ttt{get\_config\_entries} would return with the same arguments.

\par Parameters:

\begin{itemize}
  \item \ttt{filter\_or\_dataframe} \textit{(string)}: The filter expression to select the
desired parameter assignments, or a dataframe in the "raw" format.
Example: \ttt{name ={\textasciitilde} **.flowID AND itervar:numHosts ={\textasciitilde} 10}
  \item \ttt{include\_runattrs}, \ttt{include\_itervars}, \ttt{include\_param\_assignments},
\ttt{include\_config\_entries} \textit{(bool)}: Optional. When set to \ttt{True}, additional
pieces of metadata about the run is appended to the result, pivoted into
columns. See the "Metadata columns" section of the module documentation
for details.

\end{itemize}


\par Columns of the returned DataFrame:

\begin{itemize}
  \item \ttt{runID} \textit{(string)}: Identifies the simulation run
  \item \ttt{name} \textit{(string)}: The parameter assignment pattern
  \item \ttt{value} \textit{(string)}: The assigned value
  \item Additional metadata items (run attributes, iteration variables, etc.), as
requested

\end{itemize}

\subsection{Module omnetpp.scave.chart}
\label{cha:chart-api:omnetpp.scave.chart}

\par Provides access to the properties of the current chart for the chart script.

\par Note that this module is stateful. It is set up appropriately by the OMNeT++
IDE or \ttt{opp\_charttool} before the chart script is run.

\subsubsection{Class ChartScriptError}
\label{cha:chart-api:omnetpp.scave.chart:ChartScriptError}

\par Raised by chart scripts when they encounter an error.
A message parameter can be passed to the constructor,
which will be displayed on the plot area in the IDE.

\subsubsection{get\_properties()}
\label{cha:chart-api:omnetpp.scave.chart:get-properties}

\begin{flushleft}
\ttt{get\_properties()}
\end{flushleft}

\par Returns the currently set properties of the chart as a \ttt{dict}
whose keys and values are both all strings.

\subsubsection{get\_property()}
\label{cha:chart-api:omnetpp.scave.chart:get-property}

\begin{flushleft}
\ttt{get\_property(key)}
\end{flushleft}

\par Returns the value of a single property of the chart, or \ttt{None} if there is
no property with the given name (key) set on the chart.

\subsubsection{get\_name()}
\label{cha:chart-api:omnetpp.scave.chart:get-name}

\begin{flushleft}
\ttt{get\_name()}
\end{flushleft}

\par Returns the name of the chart as a string.

\subsubsection{get\_chart\_type()}
\label{cha:chart-api:omnetpp.scave.chart:get-chart-type}

\begin{flushleft}
\ttt{get\_chart\_type()}
\end{flushleft}

\par Returns the chart type, one of the strings "bar"/"histogram"/"line"/"matplotlib"

\subsubsection{is\_native\_chart()}
\label{cha:chart-api:omnetpp.scave.chart:is-native-chart}

\begin{flushleft}
\ttt{is\_native\_chart()}
\end{flushleft}

\par Returns True if this chart uses the IDE's built-in plotting widgets.

\subsubsection{set\_suggested\_chart\_name()}
\label{cha:chart-api:omnetpp.scave.chart:set-suggested-chart-name}

\begin{flushleft}
\ttt{set\_suggested\_chart\_name(name)}
\end{flushleft}

\par Sets a proposed name for the chart. The IDE may offer this name to the user
when saving the chart.

\subsubsection{set\_observed\_column\_names()}
\label{cha:chart-api:omnetpp.scave.chart:set-observed-column-names}

\begin{flushleft}
\ttt{set\_observed\_column\_names(column\_names)}
\end{flushleft}

\par Sets the DataFrame column names observed during the chart script.
The IDE may use it for content assist when the user edits the legend format string.

\subsection{Module omnetpp.scave.ideplot}
\label{cha:chart-api:omnetpp.scave.ideplot}

\par This module is the interface for displaying plots using the IDE's native
(non-Matplotlib) plotting widgets from chart scripts. The API is intentionally
very close to \ttt{matplotlib.pyplot}: most functions and the parameters they accept
are a subset of \ttt{pyplot}'s interface. If one restricts themselves to a subset of
Matplotlib's functionality, switching between \ttt{omnetpp.scave.ideplot} and
\ttt{matplotlib.pyplot} in a chart script may be as simple as much as editing the
\ttt{import} statement.

\par When the API is used outside the context of a native plotting widget (such as
during the run of \ttt{opp\_charttool}, or in IDE during image export), the functions
are emulated with Matplotlib.

\par Note that this module is stateful. It is set up appropriately by the OMNeT++ IDE
or \ttt{opp\_charttool} before the chart script is run.

\subsubsection{is\_native\_plot()}
\label{cha:chart-api:omnetpp.scave.ideplot:is-native-plot}

\begin{flushleft}
\ttt{is\_native\_plot()}
\end{flushleft}

\par Returns True if the script is running in the context of a native plotting
widget, and False otherwise.

\subsubsection{plot()}
\label{cha:chart-api:omnetpp.scave.ideplot:plot}

\begin{flushleft}
\ttt{plot(xs, ys, key=None, label=None, drawstyle=None, linestyle=None, linewidth=None, color=None, marker=None, markersize=None)}
\end{flushleft}

\par Plot y versus x as lines and/or markers. Call \ttt{plot} multiple times
to plot multiple sets of data.

\par Parameters:

\begin{itemize}
  \item \ttt{x}, \ttt{y} \textit{(array-like or scalar)}: The horizontal / vertical coordinates of the data points.
  \item \ttt{key} \textit{(string)}: Identifies the series in the native plot widget.
  \item \ttt{label} \textit{(string)}: Series label for the legend
  \item \ttt{drawstyle} \textit{(string)}: Matplotlib draw style ('default', 'steps', 'steps-pre', 'steps-mid', 'steps-post')
  \item \ttt{linestyle} \textit{(string)}: Matplotlib line style ('-', '--', '-.', ':', etc\}
  \item \ttt{linewidth} \textit{(float)}: Line width in pixels
  \item \ttt{color} \textit{(string)}: Matplotlib color name or abbreviation ('b' for blue, 'g' for green, etc.)
  \item \ttt{marker} \textit{(string)}: Matplotlib marker name ('.', ',', 'o', 'x', '+', etc.)
  \item \ttt{markersize} \textit{(float)}: Size of markers in pixels.

\end{itemize}

\subsubsection{hist()}
\label{cha:chart-api:omnetpp.scave.ideplot:hist}

\begin{flushleft}
\ttt{hist(x, bins, key=None, density=False, weights=None, cumulative=False, bottom=None, histtype="stepfilled", color=None, label=None, linewidth=None, underflows=0.0, overflows=0.0, minvalue=nan, maxvalue=nan)}
\end{flushleft}

\par Make a histogram plot. This function adds one histogram the bar plot; make
multiple calls to add multiple histograms.

\par Parameters:

\begin{itemize}
  \item \ttt{x} \textit{(array-like)}: Input values.
  \item \ttt{bins} \textit{(array-like)}: Bin edges, including left edge of first bin and right edge of last bin.
  \item \ttt{key} \textit{(string)}: Identifies the series in the native plot widget.
  \item \ttt{density} \textit{(bool)}: See \ttt{mpl.hist()}.
  \item \ttt{weights} \textit{(array-like)}: Weights.
  \item \ttt{cumulative} \textit{(bool)}: See \ttt{mpl.hist()}.
  \item \ttt{bottom} \textit{(float)}: Location of the bottom baseline for bins.
  \item \ttt{histtype} \textit{(string)}: Whether to fill the area under the plot. Accepted values are 'step' and 'stepfilled'.
  \item \ttt{color} \textit{(string)}: Matplotlib color name or abbreviation ('b' for blue, 'g' for green, etc.)
  \item \ttt{label} \textit{(string)}: Series label for the legend
  \item \ttt{linewidth} \textit{(float)}: Line width in pixels
  \item \ttt{underflows}, \ttt{overflows}: Number of values / sum of weights outside the histogram bins in both directions.
  \item \ttt{minvalue}, \ttt{maxvalue}: The minimum and maximum value, or \ttt{nan} if unknown.

\end{itemize}


\par Restrictions:

\begin{enumerate}
  \item Overflow bin data (minvalue, maxvalue, underflows and overflows) is not accepted by \ttt{pyplot.hist()}.
  \item The native plot widget only accepts a precomputed histogram (using the trick documented for \ttt{pyplot.hist()})

\end{enumerate}

\subsubsection{bar()}
\label{cha:chart-api:omnetpp.scave.ideplot:bar}

\begin{flushleft}
\ttt{bar(x, height, width=0.8, key=None, label=None, color=None, edgecolor=None)}
\end{flushleft}

\par Make a bar plot. This function adds one series to the bar plot; make
multiple calls to add multiple series.

\par The bars are positioned at x with the given alignment. Their dimensions
are given by width and height. The vertical baseline is bottom (default 0).

\par Each of x, height, width, and bottom may either be a scalar applying to
all bars, or it may be a sequence of length N providing a separate value
for each bar.

\par Parameters:

\begin{itemize}
  \item \ttt{x} \textit{(sequence of scalars)}: The x coordinates of the bars.
  \item \ttt{height} \textit{(scalar or sequence of scalars)}: The height(s) of the bars.
  \item \ttt{width} \textit{(scalar or array-like)}: The width(s) of the bars.
  \item \ttt{key} \textit{(string)}: Identifies the series in the native plot widget.
  \item \ttt{label} \textit{(string)}: The label of the series the bars represent .
  \item \ttt{color} \textit{(string)}: The fill color of the bars.
  \item \ttt{edgecolor} \textit{(string)}: The edge color of the bars.

\end{itemize}


\par The native plot implementation has the following restrictions:

\begin{itemize}
  \item widths are automatic (parameter is ignored)
  \item x coordinates are automatic (values are ignored)
  \item height must be a sequence (cannot be a scalar)
  \item in multiple calls to bar(), the lengths of the height sequence must be
equal (i.e. all series must have the same number of values)
  \item default color is grey (Matplotlib assigns a different color to each series)

\end{itemize}

\subsubsection{set\_property()}
\label{cha:chart-api:omnetpp.scave.ideplot:set-property}

\begin{flushleft}
\ttt{set\_property(key, value)}
\end{flushleft}

\par Sets one property of the native plot widget to the given value. When invoked
outside the contex of a native plot widget, the function does nothing.

\par Parameters:

\begin{itemize}
  \item \ttt{key} \textit{(string)}: Name of the property.
  \item \ttt{value} \textit{(string)}: The value to set. If any other type than string is passed in, it will be converted to string.

\end{itemize}

\subsubsection{set\_properties()}
\label{cha:chart-api:omnetpp.scave.ideplot:set-properties}

\begin{flushleft}
\ttt{set\_properties(props)}
\end{flushleft}

\par Sets several properties of the native plot widget. It is functionally equivalent to
repeatedly calling \ttt{set\_property} with the entries of the \ttt{props} dictionary.
When invoked outside the contex of a native plot widget (TODO?), the function does nothing.

\par Parameters:

\begin{itemize}
  \item \ttt{props} \textit{(dict)}: The properties to set.

\end{itemize}

\subsubsection{get\_supported\_property\_keys()}
\label{cha:chart-api:omnetpp.scave.ideplot:get-supported-property-keys}

\begin{flushleft}
\ttt{get\_supported\_property\_keys()}
\end{flushleft}

\par Returns the list of property names that the native plot widget supports, such as
'Plot.Title', 'X.Axis.Max' and 'Legend.Display', among many others.

\par Note: This method has no equivalent in \ttt{pyplot}. When the script runs outside the IDE  (TODO?),
the method returns an empty list.

\subsubsection{set\_warning()}
\label{cha:chart-api:omnetpp.scave.ideplot:set-warning}

\begin{flushleft}
\ttt{set\_warning(warning: str)}
\end{flushleft}

\par Displays the given warning text in the plot.

\subsubsection{title()}
\label{cha:chart-api:omnetpp.scave.ideplot:title}

\begin{flushleft}
\ttt{title(label: str)}
\end{flushleft}

\par Sets the plot title to the given string.

\subsubsection{xlabel()}
\label{cha:chart-api:omnetpp.scave.ideplot:xlabel}

\begin{flushleft}
\ttt{xlabel(xlabel: str)}
\end{flushleft}

\par Sets the label of the X axis to the given string.

\subsubsection{ylabel()}
\label{cha:chart-api:omnetpp.scave.ideplot:ylabel}

\begin{flushleft}
\ttt{ylabel(ylabel: str)}
\end{flushleft}

\par Sets the label of the Y axis to the given string..

\subsubsection{xlim()}
\label{cha:chart-api:omnetpp.scave.ideplot:xlim}

\begin{flushleft}
\ttt{xlim(left=None, right=None)}
\end{flushleft}

\par Sets the limits of the X axis.

\par Parameters:

\begin{itemize}
  \item \ttt{left} \textit{(float)}: The left xlim in data coordinates. Passing None leaves the limit unchanged.
  \item \ttt{right} \textit{(float)}: The right xlim in data coordinates. Passing None leaves the limit unchanged.

\end{itemize}

\subsubsection{ylim()}
\label{cha:chart-api:omnetpp.scave.ideplot:ylim}

\begin{flushleft}
\ttt{ylim(bottom=None, top=None)}
\end{flushleft}

\par Sets the limits of the Y axis.

\par Parameters:

\begin{itemize}
  \item \ttt{bottom} \textit{(float)}: The bottom ylim in data coordinates. Passing None leaves the limit unchanged.
  \item \ttt{top} \textit{(float)}: The top ylim in data coordinates. Passing None leaves the limit unchanged.

\end{itemize}

\subsubsection{xscale()}
\label{cha:chart-api:omnetpp.scave.ideplot:xscale}

\begin{flushleft}
\ttt{xscale(value: str)}
\end{flushleft}

\par Sets the scale of the X axis. Possible values are 'linear' and 'log'.

\subsubsection{yscale()}
\label{cha:chart-api:omnetpp.scave.ideplot:yscale}

\begin{flushleft}
\ttt{yscale(value: str)}
\end{flushleft}

\par Sets the scale of the Y axis.

\subsubsection{xticks()}
\label{cha:chart-api:omnetpp.scave.ideplot:xticks}

\begin{flushleft}
\ttt{xticks(ticks=None, labels=None, rotation=0)}
\end{flushleft}

\par Sets the current tick locations and labels of the x-axis.

\par Parameters:

\begin{itemize}
  \item \ttt{ticks} \textit{(array\_like)}: A list of positions at which ticks should be placed. You can pass an empty list to disable xticks.
  \item \ttt{labels} \textit{(array\_like)}: A list of explicit labels to place at the given locs.
  \item \ttt{rotation} \textit{(float)}: Label rotation in degrees.

\end{itemize}

\subsubsection{grid()}
\label{cha:chart-api:omnetpp.scave.ideplot:grid}

\begin{flushleft}
\ttt{grid(b=True, which="major")}
\end{flushleft}

\par Configure the grid lines.

\par Parameters:

\begin{itemize}
  \item \ttt{b} \textit{(bool or \ttt{None})}: Whether to show the grid lines.
  \item \ttt{which} \textit{('major', 'minor' or 'both')}: The grid lines to apply the changes on.

\end{itemize}

\subsubsection{legend()}
\label{cha:chart-api:omnetpp.scave.ideplot:legend}

\begin{flushleft}
\ttt{legend(show=None, frameon=None, loc=None)}
\end{flushleft}

\par Place a legend on the axes.

\par Parameters:

\begin{itemize}
  \item \ttt{show} \textit{(bool or \ttt{None})}: Whether to show the legend. TODO does pyplot have this?
  \item \ttt{frameon} \textit{(bool or \ttt{None})}: Control whether the legend should be drawn on a patch (frame).
  Default is \ttt{None}, which will take the value from the resource file.
  \item \ttt{loc} \textit{(string or \ttt{None})}: The location of the legend. Possible values are
  'best', 'upper right', 'upper left', 'lower left', 'lower right', 'right',
  'center left', 'center right', 'lower center', 'upper center', 'center'
  (these are the values supported by Matplotlib), plus additionally
  'outside top left', 'outside top center', 'outside top right',
  'outside bottom left', 'outside bottom center', 'outside bottom right',
  'outside left top', 'outside left center', 'outside left bottom',
  'outside right top', 'outside right center', 'outside right bottom'.

\end{itemize}

\subsection{Module omnetpp.scave.utils}
\label{cha:chart-api:omnetpp.scave.utils}

\par A collection of utility function for data manipulation and plotting, built
on top of Pandas data frames and the \ttt{chart} and \ttt{ideplot} packages from
\ttt{omnetpp.scave}. Functions in this module have been written largely to the
needs of the chart templates that ship with the IDE.

\par There are some functions which are (almost) mandatory elements in a chart script.
These are the following.

\par If you want style settings in the chart dialog to take effect:

\begin{itemize}
  \item \ttt{preconfigure\_plot()}
  \item \ttt{postconfigure\_plot()}

\end{itemize}


\par If you want image/data export to work:

\begin{itemize}
  \item \ttt{export\_image\_if\_needed()}
  \item \ttt{export\_data\_if\_needed()}

\end{itemize}

\subsubsection{make\_legend\_label()}
\label{cha:chart-api:omnetpp.scave.utils:make-legend-label}

\begin{flushleft}
\ttt{make\_legend\_label(legend\_cols, row, props=\{\})}
\end{flushleft}

\par Produces a reasonably good label text (to be used in a chart legend) for a result row from
a DataFrame. The legend label is produced as follows.

\par First, a base version of the legend label is produced:

\begin{enumerate}
  \item If the DataFrame contains a \ttt{legend} column, its content is used
  \item Otherwise, if there is a \ttt{legend\_format} property, it is used as a format string
for producing the legend label. The format string may contain references to other
columns of the DataFrame in the "\$name" or "\$\{name\}" form.
  \item Otherwise, the legend label is concatenated from the columns listed in \ttt{legend\_cols},
a list whose contents is usually produced using the \ttt{extract\_label\_columns()} function.

\end{enumerate}


\par Second, if there is a \ttt{legend\_replacements} property, the series of regular expression
find/replace operations described in it are performed. \ttt{legend\_replacements} is expected
to be a multi-line string, where each line contains a replacement in the customary
"/findstring/replacement/" form. "findstring" should be a valid regex, and "replacement"
is a string that may contain match group references ("{\textbackslash}1", "{\textbackslash}2", etc.). If "/" is unsuitable
as separator, other characters may also be used; common choices are "|" and "!".
Similar to the format string (\ttt{legend\_format}), both "findstring" and "replacement"
may contain column references in the "\$name" or "\$\{name\}" form. (Note that "findstring"
may still end in "\$" to match the end of the string, it won't collide with column references.)

\par Possible errors:

\begin{itemize}
  \item References to nonexistent columns in \ttt{legend\_format} or \ttt{legend\_replacements} (\ttt{KeyError})
  \item Malformed regex in the "findstring" parts of \ttt{legend\_replacements} (\ttt{re.error})
  \item Invalid group reference in the "replacement" parts of \ttt{legend\_replacements} (\ttt{re.error})

\end{itemize}


\par Parameters:

\begin{itemize}
  \item \ttt{row} \textit{(named tuple)}: The row from the dataframe.
  \item \ttt{props} \textit{(dict)}: The properties that control how the legend is produced
  \item \ttt{legend\_cols} \textit{(list of strings)}: The names of columns chosen for the legend.

\end{itemize}

\subsubsection{plot\_bars()}
\label{cha:chart-api:omnetpp.scave.utils:plot-bars}

\begin{flushleft}
\ttt{plot\_bars(df, errors\_df=None, meta\_df=None, props=\{\})}
\end{flushleft}

\par Creates a bar plot from the dataframe, with styling and additional input
coming from the properties. Each row in the dataframe defines a series.

\par Group names (displayed on the x axis) are taken from the column index.

\par The name of the variable represented by the values can be passed in as
the \ttt{variable\_name} argument (as it is not present in the dataframe); if so,
it will become the y axis label.

\par Error bars can be drawn by providing an extra dataframe of identical
dimensions as the main one. Error bars will protrude by the values in the
errors dataframe both up and down (i.e. range is 2x error).

\par To make the legend labels customizable, an extra dataframe can be provided,
which contains any columns of metadata for each series.

\par Colors are assigned automatically. The \ttt{cycle\_seed} property allows you
to select other combinations if the default one is not suitable.

\par Parameters:

\begin{itemize}
  \item \ttt{df}: the dataframe
  \item \ttt{props} \textit{(dict)}: the properties
  \item \ttt{variable\_name} \textit{(string)}: The name of the variable represented by the values.
  \item \ttt{errors\_df}: dataframe with the errors (in y axis units)
  \item \ttt{meta\_df}: dataframe with the metadata about each series

\end{itemize}


\par Notable properties that affect the plot:

\begin{itemize}
  \item \ttt{baseline}: The y value at which the x axis is drawn.
  \item \ttt{bar\_placement}: Selects the arrangement of bars: aligned, overlap, stacked, etc.
  \item \ttt{xlabel\_rotation}: Amount of counter-clockwise rotation of x axis labels a.k.a. group names, in degrees.
  \item \ttt{title}: Plot title (autocomputed if missing).
  \item \ttt{cycle\_seed}: Alters the sequence in which colors are assigned to series.

\end{itemize}

\subsubsection{plot\_vectors()}
\label{cha:chart-api:omnetpp.scave.utils:plot-vectors}

\begin{flushleft}
\ttt{plot\_vectors(df, props, legend\_func=make\_legend\_label)}
\end{flushleft}

\par Creates a line plot from the dataframe, with styling and additional input
coming from the properties. Each row in the dataframe defines a series.

\par Colors and markers are assigned automatically. The \ttt{cycle\_seed} property
allows you to select other combinations if the default one is not suitable.

\par A function to produce the legend labels can be passed in. By default,
\ttt{make\_legend\_label()} is used, which offers many ways to influence the
legend via dataframe columns and chart properties. In the absence of
more specified settings, the legend is normally computed from columns which best
differentiate among the vectors.

\par Parameters:

\begin{itemize}
  \item \ttt{df}: the dataframe
  \item \ttt{props} \textit{(dict)}: the properties
  \item \ttt{legend\_func}: the function to produce custom legend labels

\end{itemize}


\par Columns of the dataframe:

\begin{itemize}
  \item \ttt{vectime}, \ttt{vecvalue} \textit{(Numpy \ttt{ndarray}'s of matching sizes)}: the x and y coordinates for the plot
  \item \ttt{interpolationmode} \textit{(str, optional)}: this column normally comes from a result attribute, and determines how the points will be connected
  \item \ttt{legend} \textit{(optional)}: legend label for the series; if missing, legend labels are derived from other columns
  \item \ttt{name}, \ttt{title}, \ttt{module}, etc. \textit{(optional)}: provide input for the legend

\end{itemize}


\par Notable properties that affect the plot:

\begin{itemize}
  \item \ttt{title}: plot title (autocomputed if missing)
  \item \ttt{drawstyle}: Matplotlib draw style; if present, it overrides the draw style derived from \ttt{interpolationmode}.
  \item \ttt{linestyle}, \ttt{linecolor}, \ttt{linewidth}, \ttt{marker}, \ttt{markersize}: styling
  \item \ttt{cycle\_seed}: Alters the sequence in which colors and markers are assigned to series.

\end{itemize}

\subsubsection{plot\_vectors\_separate()}
\label{cha:chart-api:omnetpp.scave.utils:plot-vectors-separate}

\begin{flushleft}
\ttt{plot\_vectors\_separate(df, props, legend\_func=make\_legend\_label)}
\end{flushleft}

\par This is very similar to \ttt{plot\_vectors}, with identical usage.
The only difference is in the end result, where each vector will
be plotted in its own separate set of axes (coordinate system),
arranged vertically, with a shared X axis during navigation.

\subsubsection{plot\_histograms()}
\label{cha:chart-api:omnetpp.scave.utils:plot-histograms}

\begin{flushleft}
\ttt{plot\_histograms(df, props, legend\_func=make\_legend\_label)}
\end{flushleft}

\par Creates a histogram plot from the dataframe, with styling and additional input
coming from the properties. Each row in the dataframe defines a histogram.

\par Colors are assigned automatically.  The \ttt{cycle\_seed} property allows you to
select other combinations if the default one is not suitable.

\par A function to produce the legend labels can be passed in. By default,
\ttt{make\_legend\_label()} is used, which offers many ways to influence the
legend via dataframe columns and chart properties. In the absence of
more specified settings, the legend is normally computed from columns which best
differentiate among the histograms.

\par Parameters:

\begin{itemize}
  \item \ttt{df}: The dataframe.
  \item \ttt{props} \textit{(dict)}: The properties.
  \item \ttt{legend\_func} \textit{(function)}: The function to produce custom legend labels.
 See \ttt{utils.make\_legend\_label()} for prototype and semantics.

\end{itemize}


\par Columns of the dataframe:

\begin{itemize}
  \item \ttt{binedges}, \ttt{binvalues} \textit{(array-like, \ttt{len(binedges)==len(binvalues)+1})}:
 The bin edges and the bin values (count or sum of weights) for the histogram.
  \item \ttt{min}, \ttt{max}, \ttt{underflows}, \ttt{overflows} \textit{(float, optional)}: The minimum/maximum
 values, and the bin values for the underflow/overflow bins. These four columns
 must either be all present or all absent from the dataframe.
  \item \ttt{legend} \textit{(string, optional)}: Legend label for the series. If missing,
 legend labels are derived from other columns.
  \item \ttt{name}, \ttt{title}, \ttt{module}, etc. \textit{(optional)}: Provide input for the legend.

\end{itemize}


\par Notable properties that affect the plot:

\begin{itemize}
  \item \ttt{normalize} \textit{(bool)}: If true, normalize the sum of the bin values to 1. If
\ttt{normalize} is true (and \ttt{cumulative} is false), the probability density
 function (PDF) will be displayed.
  \item \ttt{cumulative} \textit{(bool)}: If true, show each bin as the sum of the previous bin
 values plus itself. If both \ttt{normalize} and \ttt{cumulative} are true, that
 results in the cumulative density function (CDF) being displayed.
  \item \ttt{show\_overflows} \textit{(bool)}: If true, show the underflow/overflow bins.
  \item \ttt{title}: Plot title (autocomputed if missing).
  \item \ttt{drawstyle}: Selects whether to fill the area below the histogram line.
  \item \ttt{linestyle}, \ttt{linecolor}, \ttt{linewidth}: Styling.
  \item \ttt{cycle\_seed}: Alters the sequence in which colors and markers are assigned to series.

\end{itemize}

\subsubsection{plot\_lines()}
\label{cha:chart-api:omnetpp.scave.utils:plot-lines}

\begin{flushleft}
\ttt{plot\_lines(df, props, legend\_func=make\_legend\_label)}
\end{flushleft}

\par Creates a line plot from the dataframe, with styling and additional input
coming from the properties. Each row in the dataframe defines a line.

\par Colors are assigned automatically.  The \ttt{cycle\_seed} property allows you to
select other combinations if the default one is not suitable.

\par A function to produce the legend labels can be passed in. By default,
\ttt{make\_legend\_label()} is used, which offers many ways to influence the
legend via dataframe columns and chart properties. In the absence of
more specified settings, the legend is normally computed from columns which best
differentiate among the lines.

\par Parameters:

\begin{itemize}
  \item \ttt{df}: The dataframe.
  \item \ttt{props} \textit{(dict)}: The properties.
  \item \ttt{legend\_func} \textit{(function)}: The function to produce custom legend labels.
 See \ttt{utils.make\_legend\_label()} for prototype and semantics.

\end{itemize}


\par Columns of the dataframe:

\begin{itemize}
  \item \ttt{x}, \ttt{y} \textit{(array-like, \ttt{len(x)==len(y)})}: The X and Y coordinates of the points.
  \item \ttt{error} \textit{(array-like, \ttt{len(x)==len(y)}, optional)}:
 The half lengths of the error bars for each point.
  \item \ttt{legend} \textit{(string, optional)}: Legend label for the series. If missing,
 legend labels are derived from other columns.
  \item \ttt{name}, \ttt{title}, \ttt{module}, etc. \textit{(optional)}: Provide input for the legend.

\end{itemize}


\par Notable properties that affect the plot:

\begin{itemize}
  \item \ttt{title}: Plot title (autocomputed if missing).
  \item \ttt{linewidth}: Line width.
  \item \ttt{marker}: Marker style.
  \item \ttt{linestyle}, \ttt{linecolor}, \ttt{linewidth}: Styling.
  \item \ttt{error\_style}: If \ttt{error} is present, controls how the error is shown.
 Accepted values: "Error bars", "Error band"
  \item \ttt{cycle\_seed}: Alters the sequence in which colors and markers are assigned to series.

\end{itemize}

\subsubsection{plot\_boxwhiskers()}
\label{cha:chart-api:omnetpp.scave.utils:plot-boxwhiskers}

\begin{flushleft}
\ttt{plot\_boxwhiskers(df, props, legend\_func=make\_legend\_label)}
\end{flushleft}

\par Creates a box and whiskers plot from the dataframe, with styling and additional
input coming from the properties. Each row in the dataframe defines one set
of a box and two whiskers.

\par Colors are assigned automatically.  The \ttt{cycle\_seed} property allows you to
select other combinations if the default one is not suitable.

\par A function to produce the legend labels can be passed in. By default,
\ttt{make\_legend\_label()} is used, which offers many ways to influence the
legend via dataframe columns and chart properties. In the absence of
more specified settings, the legend is normally computed from columns which best
differentiate among the boxes.

\par Parameters:

\begin{itemize}
  \item \ttt{df}: The dataframe.
  \item \ttt{props} \textit{(dict)}: The properties.
  \item \ttt{legend\_func} \textit{(function)}: The function to produce custom legend labels.
 See \ttt{utils.make\_legend\_label()} for prototype and semantics.

\end{itemize}


\par Columns of the dataframe:

\begin{itemize}
  \item \ttt{min}, \ttt{max}, \ttt{mean}, \ttt{stddev}, \ttt{count} \textit{(float)}: The minimum/maximum
 values, mean, standard deviation, and sample count of the data.
  \item \ttt{legend} \textit{(string, optional)}: Legend label for the series. If missing,
 legend labels are derived from other columns.
  \item \ttt{name}, \ttt{title}, \ttt{module}, etc. \textit{(optional)}: Provide input for the legend.

\end{itemize}


\par Notable properties that affect the plot:

\begin{itemize}
  \item \ttt{title}: Plot title (autocomputed if missing).
  \item \ttt{cycle\_seed}: Alters the sequence in which colors and markers are assigned to series.

\end{itemize}

\subsubsection{customized\_box\_plot()}
\label{cha:chart-api:omnetpp.scave.utils:customized-box-plot}

\begin{flushleft}
\ttt{customized\_box\_plot(percentiles, labels=None, axes=None, redraw=True, *args, **kwargs)}
\end{flushleft}

\par Generates a customized box-and-whiskers plot based on explicitly specified
percentile values. This method is necessary because \ttt{pyplot.boxplot()} insists
on computing the stats from the raw data (which we often don't have) itself.

\par The data are in the \ttt{percentiles} argument, which should be list of tuples.
One box will be drawn for each tuple. Each tuple contains 6 elements (or 5,
because the last one is optional):

\par (\textit{q1\_start}, \textit{q2\_start}, \textit{q3\_start}, \textit{q4\_start}, \textit{q4\_end}, \textit{fliers})

\par The first five elements have following meaning:

\begin{itemize}
  \item \textit{q1\_start}: y coord of bottom whisker cap
  \item \textit{q2\_start}: y coord of bottom of the box
  \item \textit{q3\_start}: y coord of median mark
  \item \textit{q4\_start}: y coord of top of the box
  \item \textit{q4\_end}: y coord of top whisker cap

\end{itemize}


\par The last element, \textit{fliers}, is a list, containing the values of the
outlier points.

\par x coords of the box-and-whiskers plots are automatic.

\par Parameters:

\begin{itemize}
  \item \ttt{percentiles}: The list of tuples.
  \item \ttt{labels}: If provided, the legend labels for the boxes.
  \item \ttt{axes}: The axes object of the plot.
  \item \ttt{redraw}: If False, redraw is deferred.
  \item \ttt{args}, \ttt{kwargs}: Passed to \ttt{axes.boxplot()}.

\end{itemize}

\subsubsection{preconfigure\_plot()}
\label{cha:chart-api:omnetpp.scave.utils:preconfigure-plot}

\begin{flushleft}
\ttt{preconfigure\_plot(props)}
\end{flushleft}

\par Configures the plot according to the given properties, which normally
get their values from setting in the "Configure Chart" dialog.
Calling this function before plotting was performed should be a standard
part of chart scripts.

\par A partial list of properties taken into account for native plots:

\begin{itemize}
  \item property keys understood by the plot widget, see \ttt{ideplot.get\_supported\_property\_keys()}

\end{itemize}


\par And for Matplotlib plots:

\begin{itemize}
  \item \ttt{plt.style}
  \item properties listed in the \ttt{matplotlibrc} property
  \item properties prefixed with \ttt{matplotlibrc.}

\end{itemize}


\par Parameters:

\begin{itemize}
  \item \ttt{props} \textit{(dict)}: the properties

\end{itemize}

\subsubsection{postconfigure\_plot()}
\label{cha:chart-api:omnetpp.scave.utils:postconfigure-plot}

\begin{flushleft}
\ttt{postconfigure\_plot(props)}
\end{flushleft}

\par Configures the plot according to the given properties, which normally
get their values from setting in the "Configure Chart" dialog.
Calling this function after plotting was performed should be a standard part
of chart scripts.

\par A partial list of properties taken into account:

\begin{itemize}
  \item \ttt{yaxis\_title}, \ttt{yaxis\_title}, \ttt{xaxis\_min},  \ttt{xaxis\_max}, \ttt{yaxis\_min},
\ttt{yaxis\_max}, \ttt{xaxis\_log}, \ttt{yaxis\_log}, \ttt{legend\_show}, \ttt{legend\_border},
\ttt{legend\_placement}, \ttt{grid\_show}, \ttt{grid\_density}
  \item properties listed in the \ttt{plot.properties} property

\end{itemize}


\par Parameters:

\begin{itemize}
  \item \ttt{props} \textit{(dict)}: the properties

\end{itemize}

\subsubsection{export\_image\_if\_needed()}
\label{cha:chart-api:omnetpp.scave.utils:export-image-if-needed}

\begin{flushleft}
\ttt{export\_image\_if\_needed(props)}
\end{flushleft}

\par If a certain property is set, save the plot in the selected image format.
Calling this function should be a standard part of chart scripts, as it is what
makes the "Export image" functionality of the IDE and \ttt{opp\_charttool} work.

\par Note that for export, even IDE-native charts are rendered using Matplotlib.

\par The properties that are taken into account:

\begin{itemize}
  \item \ttt{export\_image} \textit{(boolean)}: Controls whether to perform the exporting. This is
 normally \ttt{false}, and only set to \ttt{true} by the IDE or opp\_charttool when
 image export is requested.
  \item \ttt{image\_export\_format}: The default is SVG. Accepted formats (and their names) are the ones supported by Matplotlib.
  \item \ttt{image\_export\_folder}: The folder in which the image file is to be created.
  \item \ttt{image\_export\_filename}: The output file name. If it has no extension,
 one will be added based on the format. If missing or empty, a sanitized
 version of the chart name is used.
  \item \ttt{image\_export\_width}: Image width in inches (default: 6")
  \item \ttt{image\_export\_height}: Image height in inches (default: 4")
  \item \ttt{image\_export\_dpi}: DPI setting, default 96. For raster image formats, the
 image dimensions are produced as width (or height) times dpi.

\end{itemize}


\par Note that these properties come from two sources to allow meaningful batch
export. \ttt{export\_image}, \ttt{image\_export\_format}, \ttt{image\_export\_folder} and
\ttt{image\_export\_dpi} come from the export dialog because they are common
to all charts, while \ttt{image\_export\_filename}, \ttt{image\_export\_width} and
\ttt{image\_export\_height} come from the chart properties because they are
specific to each chart. Note that \ttt{image\_export\_dpi} is used for controlling
the resolution (for raster image formats) while letting charts maintain
their own aspect ratio and relative sizes.

\par Parameters:

\begin{itemize}
  \item \ttt{props} \textit{(dict)}: the properties

\end{itemize}

\subsubsection{get\_image\_export\_filepath()}
\label{cha:chart-api:omnetpp.scave.utils:get-image-export-filepath}

\begin{flushleft}
\ttt{get\_image\_export\_filepath(props)}
\end{flushleft}

\par Returns the file path for the image to export based on the
\ttt{image\_export\_format}, \ttt{image\_export\_folder} and
\ttt{image\_export\_filename} properties given in \ttt{props}.

\subsubsection{export\_data\_if\_needed()}
\label{cha:chart-api:omnetpp.scave.utils:export-data-if-needed}

\begin{flushleft}
\ttt{export\_data\_if\_needed(df, props, **kwargs)}
\end{flushleft}

\par If a certain property is set, save the dataframe in CSV format.
Calling this function should be a standard part of chart scripts, as it is what
makes the "Export data" functionality of the IDE and \ttt{opp\_charttool} work.

\par The properties that are taken into account:

\begin{itemize}
  \item \ttt{export\_data} \textit{(boolean)}: Controls whether to perform the exporting. This is
 normally \ttt{false}, and only set to \ttt{true} by the IDE or opp\_charttool when
 data export is requested.
  \item \ttt{data\_export\_folder}: The folder in which the CSV file is to be created.
  \item \ttt{data\_export\_filename}: The output file name. If missing or empty, a
 sanitized version of the chart name is used.

\end{itemize}


\par Note that these properties come from two sources to allow meaningful batch
export. \ttt{export\_data} and \ttt{image\_export\_folder} come from the export dialog
because they are common to all charts, and \ttt{image\_export\_filename} comes
from the chart properties because it is specific to each chart.

\par Parameters:

\begin{itemize}
  \item \ttt{df}: the dataframe to save
  \item \ttt{props} \textit{(dict)}: the properties

\end{itemize}

\subsubsection{get\_data\_export\_filepath()}
\label{cha:chart-api:omnetpp.scave.utils:get-data-export-filepath}

\begin{flushleft}
\ttt{get\_data\_export\_filepath(props)}
\end{flushleft}

\par Returns the file path for the data to export based on the
\ttt{data\_export\_format}, \ttt{data\_export\_folder} and
\ttt{data\_export\_filename} properties given in \ttt{props}.

\subsubsection{histogram\_bin\_edges()}
\label{cha:chart-api:omnetpp.scave.utils:histogram-bin-edges}

\begin{flushleft}
\ttt{histogram\_bin\_edges(values, bins=None, range=None, weights=None)}
\end{flushleft}

\par An improved version of numpy.histogram\_bin\_edges.
This will only return integer edges for input arrays consisting entirely of integers
(unless the \ttt{bins} are explicitly given otherwise).
In addition, the rightmost edge will always be strictly greater than the maximum of \ttt{values}
(unless explicitly given otherwise in \ttt{range}).

\subsubsection{confidence\_interval()}
\label{cha:chart-api:omnetpp.scave.utils:confidence-interval}

\begin{flushleft}
\ttt{confidence\_interval(alpha, data)}
\end{flushleft}

\par Returns the half-length of the confidence interval of the mean of \ttt{data}, assuming
normal distribution, for the given confidence level \ttt{alpha}.

\par Parameters:

\begin{itemize}
  \item \ttt{alpha} \textit{(float)}: Confidence level, must be in the [0..1] range.
  \item \ttt{data} \textit{(array-like)}: An array containing the values.

\end{itemize}

\subsubsection{pivot\_for\_barchart()}
\label{cha:chart-api:omnetpp.scave.utils:pivot-for-barchart}

\begin{flushleft}
\ttt{pivot\_for\_barchart(df, groups, series, confidence\_level=None)}
\end{flushleft}

\par Turns a DataFrame containing scalar results (in the format returned
by \ttt{results.get\_scalars()}) into a 3-tuple of a value, an error, and
a metadata DataFrame, which can then be passed to \ttt{utils.plot\_bars()}.
The error dataframe is None if no confidence level is given.

\par Parameters:

\begin{itemize}
  \item \ttt{df} \textit{(pandas.DataFrame)}: The dataframe to pivot.
  \item \ttt{groups} \textit{(list)}: A list of column names, the values in which will
 be used as names for the bar groups.
  \item \ttt{series} \textit{(list)}: A list of column names, the values in which will
 be used as names for the bar series.
  \item \ttt{confidence\_level} \textit{(float, optional)}:
 The confidence level to use when computing the sizes of the error bars.

\end{itemize}


\par Returns:

\begin{itemize}
  \item A triplet of DataFrames containing the pivoted data: (values, errors, metadata)

\end{itemize}

\subsubsection{pivot\_for\_scatterchart()}
\label{cha:chart-api:omnetpp.scave.utils:pivot-for-scatterchart}

\begin{flushleft}
\ttt{pivot\_for\_scatterchart(df, xaxis\_itervar, group\_by, confidence\_level=None)}
\end{flushleft}

\par Turns a DataFrame containing scalar results (in the format returned
by \ttt{results.get\_scalars()}) into a DataFrame which can then be
passed to \ttt{utils.plot\_lines()}.

\par Parameters:

\begin{itemize}
  \item \ttt{df} \textit{(pandas.DataFrame)}: The dataframe to pivot.
  \item \ttt{xaxis\_itervar} \textit{(string)}: The name of the iteration variable whose
 values are to be used as X coordinates.
  \item \ttt{group\_by} \textit{(list)}: A list of column names, the values in which will
 be used to group the scalars into lines.
  \item \ttt{confidence\_level} \textit{(float, optional)}:
 The confidence level to use when computing the sizes of the error bars.

\end{itemize}


\par Returns:

\begin{itemize}
  \item A DataFrame containing the pivoted data, with these columns:
\ttt{name}, \ttt{x}, \ttt{y}, and optionally \ttt{error} - if \ttt{confidence\_level} is given.

\end{itemize}

\subsubsection{get\_confidence\_level()}
\label{cha:chart-api:omnetpp.scave.utils:get-confidence-level}

\begin{flushleft}
\ttt{get\_confidence\_level(props)}
\end{flushleft}

\par Returns the confidence level from the \ttt{confidence\_level} property,
converted to a \ttt{float}. Also accepts "none" (returns \ttt{None} in this case),
and percentage values (e.g. "95\%").

\subsubsection{perform\_vector\_ops()}
\label{cha:chart-api:omnetpp.scave.utils:perform-vector-ops}

\begin{flushleft}
\ttt{perform\_vector\_ops(df, operations: str)}
\end{flushleft}

\par Performs the given vector operations on the dataframe, and returns the
resulting dataframe. Vector operations primarily affect the \ttt{vectime}
and \ttt{vecvalue} columns of the dataframe, which are expected to contain
\ttt{ndarray}'s of matching lengths.

\par \ttt{operations} is a multiline string where each line denotes an operation;
they are applied in sequence. The syntax of one operation is:

\par [(\ttt{compute}|\ttt{apply}) \ttt{:} ] \textit{opname} [ \ttt{(} \textit{arglist} \ttt{)} ] [ \ttt{\#} \textit{comment} ]

\par Blank lines and lines only containing a comment are also accepted.

\par \textit{opname} is the name of the function, optionally qualified with its package name.
If the package name is omitted, \ttt{omnetpp.scave.vectorops} is assumed.

\par \ttt{compute} and \ttt{apply} specify whether the newly computed vectors will replace
the input row in the DataFrame (\textit{apply}) or added as extra lines (\textit{compute}).
The default is \textit{apply}.

\par See the contents of the \ttt{omnetpp.scave.vectorops} package for more information.

\subsubsection{set\_plot\_title()}
\label{cha:chart-api:omnetpp.scave.utils:set-plot-title}

\begin{flushleft}
\ttt{set\_plot\_title(title, suggested\_chart\_name=None)}
\end{flushleft}

\par Sets the plot title. It also sets the suggested chart name (the name that
the IDE offers when adding a temporary chart to the Analysis file.)

\subsubsection{fill\_missing\_titles()}
\label{cha:chart-api:omnetpp.scave.utils:fill-missing-titles}

\begin{flushleft}
\ttt{fill\_missing\_titles(df)}
\end{flushleft}

\par Utility function to fill missing values in the \ttt{title} and \ttt{moduledisplaypath}
columns from the \ttt{name} and \ttt{module} columns. (Note that \ttt{title} and \ttt{moduledisplaypath}
normally come from result attributes of the same name.)

\subsubsection{extract\_label\_columns()}
\label{cha:chart-api:omnetpp.scave.utils:extract-label-columns}

\begin{flushleft}
\ttt{extract\_label\_columns(df, props)}
\end{flushleft}

\par Utility function to make a reasonable guess as to which column of
the given DataFrame is most suitable to act as a chart title and
which ones can be used as legend labels.

\par Ideally a "title column" should be one in which all lines have the same
value, and can be reasonably used as a title. This is often the \ttt{title}
or \ttt{name} column.

\par Label columns should be a minimal set of columns whose corresponding
value tuples uniquely identify every line in the DataFrame. These will
primarily be iteration variables and run attributes.

\par Returns:

\par A pair of a string and a list; the first value is the name of the
"title" column, and the second one is a list of pairs, each
containing the index and the name of a "label" column.

\par Example: \ttt{('title', [(8, 'numHosts'), (7, 'iaMean')])}

\subsubsection{make\_chart\_title()}
\label{cha:chart-api:omnetpp.scave.utils:make-chart-title}

\begin{flushleft}
\ttt{make\_chart\_title(df, title\_cols)}
\end{flushleft}

\par Produces a reasonably good chart title text from a result DataFrame,
given a selected list of "title" columns.

\subsubsection{pick\_two\_columns()}
\label{cha:chart-api:omnetpp.scave.utils:pick-two-columns}

\begin{flushleft}
\ttt{pick\_two\_columns(df, props=None)}
\end{flushleft}

\par Choose two columns from the dataframe which best partitions the rows
of the dataframe, and returns their names as a pair. Returns (\ttt{None}, \ttt{None})
if no such pair was found. This method is useful for creating e.g. a bar plot.

\subsubsection{select\_groups\_series()}
\label{cha:chart-api:omnetpp.scave.utils:select-groups-series}

\begin{flushleft}
\ttt{select\_groups\_series(df, props)}
\end{flushleft}

\par Extracts the column names to be used for groups and series from the \ttt{df} DataFrame,
for pivoting. The columns whose names are to be used as group names are given in
the "groups" property in \ttt{props}, as a comma-separated list.
The names for the series are selected similarly, based on the "series" property.
There should be no overlap between these two lists.

\par If both "groups" and "series" are given (non-empty), they are simply returned
as lists after some sanity checks.
If both of them are empty, a reasonable guess is made for which columns should
be used, and (["module"], ["name"]) is used as a fallback.

\par The data in \ttt{df} should be in the format as returned by \ttt{result.get\_scalars()},
and the result can be used directly by \ttt{utils.pivot\_for\_barchart()}.

\par Returns:
    - (group\_names, series\_names): A pair of lists of strings containing the
      selected names for the groups and the series, respectively.

\subsubsection{select\_xaxis\_and\_groupby()}
\label{cha:chart-api:omnetpp.scave.utils:select-xaxis-and-groupby}

\begin{flushleft}
\ttt{select\_xaxis\_and\_groupby(df, props)}
\end{flushleft}

\par Extracts an iteration variable name and the column names to be used for grouping from
the \ttt{df} DataFrame, for pivoting. The columns whose names are to be used as group
names are given in the "group\_by" property in \ttt{props}, as a comma-separated list.
The name of the iteration variable is selected similarly, from the "xaxis\_itervar" property.
The "group\_by" list should not contain the given "xaxis\_itervar" name.

\par If both "xaxis\_itervar" and "group\_by" are given (non-empty), they are simply returned
after some sanity checks, with "group\_by" split into a list.
If both of them are empty, a reasonable guess is made for which columns should be used.

\par The data in \ttt{df} should be in the format as returned by \ttt{result.get\_scalars()},
and the result can be used directly by \ttt{utils.pivot\_for\_scatterchart()}.

\par Returns:
    - (xaxis\_itervar, group\_by): An iteration variable name, and a list of strings
      containing the selected column names to be used as groups.

\subsubsection{assert\_columns\_exist()}
\label{cha:chart-api:omnetpp.scave.utils:assert-columns-exist}

\begin{flushleft}
\ttt{assert\_columns\_exist(df, cols, message="Expected column missing from DataFrame")}
\end{flushleft}

\par Ensures that the dataframe contains the given columns. If any of them are missing,
the function raises an error with the given message.

\par Parameters:

\begin{itemize}
  \item \ttt{cols} \textit{(list of strings)}: Column names to check.

\end{itemize}

\subsubsection{to\_numeric()}
\label{cha:chart-api:omnetpp.scave.utils:to-numeric}

\begin{flushleft}
\ttt{to\_numeric(df, columns=None, errors="ignore", downcast=None)}
\end{flushleft}

\par Convenience function. Runs \ttt{pandas.to\_numeric} on the given
(or all) columns of \ttt{df}. If any of the given columns doesn't
exist, throws an error.

\par Parameters:

\begin{itemize}
  \item \ttt{df} \textit{(DataFrame)}: The DataFrame to operate on
  \item \ttt{columns} \textit{(list of strings)}: The list of column names to convert.
If not given, all columns will be converted.
  \item \ttt{errors}, \ttt{downcast} \textit{(string)}: Will be passed to \ttt{pandas.to\_numeric()}

\end{itemize}

\subsubsection{parse\_rcparams()}
\label{cha:chart-api:omnetpp.scave.utils:parse-rcparams}

\begin{flushleft}
\ttt{parse\_rcparams(rc\_content)}
\end{flushleft}

\par Accepts a multiline string that contains rc file content in Matplotlib's
RcParams syntax, and returns its contents as a dictionary. Parse errors
and duplicate keys are reported via exceptions.

\subsubsection{make\_fancy\_xticklabels()}
\label{cha:chart-api:omnetpp.scave.utils:make-fancy-xticklabels}

\begin{flushleft}
\ttt{make\_fancy\_xticklabels(ax)}
\end{flushleft}

\par Only useful for Matplotlib plots. It causes the x tick labels to be rotated
by the minimum amount necessary so that they don't overlap. Note that the
necessary amount of rotation typically depends on the horizontal zoom level.

\subsubsection{split()}
\label{cha:chart-api:omnetpp.scave.utils:split}

\begin{flushleft}
\ttt{split(s, sep=",")}
\end{flushleft}

\par Split a string with the given separator (by default with comma), trim
the surrounding whitespace from the items, and return the result as a
list. Returns an empty list for an empty or all-whitespace input string.
(Note that in contrast, \ttt{s.split(',')} will return an empty array,
even for \ttt{s=''}.)

\subsection{Module omnetpp.scave.vectorops}
\label{cha:chart-api:omnetpp.scave.vectorops}

\par Contains operations that can be applied to vectors.

\par In the IDE, operations can be applied to vectors on a vector chart by means
of the plot's context menu and by editing the \textit{Vector Operations} field in
the chart configuration dialog.

\par Every vector operation is implemented as a function. The notation used
in the documentation of the individual functions is:

\begin{itemize}
  \item \textit{y[k]}: The kth value in the input
  \item \textit{t[k]}: The kth timestamp in the input
  \item \textit{yout[k]}: The kth value in the output
  \item \textit{tout[k]}: The kth timestamp in the output

\end{itemize}


\par A vector operation function accepts a DataFrame row as the first positional
argument, and optionally additional arguments specific to its operation.
When the function is invoked, the row will contain a \ttt{vectime} and a \ttt{vecvalue}
column (both containing NumPy \ttt{ndarray}'s) that are the input of the operation.
The function should return a similar row, with updated \ttt{vectime} and a \ttt{vecvalue}
columns.

\par Additionally, the operation may update the \ttt{name} and \ttt{title} columns (provided
they exist) to reflect the processing in the name. For example, an operation
that computes \textit{mean} may return \ttt{mean(\%s)} as name and \ttt{Mean of \%s} as title
(where \ttt{\%s} indicates the original name/title).

\par The \ttt{aggregate()} and \ttt{merge()} functions are special. They receive a DataFrame
instead of a row in the first argument, and return new DataFrame with the result.

\par Vector operations can be applied to a DataFrame using \ttt{utils.perform\_vector\_ops(df,ops)}.
\ttt{ops} is a multiline string where each line denotes an operation; they are
applied in sequence. The syntax of one operation is:

\par [(\ttt{compute}|\ttt{apply}) \ttt{:} ] \textit{opname} [ \ttt{(} \textit{arglist} \ttt{)} ] [ \ttt{\#} \textit{comment} ]

\par \textit{opname} is the name of the function, optionally qualified with its package name.
If the package name is omitted, \ttt{omnetpp.scave.vectorops} is assumed.

\par \ttt{compute} and \ttt{apply} specify whether the newly computed vectors will replace
the input row in the DataFrame (\textit{apply}) or added as extra lines (\textit{compute}).
The default is \textit{apply}.

\par To register a new vector operation, define a function that fulfills the above interface
(e.g. in the chart script, or an external \ttt{.py} file, that the chart script imports),
with the \ttt{omnetpp.scave.vectorops.vector\_operation} decorator on it.

\par Make sure that the registered function does not modify the data of the NumPy array
instances in the rows, because it would have an unwanted effect when used in \ttt{compute}
(as opposed to \ttt{apply}) mode.

\par Example:
\begin{filelisting}
from omnetpp.scave import vectorops

@vectorops.vector\_operation("Fooize", "foo(42)")
def foo(r, arg1, arg2=5):
    \# r.vectime = r.vectime * 2    \# <- this is okay
    \# r.vectime *= 2               \# <- this is NOT okay!

    r.vectime = r.vectime * arg1 + arg2
    if "title" in r:
        r.title = r.title + ", but fooized" \# this is also okay
    return r

\end{filelisting}

\subsubsection{perform\_vector\_ops()}
\label{cha:chart-api:omnetpp.scave.vectorops:perform-vector-ops}

\begin{flushleft}
\ttt{perform\_vector\_ops(df, operations: str)}
\end{flushleft}

\par See: utils.perform\_vector\_ops

\subsubsection{vector\_operation()}
\label{cha:chart-api:omnetpp.scave.vectorops:vector-operation}

\begin{flushleft}
\ttt{vector\_operation(label: str = None, example: str = None)}
\end{flushleft}

\par Returns, or acts as, a decorator; to be used on methods you wish to register as vector operations.
Parameters:

\begin{itemize}
  \item \ttt{label}: will be shown on the GUI for the user
  \item \ttt{example}: should be string, containing a valid invocation of the function

\end{itemize}


\par Alternatively, this can also be used directly as decorator (without calling it first).

\subsubsection{lookup\_operation()}
\label{cha:chart-api:omnetpp.scave.vectorops:lookup-operation}

\begin{flushleft}
\ttt{lookup\_operation(module, name)}
\end{flushleft}

\par Returns a function from the registered vector operations by name, and optionally module.
\ttt{module} and \ttt{name} are both strings. \ttt{module} can also be \ttt{None}, in which case it is ignored.

\subsubsection{aggregate()}
\label{cha:chart-api:omnetpp.scave.vectorops:aggregate}

\begin{flushleft}
\ttt{aggregate(df, function="average")}
\end{flushleft}

\par Aggregates several vectors into a single one, aggregating the
y values \textit{at the same time coordinate} with the specified function.
Possible values: 'sum', 'average', 'count', 'maximum', 'minimum'

\subsubsection{merge()}
\label{cha:chart-api:omnetpp.scave.vectorops:merge}

\begin{flushleft}
\ttt{merge(df)}
\end{flushleft}

\par Merges several series into a single one, maintaining increasing
time order in the output.

\subsubsection{mean()}
\label{cha:chart-api:omnetpp.scave.vectorops:mean}

\begin{flushleft}
\ttt{mean(r)}
\end{flushleft}

\par Computes mean on (0,t): yout[k] = sum(y[i], i=0..k) / (k+1).

\subsubsection{sum()}
\label{cha:chart-api:omnetpp.scave.vectorops:sum}

\begin{flushleft}
\ttt{sum(r)}
\end{flushleft}

\par Sums up values: yout[k] = sum(y[i], i=0..k)

\subsubsection{add()}
\label{cha:chart-api:omnetpp.scave.vectorops:add}

\begin{flushleft}
\ttt{add(r, c)}
\end{flushleft}

\par Adds a constant to all values in the input: yout[k] = y[k] + c

\subsubsection{compare()}
\label{cha:chart-api:omnetpp.scave.vectorops:compare}

\begin{flushleft}
\ttt{compare(r, threshold, less=None, equal=None, greater=None)}
\end{flushleft}

\par Compares value against a threshold, and optionally replaces it with a constant.
yout[k] = if y[k] < threshold and less != None then less;
     else if y[k] == threshold and equal != None then equal;
     else if y[k] > threshold and greater != None then greater;
     else y[k]
The last three parameters are all independently optional.

\subsubsection{crop()}
\label{cha:chart-api:omnetpp.scave.vectorops:crop}

\begin{flushleft}
\ttt{crop(r, t1, t2)}
\end{flushleft}

\par Discards values outside the [t1, t2] interval.
The time values are in seconds.

\subsubsection{difference()}
\label{cha:chart-api:omnetpp.scave.vectorops:difference}

\begin{flushleft}
\ttt{difference(r)}
\end{flushleft}

\par Subtracts the previous value from every value: yout[k] = y[k] - y[k-1]

\subsubsection{diffquot()}
\label{cha:chart-api:omnetpp.scave.vectorops:diffquot}

\begin{flushleft}
\ttt{diffquot(r)}
\end{flushleft}

\par Calculates the difference quotient of every value and the subsequent one:
yout[k] = (y[k+1]-y[k]) / (t[k+1]-t[k])

\subsubsection{divide\_by()}
\label{cha:chart-api:omnetpp.scave.vectorops:divide-by}

\begin{flushleft}
\ttt{divide\_by(r, a)}
\end{flushleft}

\par Divides every value in the input by a constant: yout[k] = y[k] / a

\subsubsection{divtime()}
\label{cha:chart-api:omnetpp.scave.vectorops:divtime}

\begin{flushleft}
\ttt{divtime(r)}
\end{flushleft}

\par Divides every value in the input by the corresponding time: yout[k] = y[k] / t[k]

\subsubsection{expression()}
\label{cha:chart-api:omnetpp.scave.vectorops:expression}

\begin{flushleft}
\ttt{expression(r, expression, as\_time=False)}
\end{flushleft}

\par Replaces the value with the result of evaluating the Python arithmetic expression
given as a string: yout[k] = eval(expression). The expression may use
the following variables: \ttt{t}, \ttt{y}, \ttt{tprev}, \ttt{yprev}, \ttt{tnext}, \ttt{ynext}, \ttt{k}, \ttt{n}
which stand for t[k], y[k], t[k-1], y[k-1], t[k+1] and y[k+1], k, and the
size of vector, respectively.

\par If \ttt{as\_time} is \ttt{True}, the result will be assigned to the time variable
instead of the value variable.

\par Note that for efficiency, the expression will be evaluated only once,
with the variables being \ttt{np.ndarray} instances instead of scalar \ttt{float} values.
Thus, the result is computed using vector operations instead of looping
through all vector indices in Python. Expression syntax remains the usual.
Most Numpy mathematical functions can be used without module prefix; other
Numpy functions can be used by prefixing them with \ttt{np.}.

\par Examples: \ttt{2*y+0.5}, \ttt{abs(floor(y))}, \ttt{(y-yprev)/(t-tprev)}, \ttt{fmin(yprev,ynext)},
\ttt{cumsum(y)}, \ttt{nan\_to\_num(y)}

\subsubsection{integrate()}
\label{cha:chart-api:omnetpp.scave.vectorops:integrate}

\begin{flushleft}
\ttt{integrate(r, interpolation="sample-hold")}
\end{flushleft}

\par Integrates the input as a step function ("sample-hold" or "backward-sample-hold")
or with linear ("linear") interpolation.

\subsubsection{lineartrend()}
\label{cha:chart-api:omnetpp.scave.vectorops:lineartrend}

\begin{flushleft}
\ttt{lineartrend(r, a)}
\end{flushleft}

\par Adds a linear component with the given steepness to the input series: yout[k] = y[k] + a * t[k]

\subsubsection{modulo()}
\label{cha:chart-api:omnetpp.scave.vectorops:modulo}

\begin{flushleft}
\ttt{modulo(r, m)}
\end{flushleft}

\par Computes floating point reminder (modulo) of the input values with a constant: yout[k] = y[k] \% m

\subsubsection{movingavg()}
\label{cha:chart-api:omnetpp.scave.vectorops:movingavg}

\begin{flushleft}
\ttt{movingavg(r, alpha)}
\end{flushleft}

\par Applies the exponentially weighted moving average filter with
the given smoothing coefficient in range (0.0, 1.0]:
yout[k] = yout[k-1] + alpha * (y[k]-yout[k-1])

\subsubsection{multiply\_by()}
\label{cha:chart-api:omnetpp.scave.vectorops:multiply-by}

\begin{flushleft}
\ttt{multiply\_by(r, a)}
\end{flushleft}

\par Multiplies every value in the input by a constant: yout[k] = a * y[k]

\subsubsection{removerepeats()}
\label{cha:chart-api:omnetpp.scave.vectorops:removerepeats}

\begin{flushleft}
\ttt{removerepeats(r)}
\end{flushleft}

\par Removes repeated (consecutive) y values

\subsubsection{slidingwinavg()}
\label{cha:chart-api:omnetpp.scave.vectorops:slidingwinavg}

\begin{flushleft}
\ttt{slidingwinavg(r, window\_size, min\_samples=None)}
\end{flushleft}

\par Replaces every value with the mean of values in the window:
yout[k] = sum(y[i], i=(k-winsize+1)..k) / winsize
If min\_samples is also given, allows each window to have only that many
valid (not missing [at the ends], and not NaN) samples in each window.

\subsubsection{subtractfirstval()}
\label{cha:chart-api:omnetpp.scave.vectorops:subtractfirstval}

\begin{flushleft}
\ttt{subtractfirstval(r)}
\end{flushleft}

\par Subtract the first value from every subsequent value: yout[k] = y[k] - y[0]

\subsubsection{timeavg()}
\label{cha:chart-api:omnetpp.scave.vectorops:timeavg}

\begin{flushleft}
\ttt{timeavg(r, interpolation)}
\end{flushleft}

\par Average over time (integral divided by time), possible
parameter values: 'sample-hold', 'backward-sample-hold', 'linear'

\subsubsection{timediff()}
\label{cha:chart-api:omnetpp.scave.vectorops:timediff}

\begin{flushleft}
\ttt{timediff(r)}
\end{flushleft}

\par Sets each value to the elapsed time (delta) since the previous value:
tout[k] = t[k] - t[k-1]

\subsubsection{timeshift()}
\label{cha:chart-api:omnetpp.scave.vectorops:timeshift}

\begin{flushleft}
\ttt{timeshift(r, dt)}
\end{flushleft}

\par Shifts the input series in time by a constant (in seconds): tout[k] = t[k] + dt

\subsubsection{timedilation()}
\label{cha:chart-api:omnetpp.scave.vectorops:timedilation}

\begin{flushleft}
\ttt{timedilation(r, c)}
\end{flushleft}

\par Dilates the input series in time by a constant factor: tout[k] = t[k] * c

\subsubsection{timetoserial()}
\label{cha:chart-api:omnetpp.scave.vectorops:timetoserial}

\begin{flushleft}
\ttt{timetoserial(r)}
\end{flushleft}

\par Replaces time values with their index: tout[k] = k

\subsubsection{timewinavg()}
\label{cha:chart-api:omnetpp.scave.vectorops:timewinavg}

\begin{flushleft}
\ttt{timewinavg(r, window\_size=1)}
\end{flushleft}

\par Calculates time average: Replaces the input values with one every 'window\_size'
interval (in seconds), that is the mean of the original values in that interval.
tout[k] = k * winSize,
yout[k] = average of y values in the [(k-1) * winSize, k * winSize) interval

\subsubsection{timewinthruput()}
\label{cha:chart-api:omnetpp.scave.vectorops:timewinthruput}

\begin{flushleft}
\ttt{timewinthruput(r, window\_size=1)}
\end{flushleft}

\par Calculates time windowed throughput:
tout[k] = k * winSize,
yout[k] = sum of y values in the [(k-1) * winSize, k * winSize) interval divided by window\_size

\subsubsection{winavg()}
\label{cha:chart-api:omnetpp.scave.vectorops:winavg}

\begin{flushleft}
\ttt{winavg(r, window\_size=10)}
\end{flushleft}

\par Calculates batched average: replaces every 'winsize' input values
with their mean. Time is the time of the first value in the batch.

\subsection{Module omnetpp.scave.analysis}
\label{cha:chart-api:omnetpp.scave.analysis}

\par This module allows reading, writing, creating and editing
OMNeT++ Analysis (.anf) files, querying their contents, and
running the charts scripts they contain. The main user of this
module is \ttt{opp\_charttool}.

\subsubsection{Class DialogPage}
\label{cha:chart-api:omnetpp.scave.analysis:DialogPage}

\par Represents a dialog page in a \ttt{Chart}. Dialog pages have an ID, a label
(which the IDE displays on the page's tab in the \textit{Chart Properties} dialog),
and XSWT content (which describes the UI controls on the page).

\label{cha:chart-api:omnetpp.scave.analysis:DialogPage:--init--}

\begin{flushleft}
\ttt{DialogPage(self, id: str = None, label: str = "", content: str = "")}
\end{flushleft}



\subsubsection{Class Chart}
\label{cha:chart-api:omnetpp.scave.analysis:Chart}

\par Represents a chart in an \ttt{Analysis}. Charts have an ID, a name, a chart script
(a Python script that mainly uses Pandas and the \ttt{omnetpp.scave.*} modules),
dialog pages (which make up the contents of the Chart Properties dialog in the IDE),
and properties (which are what the \textit{Chart Properties} dialog in the IDE edits).

\label{cha:chart-api:omnetpp.scave.analysis:Chart:--init--}

\begin{flushleft}
\ttt{Chart(self, id: str = None, name: str = "", type: str = "MATPLOTLIB", template: str = None, icon: str = None, script: str = "", dialog\_pages=[], properties=\{\})}
\end{flushleft}



\subsubsection{Class Folder}
\label{cha:chart-api:omnetpp.scave.analysis:Folder}

\par Represents a folder in an \ttt{Analysis}. Folders may contain charts and further folders.

\label{cha:chart-api:omnetpp.scave.analysis:Folder:--init--}

\begin{flushleft}
\ttt{Folder(self, id: str = None, name: str = "", items=[])}
\end{flushleft}



\subsubsection{Class Workspace}
\label{cha:chart-api:omnetpp.scave.analysis:Workspace}

\par This is an abstraction of an IDE workspace, and makes it possible to map
workspace paths to filesystem paths. This is necessary because the inputs
in the \ttt{Analysis} are workspace paths.

\label{cha:chart-api:omnetpp.scave.analysis:Workspace:--init--}

\begin{flushleft}
\ttt{Workspace(self, workspace\_dir, project\_paths=\{\})}
\end{flushleft}

\par Accepts the workspace location, plus a dict that contains the (absolute,
or workspace-relative) location of projects by name, for projects that
are NOT under the <workspace\_dir>/<projectname> location.

\label{cha:chart-api:omnetpp.scave.analysis:Workspace:find-workspace}

\begin{flushleft}
\ttt{\small{Workspace.}find\_workspace(dir=None)}
\end{flushleft}

\par Utility function: Find the IDE workspace directory searching from the
given directory (or the current dir if not given) upwards. The workspace
directory of the Eclipse-based IDE can be recognized by having a \ttt{.metadata}
subdir.

\label{cha:chart-api:omnetpp.scave.analysis:Workspace:get-project-location}

\begin{flushleft}
\ttt{\small{Workspace.}get\_project\_location(self, project\_name)}
\end{flushleft}

\par Returns the location of the given workspace project in the filesystem path.

\label{cha:chart-api:omnetpp.scave.analysis:Workspace:to-filesystem-path}

\begin{flushleft}
\ttt{\small{Workspace.}to\_filesystem\_path(self, wspath)}
\end{flushleft}

\par Translate workspace paths to filesystem path.

\subsubsection{Class Analysis}
\label{cha:chart-api:omnetpp.scave.analysis:Analysis}

\par Represents an OMNeT++ Analysis, i.e. the contents of an  \ttt{anf} file. Methods
allow reading/writing \ttt{anf} files, and running the charts in them for interactive
display, image/data export or other side effects.

\label{cha:chart-api:omnetpp.scave.analysis:Analysis:--init--}

\begin{flushleft}
\ttt{Analysis(self, inputs=[], items=[])}
\end{flushleft}



\label{cha:chart-api:omnetpp.scave.analysis:Analysis:collect-charts}

\begin{flushleft}
\ttt{\small{Analysis.}collect\_charts(self, folder=None)}
\end{flushleft}

\par Collects and returns a list of all charts in the specified folder, or
in this Analysis if no folder is given.

\label{cha:chart-api:omnetpp.scave.analysis:Analysis:export-data}

\begin{flushleft}
\ttt{\small{Analysis.}export\_data(self, chart, wd, workspace, format="csv", target\_folder=None, filename=None, enforce=True, extra\_props=\{\})}
\end{flushleft}

\par Runs a chart script for data export. This method just calls \ttt{run\_chart()}
with extra properties that instruct the chart script to perform data export.
(It is assumed that the chart script invokes \ttt{utils.export\_data\_if\_needed()}
or implements equivalent functionality).

\label{cha:chart-api:omnetpp.scave.analysis:Analysis:export-image}

\begin{flushleft}
\ttt{\small{Analysis.}export\_image(self, chart, wd, workspace, format="svg", target\_folder=None, filename=None, width=None, height=None, dpi=96, enforce=True, extra\_props=\{\})}
\end{flushleft}

\par Runs a chart script for image export. This method just calls \ttt{run\_chart()}
with extra properties that instruct the chart script to perform image export.
(It is assumed that the chart script invokes \ttt{utils.export\_image\_if\_needed()}
or implements equivalent functionality).

\label{cha:chart-api:omnetpp.scave.analysis:Analysis:from-anf-file}

\begin{flushleft}
\ttt{\small{Analysis.}from\_anf\_file(anf\_file\_name)}
\end{flushleft}

\par Reads the given anf file and returns its content as an \ttt{Analysis} object.

\label{cha:chart-api:omnetpp.scave.analysis:Analysis:get-item-path}

\begin{flushleft}
\ttt{\small{Analysis.}get\_item\_path(self, item)}
\end{flushleft}

\par Returns the path of the item (Chart or Folder) within the Analysis
as list of path segments (Folder items). The returned list includes
both the root folder of the Analysis and the item itself. If the
item is not part of the Analysis, None is returned.

\label{cha:chart-api:omnetpp.scave.analysis:Analysis:get-item-path-as-string}

\begin{flushleft}
\ttt{\small{Analysis.}get\_item\_path\_as\_string(self, item, separator=" / ")}
\end{flushleft}

\par Returns the path of the item (Chart or Folder) within the Analysis as a
string. Segments are joined with the given separator. The returned
string includes the item name itself, but not the root folder (i.e. for
items in the root folder, the path string equals to the item name). If
the item is not part of the Analysis, None is returned.

\label{cha:chart-api:omnetpp.scave.analysis:Analysis:run-chart}

\begin{flushleft}
\ttt{\small{Analysis.}run\_chart(self, chart, wd, workspace, extra\_props=\{\}, show=False)}
\end{flushleft}

\par Runs a chart script with the given working directory, workspace, and extra
properties in addition to the chart's properties. If \ttt{show=True}, it calls
\ttt{plt.show()} if it was not already called by the script.

\label{cha:chart-api:omnetpp.scave.analysis:Analysis:to-anf-file}

\begin{flushleft}
\ttt{\small{Analysis.}to\_anf\_file(self, filename)}
\end{flushleft}

\par Saves the analysis to the given .anf file.

\subsubsection{load\_anf\_file()}
\label{cha:chart-api:omnetpp.scave.analysis:load-anf-file}

\begin{flushleft}
\ttt{load\_anf\_file(anf\_file\_name)}
\end{flushleft}
% @ENDFILE

