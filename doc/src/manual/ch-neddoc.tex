\chapter{Documenting NED and Messages}
\label{cha:neddoc}

\section{Overview}
\label{sec:neddoc:overview}

{\opp} provides a tool which can generate HTML documentation from NED files
and message definitions. Like Javadoc and Doxygen, the NED documentation tool
makes use of source code comments. The generated HTML documentation
lists all modules, channels, messages, etc., and presents their details including
description, gates, parameters, assignable submodule parameters, and
syntax-highlighted source code. The documentation also includes clickable
network diagrams (exported from the graphical editor) and usage diagrams as
well as inheritance diagrams.

The documentation tool integrates with Doxygen, meaning that it can
hyperlink simple modules and message classes to their C++ implementation
classes in the Doxygen documentation. If the C++ documentation is generated
with some Doxygen features turned on (such as \textit{inline-sources} and
\textit{referenced-by-relation}, combined with \textit{extract-all},
\textit{extract-private} and \textit{extract-static}), the result is an
easily browsable and very informative presentation of the source code.

NED documentation generation is available as part of the {\opp} IDE, and
also as a command-line tool (\fprog{opp\_neddoc}).


\section{Documentation Comments}
\label{sec:neddoc:documentation-comments}

Documentation is embedded in normal comments. All \texttt{//} comments
that are in the ``right place'' (from the documentation tool's
point of view) will be included in the generated documentation.
  \footnote{In contrast, Javadoc and Doxygen use special comments (those
     beginning with \texttt{/**}, \texttt{///}, \texttt{//<} or a similar
     marker) to distinguish documentation from ``normal'' comments in the
     source code. In {\opp} there is no need for that: NED and the message
     syntax is so compact that practically all comments one would want to write
     in them can serve documentation purposes.}

Example:

\begin{ned}
//
// An ad-hoc traffic generator to test the Ethernet models.
//
simple Gen
{
    parameters:
        string destAddress;  // destination MAC address
        int protocolId;      // value for SSAP/DSAP in Ethernet frame
        double waitMean @unit(s); // mean for exponential interarrival times
    gates:
        output out;  // to Ethernet LLC
}
\end{ned}

One can also place comments above parameters and gates, which is better
suited for long explanations. Example:

\begin{ned}
//
// Deletes packets and optionally keeps statistics.
//
simple Sink
{
    parameters:
        // Turns statistics generation on/off. This is a very long
        // comment because it has to be described what statistics
        // are collected.
        bool collectStatistics = default(true);
    gates:
        input in;
}
\end{ned}

\subsection{Private Comments}
\label{sec:neddoc:private-comments}

Lines that start with \texttt{//\#} will not appear in the generated
documentation. Such lines can be used to make ``private'' comments like
\texttt{FIXME} or \texttt{TODO}, or to comment out unused code.

\begin{ned}
//
// An ad-hoc traffic generator to test the Ethernet models.
//# TODO above description needs to be refined
//
simple Gen
{
    parameters:
        string destAddress;  // destination MAC address
        int protocolId;      // value for SSAP/DSAP in Ethernet frame
        //# double burstiness;  -- not yet supported
        double waitMean @unit(s); // mean for exponential interarrival times
    gates:
        output out;  // to Ethernet LLC
}
\end{ned}


\subsection{More on Comment Placement}
\label{sec:neddoc:comment-placement}

Comments should be written where the tool will find them.
This is a) immediately above the documented item, or b) after the
documented item, on the same line.

In the former case, make sure there is no blank line left
between the comment and the documented item. Blank lines
detach the comment from the documented item.

Example:
\begin{ned}
// This is wrong! Because of the blank line, this comment is not
// associated with the following simple module!

simple Gen
{
    ...
}
\end{ned}

Do not try to comment groups of parameters together. The result
will be awkward.

\section{Referring to Other NED and Message Types}
\label{sec:neddoc:referring-to-other-ned-and-message-types}

One can reference other NED and message types by name in comments. There
are two styles in which references can be written: automatic linking and
tilde linking. The same style must be following throughout the whole
project, and the correct one must be selected in the documentation
generator tool when it is run.

\subsection{Automatic Linking}
\label{sec:neddoc:automatic-linking}

In the automatic linking style, words that match existing NED of message
types are hyperlinked automatically. It is usually enough to write the
simple name of the type (e.g. \ttt{TCP}), one does not need to spell out the
fully qualified type (\ttt{inet.transport.tcp.TCP}), although that is also
allowed.

Automatic hyperlinking is sometimes overly agressive. For example, when the
words \textit{IP address} appear in a comment and the project contains an
\ttt{IP} module, it will create a hyperlink to the module, which is not
desirable. One can prevent hyperlinking of a word by inserting a
backslash in front it: \ttt{{\textbackslash}IP address}. The backslash will
not appear in the HTML output. The \ttt{<nohtml>} tag will also prevent
hyperlinking words in the enclosed text: \ttt{<nohtml>IP address</nohtml>}.
On the other hand, if a backslash needs to be printed immediately
in front of a word (e.g. output \textit{``use {\textbackslash}t to print a Tab''}),
use either two backslashes (\ttt{use {\textbackslash}{\textbackslash}t...}) or the
\ttt{<nohtml>} tag (\ttt{<nohtml>use {\textbackslash}t...</nohtml>}).
Backslashes in other contexts (i.e. when not in front of a word) do not have
a special meaning, and are preserved in the output.

The detailed rules:

\begin{enumerate}
  \item Words matching a type name are automatically hyperlinked
  \item A backslash immediately followed by an identifier (i.e. letter or underscore)
        prevents hyperlinking, and the backslash is removed from the output
  \item A double backslash followed by an identifier produces a single backslash,
        plus the potentially hyperlinked identifier
  \item Backslashes in any other contexts are not interpreted, and preserved in the output
  \item Tildes are not interpreted, and are preserved in the output
  \item Inside \ttt{<nohtml>}, no backslash processing or hyperlinking takes place
\end{enumerate}

\subsection{Tilde Linking}
\label{sec:neddoc:tilde-linking}

In the tilde style, only words that are explicitly marked with a tilde are
subject to hyperlinking: \ttt{{\textasciitilde}TCP},
\ttt{{\textasciitilde}inet.transport.tcp.TCP}.

To produce a literal tilde followed by an identifier in the output (for example,
to output \textit{``the {\textasciitilde}TCP() destructor''}), the tilde character
needs to be doubled: \ttt{the {\textasciitilde}{\textasciitilde}TCP() destructor}.

The detailed rules:

\begin{enumerate}
  \item Words matching a type name are \textit{not} hyperlinked automatically
  \item A tilde immediately followed by an identifier (i.e. letter or underscore)
        will be hyperlinked, and the tilde is removed from the output. It is
        considered an error if there is no type with that name.
  \item A double tilde followed by an identifier produces a single tilde plus the identifier
  \item Tildes in any other contexts are not interpreted, and preserved in the output
  \item Backslashes are not interpreted, and are preserved in the output
  \item Inside \ttt{<nohtml>}, no tilde processing or hyperlinking takes place
\end{enumerate}

\section{Text Layout and Formatting}
\label{sec:neddoc:text-layout-and-formatting}

\subsection{Paragraphs and Lists}
\label{sec:neddoc:paragraphs-and-lists}

When writing documentation comments longer than a few sentences, one often
needs structuring and formatting facilities. NED provides paragraphs,
bulleted and numbered lists, and basic formatting support. More
sophisticated formatting can be achieved using HTML.

Paragraphs can be created by separating text by blanks lines. Lines
beginning with ``\ttt{-}'' will be turned into bulleted lists, and lines
beginning with ``\ttt{-\#}'' into numbered lists. An example:

\begin{ned}
//
// Ethernet MAC layer. MAC performs transmission and reception of frames.
//
// Processing of frames received from higher layers:
// - sends out frame to the network
// - no encapsulation of frames -- this is done by higher layers.
// - can send PAUSE message if requested by higher layers (PAUSE protocol,
//   used in switches). PAUSE is not implemented yet.
//
// Supported frame types:
// -# IEEE 802.3
// -# Ethernet-II
//
\end{ned}


\subsection{Special Tags}
\label{sec:neddoc:special-tags}

The documentation tool understands the following tags and will render them accordingly:
\ttt{@author}, \ttt{@date}, \ttt{@todo}, \ttt{@bug}, \ttt{@see}, \ttt{@since},
\ttt{@warning}, \ttt{@version}. Example usage:

\begin{ned}
//
// @author Jack Foo
// @date 2005-02-11
//
\end{ned}


\subsection{Text Formatting Using HTML}
\label{sec:neddoc:text-formatting-using-html}

Common HTML tags are understood as formatting commands.
The most useful tags are: \ttt{<i>..</i>} (italic),
\ttt{<b>..</b>} (bold), \ttt{<tt>..</tt>} (typewriter font),
\ttt{<sub>..</sub>} (subscript), \ttt{<sup>..</sup>} (superscript),
\ttt{<br>} (line break), \ttt{<h3>} (heading),
\ttt{<pre>..</pre>} (preformatted text) and \ttt{<a href=..>..</a>} (link),
as well as a few other tags used for table creation (see below).
For example, \texttt{<i>Hello</i>} will be rendered as ``\textit{Hello}''
(using an italic font).

The complete list of HTML tags interpreted by the documentation tool are:
\texttt{<a>}, \texttt{<b>}, \texttt{<body>}, \texttt{<br>}, \texttt{<center>},
\texttt{<caption>}, \texttt{<code>}, \texttt{<dd>}, \texttt{<dfn>}, \texttt{<dl>},
\texttt{<dt>}, \texttt{<em>}, \texttt{<form>}, \texttt{<font>}, \texttt{<hr>},
\texttt{<h1>}, \texttt{<h2>}, \texttt{<h3>}, \texttt{<i>}, \texttt{<input>}, \texttt{<img>},
\texttt{<li>}, \texttt{<meta>}, \texttt{<multicol>}, \texttt{<ol>}, \texttt{<p>}, \texttt{<small>},
\texttt{<span>}, \texttt{<strong>},
\texttt{<sub>}, \texttt{<sup>}, \texttt{<table>}, \texttt{<td>}, \texttt{<th>}, \texttt{<tr>},
\texttt{<tt>}, \texttt{<kbd>}, \texttt{<ul>}, \texttt{<var>}.

Any tags not in the above list will not be interpreted as formatting commands
but will be printed verbatim -- for example, \texttt{<what>bar</what>}
will be rendered literally as ``<what>bar</what>'' (unlike HTML where
unknown tags are simply ignored, i.e. HTML would display ``bar'').

With links to external pages or web sites, its useful to add the
\ttt{target="\_blank"} attribute to ensure pages come up in a new browser
tab, and not in the current frame. Alternatively, one can use the
\ttt{target="\_top"} attribute which replaces all frames in the current
browser.

Examples:

\begin{ned}
//
// For more info on Ethernet and other LAN standards, see the
// <a href="http://www.ieee802.org/" target="_blank">IEEE 802
// Committee's site</a>.
//
\end{ned}

One can also use the \ttt{<a href=..>} tag to create links within the page:

\begin{ned}
//
// See the <a href="#resources">resources</a> in this page.
// ...
// <a name="resources"><b>Resources</b></a>
// ...
//
\end{ned}

One can use the \texttt{<pre>..</pre>} HTML tag to insert source code examples
into the documentation. Line breaks and indentation will be preserved,
but HTML tags continue to be interpreted (they can be turned off with
\texttt{<nohtml>}, see later).

Example:

\begin{ned}
// <pre>
// // my preferred way of indentation in C/C++ is this:
// <b>for</b> (<b>int</b> i = 0; i < 10; i++) {
//     printf(<i>"%d\n"</i>, i);
// }
// </pre>
\end{ned}

will be rendered as

\begin{Verbatim}[commandchars=\\\{\}]
// my preferred way of indentation in C/C++ is this:
\textbf{for} (\textbf{int} i = 0; i < 10; i++) \{
    printf(\textit{"%d{\textbackslash}n"}, i);
\}
\end{Verbatim}

HTML is also the way to create tables. The example below

\begin{ned}
//
// <table border="1">
//   <tr>  <th>#</th> <th>number</th> </tr>
//   <tr>  <td>1</td> <td>one</td>    </tr>
//   <tr>  <td>2</td> <td>two</td>    </tr>
//   <tr>  <td>3</td> <td>three</td>  </tr>
// </table>
//
\end{ned}

will be rendered approximately as:

\begin{longtable}{|l|l|}
\hline
\tabheadcol
\tbf{\#} & \tbf{number} \\\hline
1 & one \\\hline
2 & two \\\hline
3 & three \\\hline
\end{longtable}


\subsection{Escaping HTML Tags}
\label{sec:neddoc:escaping-html-tags}

In some cases, one needs to turn off interpreting HTML tags (\ttt{<i>},
\ttt{<b>}, etc.) as formatting, and rather include them as literal text in
the generated documentation. This can be achieved by surrounding the text
with the \ttt{<nohtml>}...\ttt{</nohtml>} tag. For example,

\begin{ned}
// Use the <nohtml><i></nohtml> tag (like <tt><nohtml><i>this</i></nohtml><tt>)
// to write in <i>italic</i>.
\end{ned}

will be rendered as ``Use the <i> tag (like \texttt{<i>this</i>}) to write
in \textit{italic}.''

\ttt{<nohtml>}...\ttt{</nohtml>} will also prevent \fprog{opp\_neddoc}
from hyperlinking words that are accidentally the same as an existing
module or message name. Prefixing the word with a backslash will achieve
the same. That is, either of the following will do:

\begin{ned}
// In <nohtml>IP</nohtml> networks, routing is...
\end{ned}

\begin{ned}
// In \IP networks, routing is...
\end{ned}

Both will prevent hyperlinking the word \textit{IP} in case there is an
\ttt{IP} module in the project.



\section{Incorporating Extra Content}
\label{sec:neddoc:incorporating-extra-content}

\subsection{Adding a Custom Title Page}
\label{sec:neddoc:adding-custom-title-page}

The title page is the one that appears in the main frame after opening the
documentation in the browser. By default, it contains a boilerplate text
with the title \textit{``{\opp} Model Documentation''}. Model authors will
probably want to customize that, and at least change the title be more
specific.

A title page is defined with a \ttt{@titlepage} directive. It needs to
appear in a file-level comment.

\begin{note}
A file-level comment is one that appears at the top of a NED file, and is
separated from any other NED content by at least one \textit{blank line}.
\end{note}

While one can place the title page definition into any NED or MSG file, it is
probably a good idea to create a dedicated NED file for it. Lines up to the
next \ttt{@page} line or the end of the comment (whichever comes first) are
interpreted as part of the page.

The page should start with a title, as the documentation tool doesn't add
one. Use the \ttt{<h1>..</h1>} HTML tag for that.

Example:

\begin{ned}
//
// @titlepage
// <h1>Ethernet Model Documentation</h1>
//
// This documents the Ethernet model created by David Wu and refined by Andras
// Varga at CTIE, Monash University, Melbourne, Australia.
//
\end{ned}


\subsection{Adding Extra Pages}
\label{sec:neddoc:adding-extra-pages}

One can add new pages to the documentation using the \ttt{@page} directive.
\ttt{@page} may appear in any file-level comment, and has the following
syntax:

\begin{ned}
// @page filename.html, Title of the Page
\end{ned}

Choose a file name that doesn't collide with other files generated
by the documentation tool. If the file name does not end in \ttt{.html},
it will be appended. The page title will appear at the top of
the page as well as in the page index.

The lines after the \ttt{@page} line up to the next \ttt{@page} line or the
end of the comment will be used as the page body. One does not need to add
a title because the documentation tool automatically inserts the one
specified in the \ttt{@page} directive.

Example:
\begin{ned}
//
// @page structure.html, Directory Structure
//
// The model core model files and the examples have been placed
// into different directories. The <tt>examples/</tt> directory...
//
//
// @page examples.html, Examples
// ...
//
\end{ned}

One can create links to the generated pages using standard HTML,
using the \ttt{<a href="...">...</a>} tag. All HTML files are
placed in a single directory, so one doesn't have to worry about
directories.

Example:
\begin{ned}
//
// @titlepage
// ...
// The structure of the model is described <a href="structure.html">here</a>.
//
\end{ned}


\subsection{Incorporating Externally Created Pages}
\label{sec:neddoc:externally-created-pages}

The \ttt{@externalpage} directive allows one to add externally created
pages into the generated documentation. \ttt{@externalpage} may appear
in a file-level comment, and has a similar syntax as \ttt{@page}:

\begin{ned}
// @externalpage filename.html, Title of the Page
\end{ned}

The directive causes the page to appear in the page index. However, the
documentation tool does not check if the page exists, and it is the user's
responsibility to copy the file into the directory of the generated
documentation.

External pages can be linked to from other pages using the \ttt{<a
href="...">...</a>} tag.


\subsection{File Inclusion}
\label{sec:neddoc:file-inclusion}

The \ttt{@include} directive allows one to include the content of a file
into a documentation comment. \ttt{@include} expects file name or path; if
a relative path is given, it is interpreted as relative to the file that
includes it.

The line of the \ttt{@include} directive will be replaced by the
content of the file. The lines of the included file do not need
to start with \texttt{//}, but otherwise they are processed in the same way
as the NED comments. They can include other files, but circular
includes are not allowed.

\begin{ned}
// ...
// @include ../copyright.txt
// ...
\end{ned}

\subsection{Extending Type Pages with Extra Content}
\label{sec:neddoc:extending-type-pages}

Sometimes it is useful to customize the generated documentation pages that describe
NED and MSG types by adding extra content. It is possible to provide a documentation
fragment file in XML format which can be used by the documentation tool to add it to 
the generated documentation. 

The fragment file may contain multiple top-level \ttt{<docfragment>} 
elements in the XML file's root element. Each \ttt{<docfragment>} element must
have one of the \ttt{nedtype}, \ttt{msgtype} or \ttt{filename} attributes
depending on which page it extends. Additionally, it must provide an
\ttt{anchor} attribute to define a point in the page where the fragment's
content should be inserted. The content of the fragment must be provided in
a \ttt{<![CDATA[]]>} section.

\begin{xml}
<docfragments>
    <docfragment nedtype="fully.qualified.NEDTypeName" anchor="after-signals">
    <![CDATA[
        <h3 class="subtitle">Doc fragment after the signals section</h3>
        ...
    ]]>
    </docfragment>
    <docfragment msgtype="fully.qualified.MSGType" anchor="top">
    <![CDATA[
        <h3 class="subtitle">Doc fragment at the top of MSG type page</h3>
        ...
    ]]>
    </docfragment>
    <docfragment filename="project_relative_path/somefile.msg" anchor="bottom">
    <![CDATA[
        <h3 class="subtitle">Doc fragment at the end of the file listing page</h3>
        ...
    ]]>
    </docfragment>
</docfragments>
\end{xml}
    
Possible attribute values:
\begin{itemize}
    \item \texttt{nedtype}: Fully qualified NED type name. 
    \item \texttt{msgtype}: Fully qualified MSG type name.
    \item \texttt{filename}: Project relative file path of a NED or MSG source file. 
    The fragment will be inserted in the file's source listing page. 
    \item \texttt{anchor}: Specifies the place where the content should inserted.
    Possible values for NED type: \texttt{top, after-types, after-description,
    after-image, after-diagrams, after-usage, after-inheritance, after-parameters,
    after-properties, after-gates, after-signals, after-statistics, 
    after-unassigned-parameters, bottom}; for MSG type: \texttt{top, after-description,
    after-diagrams, after-inheritance, after-fields, after-properties, bottom};
    for file listing: \texttt{top, after-types, bottom}
\end{itemize}        

