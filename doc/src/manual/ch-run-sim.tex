\chapter{Running Simulations}
\label{cha:run-sim}

\section{Introduction}
\label{cha:run-sim:intro}

This chapter describes how to run simulations. It covers basic usage, user
interfaces, running simulation campaigns, and many other topics.

\section{Simulation Executables vs Libraries}
\label{sec:run-sim:running}

As we have seen in the \textit{Build} chapter, simulations may be compiled to an
executable or to a shared library. When the build output is an executable,
it can be run directly. For example, the Fifo example simulation can be
run with the following command:

\begin{commandline}
$ ./fifo
\end{commandline}

Simulations compiled to a shared library can be run using the \fprog{opp\_run}
program. For example, if we compiled the Fifo simulation to a
shared library on Linux, the build output would be a \ttt{libfifo.so} file that
could be run with the following command:

\begin{commandline}
$ opp_run -l fifo
\end{commandline}

The \fopt{-l} option tells \fprog{opp\_run} to load the given shared library.
The \fopt{-l} option will be covered in detail in section
\ref{sec:run-sim:loading-extra-libraries}.

\begin{note}
Normal simulation executables like the above \ttt{fifo} are also capable of
loading additional shared libraries in the same way. What's more, \fprog{opp\_run} is
actually nothing else but a specially-named simulation executable with no model
code in it.
\end{note}


\section{Command-Line Options}
\label{sec:run-sim:command-line-options}

The above commands illustrate just the simplest case. Usually you will
need to add extra command-line options, for example to specify what ini file(s)
to use, which configuration to run, which user interface to activate, where
to load NED files from, and so on. The rest of this chapter will cover
these options.

To get a complete list of command line options accepted by simulations,
run the \fprog{opp\_run} program (or any other simulation executable) with
\fopt{-h}:

\begin{commandline}
$ opp_run -h
\end{commandline}

Or:
\begin{commandline}
$ ./fifo -h
\end{commandline}


\section{Configuration Options on the Command Line}
\label{sec:run-sim:config-options-on-cmdline}

Configuration options can also be specified on the command line, not only in
ini files. To do so, prefix the option name with a double dash, and append the
value with an equal sign. Be sure not to have spaces around the equal sign. If
the value contains spaces or shell metacharacters, you'll need to protect the
value (or the whole option) with quotes or apostrophes.

Example:

\begin{commandline}
$ ./fifo --debug-on-errors=true
\end{commandline}

In case an option is specified both on the command line and in an ini file,
the command line takes precedence.

To get the list of all possible configuration options, use the \fopt{-h config}
option. (The additional \fopt{-s} option below just makes the output less
verbose.)

\begin{commandline}
$ opp_run -s -h config
Supported configuration options:
  **.bin-recording=<bool>, default:true; per-object setting
  check-signals=<bool>, default:true; per-run setting
  cmdenv-autoflush=<bool>, default:false; per-run setting
  cmdenv-config-name=<string>; global setting
  ...
\end{commandline}

To see the option descriptions as well, use \fopt{-h configdetails}.

\begin{commandline}
$ opp_run -h configdetails
\end{commandline}


\section{Specifying Ini Files}
\label{sec:run-sim:specifying-ini-files}

The default ini file is \ffilename{omnetpp.ini}, and is
loaded if no other ini file is given on the command line.

Ini files can be specified both as plain arguments and with the \fopt{-f}
option, so the following two commands are equivalent:

\begin{commandline}
$ ./fifo experiment.ini common.ini
$ ./fifo -f experiment.ini -f common.ini
\end{commandline}

Multiple ini files can be given, and their contents will be merged. This
allows for partitioning the configuration into separate files, for example
to simulation options, module parameters and result recording options.


\section{Specifying the NED Path}
\label{sec:run-sim:specifying-ned-path}

NED files are loaded from directories listed on the NED path. More precisely,
they are loaded from the listed directories and their whole subdirectory trees.
Directories are separated with a semicolon (\ttt{;}).

\begin{note}
Semicolon is used as separator on both Unix and Windows.
\end{note}

The NED path can be specified in several ways:
\begin{itemize}
  \item using the \ttt{NEDPATH} environment variable
  \item using the \fopt{-n} command-line option
  \item in ini files, with the \fconfig{ned-path} configuration option
\end{itemize}

NED path resolution rules are as follows:

\begin{enumerate}
  \item {\opp} checks for NED path specified on the command line with the \fopt{-n} option
  \item If not found on the command line, it checks for the \ttt{NEDPATH} environment variable
  \item The \fconfig{ned-path} option value from the ini file is appended to the result of the above steps
  \item If the result is still empty, it falls back to "." (the current directory)
\end{enumerate}


\section{Selecting a User Interface}
\label{sec:run-sim:selecting-user-interface}

{\opp} simulations can be run under different user interfaces a.k.a. runtime
environments. Currently the following user interfaces are supported:

\begin{itemize}
  \item Qtenv: Qt-based graphical user interface, available since {\opp} 5.0
  \item Cmdenv: command-line user interface for batch execution
\end{itemize}

You would typically test and debug your simulation under Qtenv,
then run actual simulation experiments from the command line or shell
script, using Cmdenv. Qtenv is also better suited for
educational and demonstration purposes.

User interfaces are provided in the form of libraries that can be linked with
statically, dynamically, or can be loaded at runtime.\footnote{Via the \fopt{-l}
option, see section \ref{sec:run-sim:loading-extra-libraries}} When several user
interface libraries are available in a simulation program, the user can select
via command-line or ini file options which one to use. In the absence of such an
option, the one with the highest priority will be started. Currently priorities
are set such that Qtenv has the highest priority, then Cmdenv.
By default, simulations are linked with all available user interfaces,
but this can be controlled via \fprog{opp\_makemake} options or in the {\opp}
global build configuration as well. The user interfaces available in a
simulation program can be listed by running it the \fopt{-h userinterfaces}
option.

You can explicitly select a user interface on the command line with the \fopt{-u}
option (specify \ttt{Qtenv} or \ttt{Cmdenv} as its argument), or by
adding the \fconfig{user-interface} option to the configuration. If both
the config option and the command line option are present, the command line option
takes precedence.

Since the graphical interfaces are the default (have higher priority), the most
common use of the \fopt{-u} option is to select Cmdenv, e.g. for batch execution.
The following example performs all runs of the Aloha example simulation using
Cmdenv:

\begin{commandline}
$ ./aloha -c PureAlohaExperiment -u Cmdenv
\end{commandline}


\section{Selecting Configurations and Runs}
\label{sec:run-sim:selecting-configuration-and-runs}

All user interfaces support the \fopt{-c <configname>} and \fopt{-r <runfilter>}
options for selecting which simulation(s) to run.

The \fopt{-c} option expects the name of an ini file configuration
as an argument. The \fopt{-r} option may be needed when the configuration
expands to multiple simulation runs. That is the case when the
configuration defines a \textit{parameter study} (see section
\ref{sec:config-sim:parameter-studies}), or when it contains
a \fconfig{repeat} configuration option that prescribes
multiple repetitions with different RNG seeds (see section
\ref{sec:config-sim:repeating-runs-with-different-seeds}).
The \fopt{-r} option can then be used to select a subset of all runs (or one
specific run, for that matter). A missing \fopt{-r} option selects all runs in
the given configuration.

It depends on the particular user interface how it interprets the
\fopt{-c} and \fopt{-r} options. Cmdenv performs all selected simulation runs
(optionally stopping after the first one that finishes with an error).
GUI interfaces like Qtenv may use this information to fill the
run selection dialog (or to set up the simulation automatically if there is
only one matching run.)


\subsection{Run Filter Syntax}
\label{sec:run-sim:selecting-configuration-and-runs:syntax}

The run filter accepts two syntaxes: a comma-separated list of run numbers or
run number ranges (for example \ttt{1,2,5-10}), or an arithmetic expression.
The arithmetic expression is similar to constraint expressions in the
configuration (see section \ref{sec:config-sim:constraint-expression}).
It may refer to iteration variables and to the repeat counter with the dollar
syntax: \ttt{\$numHosts}, \ttt{\$repetition}. An example: \ttt{\$numHosts>10
\&\& \$mean==2}.

Note that due to the presence of the dollar sign (and spaces), the expression
should be protected against shell expansion, e.g. using apostrophes:

\begin{commandline}
$ ./aloha -c PureAlohaExperiment -r '$numHosts>10 && $mean<2'
\end{commandline}

Or, with double quotes:

\begin{commandline}
$ ./aloha -c PureAlohaExperiment -r "\$numHosts>10 && \$mean<2"
\end{commandline}


\subsection{The Query Option}
\label{sec:run-sim:selecting-configuration-and-runs:queryoption}

The \fopt{-q} (query) option complements \fopt{-c} and \fopt{-r}, and
allows one to list the runs matched by the run filter.
\fopt{-q} expects an argument that defines the format and verbosity of the
output. Several formats are available: \ttt{numruns}, \ttt{runnumbers},
\ttt{runs}, \ttt{rundetails}, \ttt{runconfig}. Use \ttt{opp\_run -h} to get a
complete list.

\fopt{-q runs} prints one line of information with the iteration variables
about each run that the run filter matches. An example:

\begin{commandline}
$ ./aloha -s -c PureAlohaExperiment -r '$numHosts>10 && $mean<2' -q runs
Run 14: $numHosts=15, $mean=1, $repetition=0
Run 15: $numHosts=15, $mean=1, $repetition=1
Run 28: $numHosts=20, $mean=1, $repetition=0
Run 29: $numHosts=20, $mean=1, $repetition=1
\end{commandline}

The \fopt{-s} option just makes the output less verbose.

If you need more information, use \fopt{-q rundetails} or \fopt{-q runconfig}.
\ttt{rundetails} is like \ttt{numruns}, but it also prints the values of the
iteration variables and a summary of the configuration (the expanded
values of configuration entries that contain iteration variables)
for each matching run:

\begin{commandline}
$ ./aloha -s -c PureAlohaExperiment -r '$numHosts>10 && $mean<2' -q rundetails
Run 14: $numHosts=15, $mean=1, $repetition=0
    Aloha.numHosts = 15
    Aloha.host[*].iaTime = exponential(1s)

Run 15: $numHosts=15, $mean=1, $repetition=1
    Aloha.numHosts = 15
    Aloha.host[*].iaTime = exponential(1s)
...
\end{commandline}

The \ttt{numruns} and \ttt{runnumbers} formats are mainly intended for use in
scripts. They just print the number of matching runs and the plain run number
list, respectively.

\begin{commandline}
$ ./aloha -s -c PureAlohaExperiment -r '$numHosts>10 && $mean<2' -q numruns
4
$ ./aloha -s -c PureAlohaExperiment -r '$numHosts>10 && $mean<2' -q runnumbers
 14 15 28 29
\end{commandline}

The \fopt{-q} option encapsulates some unrelated functionality, as well:
\fopt{-q sectioninheritance} ignores \fopt{-r}, and prints the inheritance chain
of the  inifile sections (the inheritance graph after linearization) for the
configuration denoted by \fopt{-c}.


\section{Loading Extra Libraries}
\label{sec:run-sim:loading-extra-libraries}

{\opp} allows you to load shared libraries at runtime. These shared libraries
may contain model code (e.g. simple module implementation classes),
dynamically registered classes that extend the simulator's functionality (for
example NED functions, result filters/recorders, figures types, schedulers,
output vector/scalar writers, Qtenv inspectors, or even custom user interfaces),
or other code.

\begin{hint}
Building shared libraries and loading them dynamically has several
advantages over static linking or building executables. Advantages include
modularity, reduced build times (versus statically linking a huge executable),
and better reuse (being able to use the same library in several projects without
change).
\end{hint}

Libraries can be specified with the \fopt{-l <libraryname>} command line option
(there can be several \fopt{-l}'s on the command line), or with the \fconfig{load-libs}
configuration option. The values from the command line and the config file will
be merged.

The prefix and suffix from the library name can be omitted (the extensions
\ttt{.dll}, \ttt{.so}, \ttt{.dylib}, and also the common \ttt{lib} prefix
on Unix systems). This means that you can specify the library name in a
platform independent way: if you specify \fopt{-l foo}, then {\opp} will
look for \ttt{foo.dll}, \ttt{libfoo.dll}, \ttt{libfoo.so} or \ttt{libfoo.dylib},
depending on the platform.

{\opp} will use the \ffunc{dlopen()} or \ffunc{LoadLibrary()} system call to
load the library. To ensure that the system call finds the file, either
specify the library name with a full path (pre- and postfixes of the library
file name still can be omitted), or adjust the shared library path environment
variable of your OS: \ttt{PATH} on Windows, \ttt{LD\_LIBRARY\_PATH} on Unix,
and \ttt{DYLD\_LIBRARY\_PATH} on Mac OS X.

\begin{note}
  Runtime loading is not needed if your executable or shared lib was
  already linked against the library in question. In that case,
  the platform's dynamic loader will automatically load the library.
\end{note}


\section{Stopping Condition}
\label{sec:run-sim:stopping-condition}

The most common way of specifying when to finish the simulation is to set a
time limit. There are several time limits that can be set with the following
configuration options:

\begin{itemize}
  \item \fconfig{sim-time-limit} : Limits how long the simulation should run (in simulation time)
  \item \fconfig{cpu-time-limit} : Limits how much CPU time the simulation can use
  \item \fconfig{real-time-limit} : Limits how long the simulation can run (in real time)
\end{itemize}

\begin{note}
\fconfig{cpu-time-limit} and \fconfig{real-time-limit} may look
similar, but in practice, you'll almost always need \fconfig{cpu-time-limit} of
the two. Its alternative, \fconfig{real-time-limit} simply measures elapsed
time (wall-clock interval), so it does not imply how many cycles the CPU has
spent on running your simulation. On a heavily overloaded system where the CPU
is shared among a number of computationally intensive jobs,
\fconfig{real-time-limit} may stop your simulation much too early.
\end{note}

An example:

\begin{commandline}
$ ./fifo --sim-time-limit=500s
\end{commandline}

If several time limits are set together, the simulation will stop when the first
one is hit.


If needed, the simulation may also be stopped programmatically, for example when
results of a (steady-state) simulation have reached the desired accuracy.
This can be done by calling the \ffunc{endSimulation()} method.


\section{Controlling the Output}
\label{sec:run-sim:output-control}

The following options can be used to enable/disable the creation of various
output files during simulation.

\begin{itemize}
  \item \fconfig{record-eventlog} : Turns on the recording of the simulator
        events into an event log file. The resulting \ttt{.elog} file can be
        analyzed later in the IDE with the Sequence Chart tool.

  \item \fconfig{scalar-recording} : This option is originally a
        per-object setting, intended for selectively turning on or off the
        recording of certain scalar results. However, when it is specified
        globally to turn off all scalars, no output scalar file
        (\ttt{.sca}) will be created either.

  \item \fconfig{vector-recording} : Similar to \fconfig{scalar-recording},
        this option can be used to turn off creating an output vector file
        (\ttt{.vec}).

  \item \fconfig{cmdenv-redirect-output} : This is a Cmdenv-specific option,
        only mentioned here for completeness. It tells Cmdenv to save its
        standard output to files, one file per run. This option is mainly
        helpful when running simulation batches.

\end{itemize}

These configuration options, like any other, can be specified both in ini
files and on the command line. An example:

\begin{commandline}
$ ./fifo --record-eventlog=true --scalar-recording=false --vector-recording=false
\end{commandline}


\section{Debugging}
\label{sec:run-sim:debugging-support}

Debugging is a task that comes up often during model development. The following
configuration options are related to C++ debugging:

\begin{itemize}
  \item \fconfig{debug-on-errors} : If the runtime detects any error, it will
        trigger a debugger trap (programmatic breakpoint) so you will be able
        to check the location and the context of the problem in your debugger.
        This option does not start a debugger, the simulation must already
        have been launched under a debugger.

  \item \fconfig{debugger-attach-on-error} : Controls just-in-time debugging.
        When this option is enabled and an error occurs during simulation, the
        simulation program will launch an external debugger, and have it
        attached to the simulation process. Related configuration options are
        \fconfig{debugger-attach-on-startup}, \fconfig{debugger-attach-command}
        and \fconfig{debugger-attach-wait-time}.

        \begin{hint}
        Just-in-time debugging is useful when trying to debug a rarely occurring
        crash in a large simulation batch, or in cases where the simulation is
        started from a script or another program that cannot be easily modified
        to start the simulation in a debugger.
        \end{hint}

\end{itemize}

An example that launches the simulation under the \fprog{gdb} debugger:

\begin{commandline}
$ gdb --args ./aloha --debug-on-errors=true
\end{commandline}


\section{Debugging Leaked Messages}
\label{sec:run-sim:leaked-messages}

The most common cause of memory leaks in {\opp} simulations is forgetting to
delete messages. When this happens, the simulation process will continually grow
in size as the simulation progresses, and when left to run long enough, it will
eventually cause an out-of-memory condition.

Luckily, this problem is easy to indentify, as all user interfaces display the
number of message objects currently in the system. Take a look at the following
example Cmdenv output:

\begin{commandline}
...
** Event #1908736   t=58914.051870113485   Elapsed: 2.000s (0m 02s)
     Speed:     ev/sec=954368   simsec/sec=29457   ev/simsec=32.3987
     Messages:  created: 561611   `\tbf{present:\ 21}`   in FES: 34
** Event #3433472   t=106067.401570204991   Elapsed: 4.000s (0m 04s)
     Speed:     ev/sec=762368   simsec/sec=23576.7   ev/simsec=32.3357
     Messages:  created: 1010142   `\tbf{present:\ 354}`   in FES: 27
** Event #5338880   t=165025.763387178965   Elapsed: 6.000s (0m 06s)
     Speed:     ev/sec=952704   simsec/sec=29479.2   ev/simsec=32.3179
     Messages:  created: 1570675   `\tbf{present:\ 596}`   in FES: 21
** Event #6850304   t=211763.433233042017   Elapsed: 8.000s (0m 08s)
     Speed:     ev/sec=755712   simsec/sec=23368.8   ev/simsec=32.3385
     Messages:  created: 2015318   `\tbf{present:\ 732}`   in FES: 38
** Event #8753920   t=270587.781554343184   Elapsed: 10.000s (0m 10s)
     Speed:     ev/sec=951808   simsec/sec=29412.2   ev/simsec=32.361
     Messages:  created: 2575634   `\tbf{present:\ 937}`   in FES: 32
** Event #10270208   t=317495.244698246477   Elapsed: 12.000s (0m 12s)
     Speed:     ev/sec=758144   simsec/sec=23453.7   ev/simsec=32.3251
     Messages:  created: 3021646   `\tbf{present:\ 1213}`   in FES: 20
...
\end{commandline}

The interesting parts are in bold font. The steadily increasing numbers are an
indication that the simulation model, i.e. one or more modules in it, are
missing some \ttt{delete msg} calls. It is best to use Qtenv to narrow
down the issue to specific modules and/or message types.

Qtenv is also able to display the number of messages currently
in the simulation. The numbers are displayed on the status bar. If you find that
the number of messages is steadily increasing, you need to find where the
message objects are located. This can be done with the help of the
\textit{Find/Inspect Objects} dialog.

If the number of messages is stable, it is still possible that the simulation
is leaking other \cclass{cObject}-based objects; they can also be found
using the \textit{Find/Inspect Objects} dialog.

If the simulation is leaking non-{\opp} objects (i.e. not something derived
from \cclass{cObject}) or other memory blocks, Cmdenv and Qtenv cannot
help in tracking down the issue.


\section{Debugging Other Memory Problems}
\label{sec:run-sim:memory-leaks-and-crashes}

Technically, memory leaks are only a subset of problems associated with memory
allocations, i.e. the usage of \ttt{new} and \ttt{delete} in C++.

\begin{itemize}
   \item \textit{memory leaks,} that is, forgetting to delete objects
     or memory blocks no longer used, usually just prevents the user from
     being able to run the simulation program long enough;
   \item \textit{dereferencing dangling pointers}, i.e. accessing
    an already deleted object or memory block (or trying to delete one
    for a second time) usually results in a crash;
   \item \textit{heap corruption}, caused by e.g. writing past the end of
   an allocated array, usually also results in a crash.
\end{itemize}


There are specialized tools that can help in tracking down memory allocation
problems (memory leak, double-deletion, referencing deleted blocks, etc). Some
of these tools are listed below.

\begin{itemize}
  \item \textit{Valgrind}, our primary recommendation, is a CPU emulator and
        memory debugger tool for Linux.
  \item Other memory debugger libraries/tools include \textit{MemProf},
        \textit{MPatrol}, \textit{dmalloc} and \textit{ElectricFence}.
        Most of these tools support tracking down memory leaks as well as
        detecting double deletion, writing past the end of an allocated block,
        etc.
  \item There are several commercial offerings as well, e.g. \textit{Purify}
        and \textit{Insure++}.
\end{itemize}


\section{Profiling}
\label{sec:run-sim:profiling}

When a simulation runs correctly but is too slow, you might want to
\textit{profile} it. Profiling basically means collecting runtime
information about how much time is spent at various parts of the
program, in order to find places where optimizing the code would have
the most impact.

However, there are a few other options you can try before resorting to profiling
and optimizing. First, verify that it  is the simulation itself that is slow.
Make sure features like eventlog recording is not accidentally turned on. Run
the simulation under Cmdenv to eliminate any possible overhead from Qtenv.
If you must run the simulation under Qtenv, you can still gain speed by
disabling animation features, closing all inspectors, hiding UI elements like
the timeline, and so on.

Also, compile your code in release mode (with \ttt{make MODE=release}, see
\ref{sec:build-sim-progs:debug-and-release-builds}) instead of debug. That
can make a huge difference, especially with heavily templated code.

\begin{hint}
If you decide to optimize the program, we recommend that you don't skip the
profiling step. Even for experienced programmers, a profiling session is often
full of surprises, and CPU time is spent at other places than one would
expect.
\end{hint}

Some profiling software:

\begin{itemize}
  \item \tbf{Debuggers}. A very simple but frequently useful way of profiling is
        stopping the program in a debugger from time to time, and looking at the
        stack trace before resuming (manual statistical profiling). If
        the program always stops at the same place in the code, that might be
        the bottleneck.
  \item \tbf{Valgrind/KCachegrind}. KCachegrind is a graphical visualizer
        for traces generated by \textit{valgrind} and its \textit{callgrind}
        tool in Linux. These tools are free and open source software,
        packaged with most Linux distributions.
  \item There are also commercial C/C++ profilers like RotateRight's Zoom.
        Profilers are also part of larger packages like PurifyPlus or
        Parasoft Insure++.
\end{itemize}


\section{Checkpointing}
\label{sec:run-sim:checkpointing}

Debugging long-running simulations can be challenging, as it often requires
running the simulation for extended periods before reaching the point of failure
and commencing debugging.

Checkpointing can significantly ease the debugging process by enabling the
creation of snapshots of the program's state, allowing for the resumption of
execution from these checkpoints, even multiple times. Unfortunately, {\opp}
itself does not natively include checkpointing functionality; however, this
capability is available through external tools. It should be noted that
restoring GUI windows is typically not supported by these tools.

Currently, the dominant and actively maintained checkpointing software on Linux
is CRIU (Checkpoint/Restore In Userspace). CRIU offers a user-space
checkpointing library, which has gained widespread adoption due to its
reliability and continued development.\footnote{Other checkpointing packages
include BLCR (Berkeley Lab Checkpoint/Restart) and DMTCP (Distributed
MultiThreaded Checkpointing), but these tools have become obsolete and have not
received updates for several years.}

Furthermore, it is worth mentioning that Docker and its underlying technologies
also incorporate a checkpoint and restore mechanism, providing additional
options for checkpointing long-running applications.

An example session with CRIU:

\begin{commandline}
$ ./aloha -u Cmdenv -c PureAloha2 --cmdenv-redirect-output=true &
$ pid=$!  # remember process ID
...
...
$ mkdir checkpoint1
$ sudo criu --shell-job dump -t $pid -D ./checkpoint1
...
...
$ sudo criu --shell-job restore -D ./checkpoint1
\end{commandline}


\section{Using Cmdenv}
\label{sec:run-sim:cmdenv}

Cmdenv\index{Cmdenv} is a lightweight, command line user interface that
compiles and runs on all platforms. Cmdenv is designed primarily for batch
execution.

Cmdenv simply executes some or all simulation runs that are described
in the configuration file. The runs to be executed can be
passed via command-line arguments or configuration options.

Cmdenv runs simulations in the same process. This means that e.g. if one
simulation run writes a global variable, subsequent runs will also
see the change. This is one reason why global variables in models are
strongly discouraged.

\subsection{Sample Output}
\label{sec:run-sim:cmdenv:sample-output}

When you run the Fifo example under Cmdenv, you should see
something like this:

\begin{commandline}
$ ./fifo -u Cmdenv -c Fifo1

OMNeT++ Discrete Event Simulation  (C) 1992-2017 Andras Varga, OpenSim Ltd.
Version: 5.0, edition: Academic Public License -- NOT FOR COMMERCIAL USE
See the license for distribution terms and warranty disclaimer
Setting up Cmdenv...
Loading NED files from .: 5

Preparing for running configuration Fifo1, run #0...
Scenario: $repetition=0
Assigned runID=Fifo1-0-20090104-12:23:25-5792
Setting up network 'FifoNet'...
Initializing...
Initializing module FifoNet, stage 0
Initializing module FifoNet.gen, stage 0
Initializing module FifoNet.fifo, stage 0
Initializing module FifoNet.sink, stage 0

Running simulation...
** Event #1   t=0   Elapsed: 0.000s (0m 00s)  0% completed
     Speed:     ev/sec=0   simsec/sec=0   ev/simsec=0
     Messages:  created: 2   present: 2   in FES: 1
** Event #232448   t=11719.051014922336   Elapsed: 2.003s (0m 02s)  3% completed
     Speed:     ev/sec=116050   simsec/sec=5850.75   ev/simsec=19.8351
     Messages:  created: 58114   present: 3   in FES: 2
...
** Event #7206882   t=360000.52066583684   Elapsed: 78.282s (1m 18s)  100% completed
     Speed:     ev/sec=118860   simsec/sec=5911.9   ev/simsec=20.1053
     Messages:  created: 1801723   present: 3   in FES: 2

<!> Simulation time limit reached -- simulation stopped.

Calling finish() at end of Run #0...
End.
\end{commandline}

As Cmdenv runs the simulation, it periodically prints the sequence number
of the current event, the simulation time, the elapsed (real) time,
and the performance of the simulation (how many events are processed per
second; the first two values are 0 because there wasn't enough data
for it to calculate yet). At the end of the simulation, the \ffunc{finish()}
methods of the simple modules are run, and the outputs from them are displayed.


\subsection{Selecting Runs, Batch Operation}
\label{sec:run-sim:cmdenv-config-options}

The most important command-line options for Cmdenv are \fopt{-c} and
\fopt{-r} for selecting which simulations to perform. (They were described
in section \ref{sec:run-sim:selecting-configuration-and-runs}.) They also have
their equivalent configuration options that can be written in files as well:
\fconfig{cmdenv-config-name} and \fconfig{cmdenv-runs-to-execute}.

Another configuration option, \fconfig{cmdenv-stop-batch-on-error} controls
Cmdenv's behavior when performing multiple runs: it determines
whether Cmdenv should stop after the first run that finishes with an
error. By default, it does.

When performing multiple runs, Cmdenv prints run statistics at the end. Example
output:

\begin{filelisting}
$ ./aloha -c PureAlohaExperiment -u Cmdenv
...
Run statistics: total 42, successful 30, errors 1, skipped 11
\end{filelisting}


\subsection{Express Mode}
\label{sec:run-sim:cmdenv:express-mode}

Cmdenv can execute simulations in two modes:

\begin{itemize}
    \item \tbf{Normal} (non-express) mode is for debugging; detailed information
        will be written to the standard output (event banners, module log,
        etc).
    \item \tbf{Express} mode can be used for long simulation runs; only
        periodical status updates are displayed about the progress of the
        simulation.
\end{itemize}

The default mode is Express. To turn off Express mode, specify \ttt{false} for
the \fconfig{cmdenv-express-mode} configuration option:

\begin{commandline}
$ ./fifo -u Cmdenv -c Fifo1 --cmdenv-express-mode=false
\end{commandline}

There are several other options that also affect Express-mode and Normal
mode behavior:

\begin{itemize}
  \item Express: \fconfig{cmdenv-performance-display}, \fconfig{cmdenv-status-frequency}
  \item Normal: \fconfig{cmdenv-event-banners}, \fconfig{cmdenv-event-banner-details},
        \fconfig{cmdenv-log-level}, \fconfig{cmdenv-log-prefix}, etc.
\end{itemize}

See Appendix \ref{cha:config-options} for more information about these options.

\subsubsection{Interpreting Express-Mode Output}
\label{sec:run-sim:cmdenv:express-mode:output}

When the simulation is running in Express mode with detailed
performance display enabled (\ttt{cmdenv-performance-display=true}), Cmdenv
periodically outputs a three-line status report about the progress of the simulation.
The output looks like this:

\begin{commandline}
...
** Event #250000   t=123.74354 ( 2m  3s)    Elapsed: 0m 12s
     Speed:     ev/sec=19731.6   simsec/sec=9.80713   ev/simsec=2011.97
     Messages:  created: 55532   present: 6553   in FES: 8
** Event #300000   t=148.55496 ( 2m 28s)    Elapsed: 0m 15s
     Speed:     ev/sec=19584.8   simsec/sec=9.64698   ev/simsec=2030.15
     Messages:  created: 66605   present: 7815   in FES: 7
...
\end{commandline}

The first line of the status display (beginning with \ttt{**})
contains:

\begin{itemize}
   \item how many events have been processed so far
   \item the current simulation time (t), and
   \item the elapsed time (wall clock time) since the beginning of the simulation run.
\end{itemize}

The second line displays simulation performance metrics:

\begin{itemize}
   \item \ttt{ev/sec} indicates \textit{performance}: how many events are
     processed in one real-time second.  On one hand it depends on your hardware
     (faster CPUs process more events per second), and on the other hand
     it depends on the complexity (amount of calculations) associated
     with processing one event. For example, protocol simulations tend to require
     more processing per event than e.g. queueing networks, thus the latter
     produce higher ev/sec values. In any case, this value is largely
     independent of the size of your model, i.e. the number of modules in it.
   \item \ttt{simsec/sec} shows \textit{relative speed} of the simulation, that
     is, how fast the simulation is progressing compared to real time, how many
     simulated seconds can be done in one real second. This value virtually depends
     on everything: on the hardware, on the size of the simulation model,
     on the complexity of events, and the average simulation time between events as well.
   \item \ttt{ev/simsec} is the \textit{event density}: how many events are
     there per simulated second. Event density only depends on the simulation model,
     regardless of the hardware used to simulate it: in a high-speed
     optical network simulation you will have very high values ($10^9$),
     while in a call center simulation this value is probably well
     under 1. It also depends on the size of your model: if you double the
     number of modules in your model, you can expect the event density to
     double, too.
\end{itemize}

The third line displays the number of messages, and it is important
because it may indicate the ``health'' of your simulation.

\begin{itemize}
   \item{\ttt{Created}: total number of message objects created since the
     beginning of the simulation run. This does not mean that this many message
     object actually exist, because some (many) of them may have been deleted
     since then. It also does not mean that \textit{you} created all those
     messages -- the simulation kernel also creates messages for its own use
     (e.g. to implement \ttt{wait()} in an \ttt{activity()} simple module).}
   \item{\ttt{Present}: the number of message objects currently present
     in the simulation model, that is, the number of messages created (see above)
     minus the number of messages already deleted. This number includes
     the messages in the FES\index{FES}.}
   \item{\ttt{In FES}: the number of messages currently scheduled in the
     Future Event Set.}
\end{itemize}


The second value, the number of messages present, is more useful than
perhaps one would initially think. It can be an indicator of the ``health'' of the simulation;
if it is growing steadily, then either you have a memory leak and losing
messages (which indicates a programming error), or the network you simulate is
overloaded and queues are steadily filling up (which might indicate wrong input
parameters).

Of course, if the number of messages does not increase, it does not mean
that you do \textit{not} have a memory leak (other memory leaks are also
possible). Nevertheless the value is still useful, because by far the
most common way of leaking memory in a simulation is by not deleting messages.

\subsection{Other Options}
\label{sec:run-sim:cmdenv:other-options}

Cmdenv has more configuration options than mentioned in this section; see
the options beginning with \ttt{cmdenv-} in Appendix \ref{cha:config-options}
for the complete list.


\section{The Qtenv Graphical User Interface}
\label{sec:run-sim:qtenv}

Qtenv\index{Qtenv} is a runtime simulation GUI. Qtenv supports
interactive simulation execution, animation, tracing and debugging.
Qtenv is recommended in the development stage of a simulation and for
presentation purposes, since it allows one to get a detailed picture of the
state  of simulation at any point of execution and to follow what happens
inside the network. Note that 3D visualization support and smooth animation
support are only available in Qtenv.

\begin{note}
This section only covers the command-line and configuration options
of Qtenv; the user interface is described in the Qtenv chapter of the
{\opp} User Guide.
\end{note}

\subsection{Command-Line and Configuration Options}
\label{sec:run-sim:qtenv-options}

Simulations run under Qtenv accept all general command line
and configuration options, including \fopt{-c} and \fopt{-r}.
The configuration options specific to Qtenv include:

\begin{itemize}
  \item \fconfig{qtenv-default-config}:
    Specifies which config Qtenv should set up automatically on
    startup. The default is to ask the user. This option is equivalent to the
    \fopt{-c} command-line option.

  \item \fconfig{qtenv-default-run}: Specifies which run (of the default
    config, see qtenv-default-config) Qtenv should set up automatically on startup.
    The default is to ask the user. This option is equivalent to the \fopt{-r}
    command-line option.

  \item \fconfig{qtenv-extra-stack}:
    Specifies the extra amount of stack that is reserved for each activity()
    simple module when the simulation is run under Qtenv.
\end{itemize}

Qtenv is also affected by the following option:

\begin{itemize}
  \item \fconfig{image-path}: Specifies the path for loading module icons.
\end{itemize}

See Appendix \ref{cha:config-options} for the list of possible configuration options.


\section{Running Simulation Campaigns}
\label{sec:run-sim:simulation-campaigns}

Once your model works reliably, you will usually want to run several
simulations, either to explore the parameter space via a \textit{parameter
study} (see section \ref{sec:config-sim:parameter-studies}), or to do
multiple repetitions with different RNG seeds to increase the statistical
accuracy of the results (see section
\ref{sec:config-sim:repeating-runs-with-different-seeds}).

In this section, we will explore several ways to run batches of
simulations efficently.

\subsection{The Naive Approach}
\label{sec:run-sim:campaigns-naive-approach}

Assume that you want to run the parameter study in the Aloha example
simulation for the $numHosts>15$ cases.

The first idea is that Cmdenv is capable of running simulation batches.
The following command will do the job:

\begin{commandline}
$ ./aloha -u Cmdenv -c PureAlohaExperiment -r '$numHosts>15'
...
Run statistics: total 14, successful 14
End.
\end{commandline}

It works fine. However, this approach has some drawbacks which becomes
apparent when running hundreds or thousands of simulation runs.

\begin{enumerate}
  \item It uses only one CPU. In the age of multi-core CPUs, this is not
        very efficient.
  \item More prone to C++ programming errors in the model. A failure in a single
        run may abort execution (segfault) or corrupt the process
        state, possibly invalidating the results of subsequent runs.
\end{enumerate}

To address the second drawback, we can execute each simulation run in its own
Cmdenv instance.

\begin{commandline}
$ ./aloha -c PureAlohaExperiment -r '$numHosts>15' -s -q runnumbers
28 29 30 31 32 33 34 35 36 37 38 39 40 41
$ ./aloha -u Cmdenv -c PureAlohaExperiment -r 28
$ ./aloha -u Cmdenv -c PureAlohaExperiment -r 29
$ ./aloha -u Cmdenv -c PureAlohaExperiment -r 30
...
$ ./aloha -u Cmdenv -c PureAlohaExperiment -r 41
\end{commandline}

It's a lot of commands to issue manually, but luckily it can be automated with a
shell script like this:

\begin{filelisting}
#! /bin/sh
RUNS=$(./aloha -c PureAlohaExperiment -r '$numHosts>15' -s -q runnumbers)
for i in $RUNS; do
    ./aloha -u Cmdenv -c PureAlohaExperiment -r $i
done
\end{filelisting}

Save the above into a text file called e.g. \ttt{runAloha}. Then give it
executable permission, and run it:

\begin{commandline}
$ chmod +x runAloha
$ ./runAloha
\end{commandline}

It will execute the simulations one-by-one, each in its own Cmdenv instance.

This approach involves a process start overhead for each simulation.
Normally, this overhead is small compared to the time spent simulating. However,
it may become more of a problem when running a large number of very short
simulations (<<1s in CPU time). This effect may be mitigated by letting Cmdenv
do several (e.g. 10) simulations in one go.

And then, the script still uses only one CPU. It would be better to keep
all CPUs busy. For example, if you have 8 CPUs, there should be eight processes
running all the time -- when one terminates, another would be launched in its place.
You might notice that this behavior is similar to what GNU Make's
\fopt{-j<numJobs>} option does. The \fprog{opp\_runall} utility, to be covered
in the next section, exploits GNU Make to schedule the running of simulations on
multiple CPUs.


\subsection{Using opp\_runall}
\label{sec:run-sim:batches-using-opp-runall}

{\opp} has a utility program called \fprog{opp\_runall}, which allows you to
execute simulations using multiple CPUs and multiple processes.

\fprog{opp\_runall} groups simulation runs into batches. Every batch
corresponds to a Cmdenv process, that is, runs of a batch execute sequentially
inside the same Cmdenv process. Batches (i.e. Cmdenv instances) are scheduled
for running so that they keep all CPUs busy. The batch size as well as the
number of CPUs to use have sensible defaults but can be overridden.

\subsubsection{Command Line}
\label{sec:run-sim:opp-runall:comandline}

\fprog{opp\_runall} expects the normal simulation command in its argument list.
The first positional (non-option) argument and all following arguments are
treated as the simulation command (simulation program and its arguments).

Thus, to modify a normal Cmdenv simulation command to make use of multiple
CPUs, simply prefix it with \ttt{opp\_runall}:

\begin{commandline}
$ opp_runall ./aloha -u Cmdenv -c PureAlohaExperiment -r '$numHosts>15'
\end{commandline}

Options intended for \fprog{opp\_runall} should come before the the simulation
command. These options include \fopt{-b<N>} for specifying the batch size, and
\fopt{-j<N>} to specify the number of CPUs to use.

\begin{commandline}
$ opp_runall -j8 -b4 ./aloha -u Cmdenv -c PureAlohaExperiment -r '$numHosts>15'
\end{commandline}


\subsubsection{How It Works}
\label{sec:run-sim:opp-runall:operation}

First, \fprog{opp\_runall} invokes the simulation command with extra command
arguments (\ttt{-s -q runnumbers}) to figure out the list of runs it needs to
perform, and groups the run numbers into batches. Then it exploits GNU make and
its \fopt{-j<N>} option to do the heavy lifting. Namely, it generates a
temporary makefile that allows \fprog{make} to run batches in parallel, and
invokes \fprog{make} with the appropriate \ttt{-j} option. It is also possible
to export the makefile for inspection and/or running it manually.

To illustrate the above, here is the content of such a makefile:

\begin{filelisting}
#
# This makefile was generated with the following command:
# opp_runall -j2 -b4 -e tmp ./aloha -u Cmdenv -c PureAlohaExperiment -r $numHosts>15
#

SIMULATIONCMD = ./aloha -u Cmdenv -c PureAlohaExperiment -s \
                --cmdenv-redirect-output=true
TARGETS =  batch0 batch1 batch2 batch3

.PHONY: $(TARGETS)

all: $(TARGETS)
    @echo All runs completed.

batch0:
    $(SIMULATIONCMD) -r 28,29,30,31

batch1:
    $(SIMULATIONCMD) -r 32,33,34,35

batch2:
    $(SIMULATIONCMD) -r 36,37,38,39

batch3:
    $(SIMULATIONCMD) -r 40,41
\end{filelisting}


\subsection{Exploiting Clusters}
\label{sec:run-sim:opp-runall:exploiting-clusters}

With large scale simulations, using one's own desktop computer might not be
enough. The solution could be to run the simulation on remote machines,
that is, to employ a computing cluster.

In simple setups, cross-mounting the file system that contains {\opp} and the
model, and using \ttt{ssh} to run the simulations might already provide a good
solution.

In other cases, submitting simulation jobs and harvesting the results might be
done via batch-queuing, cluster computing or grid computing middleware. The
following list contains some pointers to such software:

\begin{itemize}

  \item \tbf{HTCondor}, previously called \tbf{Condor}, is an open source
      software package that enables High Throughput Computing (HTC)
      on large collections of distributively owned computing resources.
      HTCondor can manage a dedicated cluster of workstations, and it can also
      harness non-dedicated, preexisting resources under distributed ownership.
      A user can submit jobs to HTCondor. HTCondor finds an available machine
      on the network and begins running the job on that machine. HTCondor also
      supports checkpointing and migrating jobs.

\item \tbf{Open Grid Scheduler/Grid Engine} is a commercially supported
      open-source batch-queuing system for distributed resource management.
      OGS/GE is based on Sun Grid Engine (SGE), and maintained by the same group
      of external (i.e. non-Sun) developers who started contributing code since
      2001. There is also a commercial SGE successor, \tbf{Univa Grid Engine},
      formerly called Oracle Grid Engine.

\item \tbf{Slurm Workload Manager}, or Slurm, is a free and open-source job
      scheduler for Linux and Unix-like kernels, used by many of the world's
      supercomputers and computer clusters.

\item \tbf{Apple's Xgrid} has unfortunately been removed from Mac OS X with the
     release of Mountain Lion (2012). Xgrid was distributed computing for
     the masses -- easy, plug and play, not complicated. You could network your
     Mac computers together, and use that power on one computer to do something
     that took a lot of computing power. Currently, Pooch is advertised
     as software providing the easiest way to assemble and operate a
     high-performance parallel computer from Macs.

\end{itemize}



\section{Akaroa Support: Multiple Replications in Parallel}
\label{sec:run-sim:akaroa}
\index{Akaroa}
\index{Multiple Replications in Parallel}

\subsection{Introduction}
\label{sec:run-sim:akaroa-introduction}

Typical simulations are Monte-Carlo simulations: they use
(pseudo-)random numbers to drive the simulation model.
For the simulation to produce statistically reliable results,
one has to carefully consider the following:

\begin{itemize}
  \item{When the initial transient is over, when can we start
    collecting data? We usually don't want to include the
    initial transient when the simulation is still ``warming up.''}
  \item{When can we stop the simulation? We want to wait long enough
    so that the statistics we are collecting can ``stabilize'',
    or reach the required sample size to be statistically trustable.}
\end{itemize}

Neither question is trivial to answer. One might just suggest
to wait ``very long'' or ``long enough''. However, this is neither
simple (how do you know what is ``long enough''?) nor practical
(even with today's high speed processors simulations of modest complexity
can take hours, and one may not afford multiplying runtimes by,
say, 10, ``just to be safe.'') If you need further convincing,
please read \cite{Pawlikowsky02} and be horrified.

A possible solution is to look at the statistics while the simulation
is running, and decide at runtime when enough data have been
collected for the results to have reached the required accuracy.
One possible criterion is given by the confidence level,
more precisely, by its width relative to the mean.
But ex ante it is unknown how many observations have to be collected
to achieve this level -- it must be determined at runtime.


\subsection{What Is Akaroa}
\label{sec:run-sim:what-is-akaroa}

Akaroa \cite{Akaroa99} addresses the above problem.
According to its authors, Akaroa (Akaroa2) is a ``fully automated
simulation tool designed for running distributed stochastic simulations
in MRIP scenario'' in a cluster computing environment.

MRIP stands for \textit{Multiple Replications in Parallel}.
In MRIP, the computers of the cluster run independent replications
of the whole simulation process (i.e. with the same parameters but
different seed for the RNGs (random number generators)),
generating statistically equivalent streams of simulation output data.
These data streams are fed to a global data analyser responsible for
analysis of the final results and for stopping the simulation
when the results reach a satisfactory accuracy.

The independent simulation processes run independently of one another
and continuously send their observations to the central analyser
and control process. This process \textit{combines} the independent data streams,
and calculates from these observations an overall estimate of the mean value
of each parameter.
Akaroa2 decides by a given confidence level and precision
whether it has enough observations or not. When it judges that is
has enough observations it halts the simulation.

If \textit{n} processors are used, the needed simulation execution time
is usually \textit{n} times smaller compared to a one-processor
simulation (the required number of observations are produced sooner).
Thus, the simulation would be sped up approximately in proportion
to the number of processors used and sometimes even more.

Akaroa was designed at the University of Canterbury in Christchurch, New Zealand
and can be used free of charge for teaching and non-profit research activities.


\subsection{Using Akaroa with {\opp}}
\label{sec:run-sim:using-akaroa}

\subsubsection{Starting Akaroa}
\label{sec:run-sim:starting-up-akaroa}

Before the simulation can be run in parallel under Akaroa, you have to
start up the system:

\begin{itemize}
  \item{Start \ttt{akmaster} running in the background on some host.}
  \item{On each host where you want to run a simulation engine,
     start \ttt{akslave} in the background.}
\end{itemize}

Each \ttt{akslave} establishes a connection with the \ttt{akmaster}.

Then you use \ttt{akrun} to start a simulation. \ttt{akrun} waits
for the simulation to complete, and writes a report of the results
to the standard output. The basic usage of the \ttt{akrun} command is:

\begin{commandline}
$ akrun -n num_hosts command [argument..]
\end{commandline}

where \textit{command} is the name of the simulation you want to start.
Parameters for Akaroa are read from the file named \ttt{Akaroa} in
the working directory. Collected data from the processes are
sent to the \ttt{akmaster} process, and when the required precision
has been reached, \ttt{akmaster} tells the simulation processes to
terminate. The results are written to the standard output.

The above description is not detailed enough to help you
set up and successfully use Akaroa -- for that you need to read the
Akaroa manual.

\subsubsection{Configuring {\opp} for Akaroa}
\label{sec:run-sim:configuring-akaroa}

First of all, you have to compile {\opp} with Akaroa support enabled.

The {\opp} simulation must be configured in \ffilename{omnetpp.ini}
so that it passes the observations to Akaroa. The simulation model itself does
not need to be changed -- it continues to write
the observations into output vectors (\cclass{cOutVector} objects,
see chapter \ref{cha:sim-lib}). You can place some of
the output vectors under Akaroa control.

You need to add the following to \ffilename{omnetpp.ini}:

\begin{inifile}
[General]
rng-class = "cAkaroaRNG"
outputvectormanager-class = "cAkOutputVectorManager"
\end{inifile}

These lines cause the simulation to obtain random numbers from Akaroa,
and allows data written to selected output vectors to be passed to Akaroa's
global data analyser.
    \footnote{For more details on the plugin mechanism these settings make use of,
    see \ref{cha:plugin-exts}.}

Akaroa's RNG is a Combined Multiple Recursive pseudorandom
number generator (CMRG) with a period of approximately $2^{191}$
random numbers, and provides a unique stream of random numbers
for every simulation engine.

\begin{note}
It is vital that you obtain random numbers from Akaroa; otherwise,
all simulation processes will run with the same RNG seeds, and
produce exactly the same results.
\end{note}

Then you need to specify which output vectors you want to
be under Akaroa control (by default, none of them are).
You can use the \ttt{*}, \ttt{**} wildcards (see
section \ref{sec:config-sim:wildcards}) to
place certain vectors under Akaroa control.

\begin{inifile}
<modulename>.<vectorname1>.with-akaroa = true
<modulename>.<vectorname2>.with-akaroa = true
\end{inifile}


\subsubsection{Using Shared File Systems}
\label{sec:run-sim:akaroa-using-shared-filesystems}

It is usually practical to have the same physical disk mounted (e.g. via NFS or
Samba) on all computers in the cluster. However, because all {\opp} simulation
processes run with the same settings, they would overwrite each other's
output files. Your can prevent this from happening using the
\fconfig{fname-append-host} ini file entry:

\begin{inifile}
[General]
fname-append-host = true
\end{inifile}

When turned on, it appends the host name to the names of the output
files (output vector, output scalar, snapshot files).



%%% Local Variables:
%%% mode: latex
%%% TeX-master: "usman"
%%% End:
