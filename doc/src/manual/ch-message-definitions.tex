\chapter{Message Definitions}
\label{cha:msg-def}


\section{Introduction}
\label{sec:msg-defs:intro}

In practice, one needs to add various fields to \cclass{cMessage} or
\cclass{cPacket} to make them useful. For example, when modeling
communication networks, message/packet objects need to carry protocol
header fields. Since the simulation library is written in C++, the natural
way of extending \cclass{cMessage}/\cclass{cPacket} is via subclassing
them. However, at least three items has to be added to the new class for
each field (a private data member, a getter and a setter method) and the
resulting class needs to integrate with the simulation framework, which
means that writing the necessary C++ code can be a tedious and
time-consuming task.

{\opp} offers a more convenient way called \textit{message definitions}.
Message definitions offer a compact syntax to describe message contents,
and the corresponding C++ code is automatically generated from the
definitions. When needed, the generated class can also be customized via
subclassing. Even when the generated class needs to be heavily customized,
message definitions can still save the programmer a great deal of manual
work.


\subsection{The First Message Class}
\label{sec:msg-defs:first-msg-class}

Let us begin with a simple example. Suppose that we need a packet type that
carries a source and a destination address as well as a hop count. The
corresponding C++ code can be generated from the following definition in a
\ttt{MyPacket.msg} file:

\begin{msg}
packet MyPacket
{
     int srcAddress;
     int destAddress;
     int remainingHops = 32;
};
\end{msg}

It is the task of the {\opp} \textit{message compiler}, \fprog{opp\_msgc} or
\fprog{opp\_msgtool}, to translate the definition into a C++ class that can be
instantiated from C++ model code. The message compiler is normally invoked for
\ttt{.msg} files automatically, as part of the build process.

When the message compiler processes \ttt{MyPacket.msg}, it creates two files:
\ttt{MyPacket\_m.h} and \ttt{MyPacket\_m.cc}. The generated \ttt{MyPacket\_m.h}
will contain the following class declaration (abbreviated):

\begin{cpp}
class MyPacket : public cPacket {
  protected:
    int srcAddress;
    int destAddress;
    int remainingHops = 32;
  public:
    MyPacket(const char *name=nullptr, short kind=0);
    MyPacket(const MyPacket& other);
    MyPacket& operator=(const MyPacket& other);
    virtual MyPacket *dup() const override {return new MyPacket(*this);}
    ...

    // field getter/setter methods
    virtual int getSrcAddress() const;
    virtual void setSrcAddress(int srcAddress);
    virtual int getDestAddress() const;
    virtual void setDestAddress(int destAddress);
    virtual int getRemainingHops() const;
    virtual void setRemainingHops(int remainingHops);
};
\end{cpp}

As you can see, for each field the generated class contains a protected data
member, and a public getter and a setter method. The names of the methods will
begin with \ttt{get} and \ttt{set}, followed by the field name with its first
letter converted to uppercase.

The \ttt{MyPacket\_m.cc} file contains implementation of the generated
\ttt{MyPacket} class as well as ``reflection'' code (see
\cclass{cClassDescriptor}) that allows inspection of these data structures under
graphical user interfaces like Qtenv. The \ttt{MyPacket\_m.cc} file should be
compiled and linked into the simulation; this is normally taken care of
automatically.

In order to use the \ttt{MyPacket} class from a C++ source file, the
generated header file needs to be included:

\begin{cpp}
#include "MyPacket_m.h"

...
MyPacket *pkt = new MyPacket("pkt");
pkt->setSrcAddress(localAddr);
...
\end{cpp}


\subsection{Ingredients of Message Files}
\label{sec:msg-defs:ingredients}

Message files contain the following ingredients:

\begin{itemize}
  \item \textit{Packet, message, and class definitions} translate into C++ class
    definitions. The three types are very similar, they practically only differ
    in the choice of the default base class (\cclass{cPacket},
    \cclass{cMessage}, and no base class, respectively).
  \item \textit{Struct definitions} translate into C-like structs, where fields
    are represented with public data members (there are no getters and setters).
  \item \textit{Enum definitions} translate into C++ enums.
  \item \textit{Namespace declarations} define the namespace for subsequent definitions.
  \item \textit{Imports} allow reusing definitions from other \ttt{.msg} files.
  \item \textit{Properties} are metadata annotations of the syntax \ttt{@name}
    or \ttt{@name(...)} that may occur on file, class (packet, struct, etc.)
    definition, and field level as well. There are many predefined properties,
    and a large subset of them deal with the details of what C++ code to
    generate for the item they occur with. For example, \ttt{@getter(getFoo)}
    on a field requests that the generated getter function have the name \ttt{getFoo}.
  \item \textit{C++ blocks} are used for injecting literal C++ code fragments
    into the generated source files. The target (the place where to insert the
    code) can be specified.
\end{itemize}

The following sections describe all of the above elements in detail.


\section{Classes, Messages, Packets, Structs}
\label{sec:msg-defs:classes-messages-packets-structs}

As shown above, the message description language allows you to generate C++ data
classes and structs from concise descriptions that have a syntax resembling C
structs. The descriptions contain the choice of the base class (message
descriptions only support single inheritance), the list of fields the class
should have, and possibly various metadata annotations that e.g. control the
details of the code generation.

A description starts with one of the \fkeyword{packet}, \fkeyword{message},
\fkeyword{class}, \fkeyword{struct} keywords. The first three are very
similar: they all generate C++ classes, and only differ on the choice of the
default base class (and related details such as the argument list of the
constructor). The fourth one generates a plain (C-style) struct.


\subsection{Classes, Messages, Packets}
\label{sec:msg-defs:classes-messages-packets}
\label{sec:msg-defs:defining-classes} % legacy
\label{sec:msg-defs:messages-and-packets} % legacy
\label{sec:msg-defs:defining-messages-and-packets} % legacy

For \fkeyword{packet}, the default base class is \cclass{cPacket}; or if a base
class is explicitly named, it must be a subclass of \cclass{cPacket}. Similarly,
for \fkeyword{message}, the default base class is \cclass{cMessage}, or if a
base class is specified, it must be a subclass of \cclass{cMessage}.

For \fkeyword{class}, the default is \textit{no} base class. However, it is often a
good idea to choose \cclass{cObject} as a base class.\footnote{Until \omnetpp
6.0, the default base class was \cclass{cObject}. Thus, when migrating code from
version 5.x or earlier, one needs to add \ttt{extends cObject} to class
definitions lacking an "extends" clause.}

\begin{note}
  It is recommended to use \cclass{cObject} as base class, because it adds zero
  overhead to the generated class, and at the same time makes the class more
  interoperable with the rest of the simulation library. \cclass{cObject}
  only defines virtual methods but no data members, so the only overhead would
  be the \textit{vptr}; however, the generated class already has a vptr because
  the generated methods are also virtual.
\end{note}

The base class is specified with the \fkeyword{extends} keyword. For example:

\begin{msg}
packet FooPacket extends PacketBase
{
    ...
};
\end{msg}

The generated C++ class will look like this:

\begin{cpp}
class FooPacket : public PacketBase {
    ...
};
\end{cpp}

The generated class will have a constructor and also a copy constructor. An
assignment operator (\ffunc{operator=()}) and cloning method (\ffunc{dup()})
will also be generated.

The argument list of the generated constructor depends on the base class. For
classes derived from \cclass{cMessage}, it will accept an object name and
message kind. For classes derived from \cclass{cNamedObject}, it will accept an
object name. The arguments are optional (they have default values).

\begin{cpp}
class FooPacket : public PacketBase
{
  public:
    FooPacket(const char *name=nullptr, int kind=0);
    FooPacket(const FooPacket& other);
    FooPacket& operator=(const FooPacket& other);
    virtual FooPacket *dup() const;
    ...
\end{cpp}

Additional base classes can be added by listing them in the \fprop{@implements}
class property.


\subsection{Structs}
\label{sec:msg-defs:defining-structs}

Message definitions allow one to define C-style structs, ``C-style''
meaning ``containing only data and no methods''. These structs can be
useful as fields in message classes.

The syntax is similar to that of defining messages:

\begin{msg}
struct Place
{
    int type;
    string description;
    double coords[3];
};
\end{msg}

The generated struct has public data members, and no getter or setter
methods. The following code is generated from the above definition:

\begin{cpp}
// generated C++
struct Place
{
    int type;
    omnetpp::opp_string description;
    double coords[3];
};
\end{cpp}

Note that \fkeyword{string} fields are generated with the \ttt{opp\_string} C++
type, which is a minimalistic string class that wraps \ttt{const char*} and
takes care of allocation/deallocation. It was chosen instead of
\ttt{std::string} because of its significantly smaller memory footprint.
(\ttt{std::string} is significantly larger than a \ttt{const char*} pointer
because it also needs to store length and capacity information in some form.)

Inheritance is supported for structs:

\begin{msg}
struct Base
{
    ...
};

struct Extended extends Base
{
    ...
};
\end{msg}

However, because a struct has no member functions, there are limitations:

\begin{itemize}
   \item variable-size arrays are not supported;
   \item customization via inheritance and \fkeyword{abstract} fields
      (see later in \ref{sec:msg-defs:customizing-via-inheritance})
      cannot be used;
   \item cannot have classes subclassed from \cclass{cOwnedObject} as fields,
      because structs cannot be owners.
\end{itemize}


\section{Enums}
\label{sec:msg-defs:enums}

An enum is declared with the \fkeyword{enum} keyword, using the following
syntax:

\begin{msg}
enum PayloadType
{
   NONE = 0;
   VOICE = 1;
   VIDEO = 2;
   DATA = 3;
};
\end{msg}

Enum values need to be unique.

The message compiler translates an enum into a normal C++ enum, plus also
generates a descriptor that stores the symbolic names as strings. The latter
makes it possible for Qtenv to display symbolic names for enum values.

Enums can be used in two ways. The first is simply to use the enum's name as
field type:

\begin{msg}
packet FooPacket
{
    PayloadType payloadType;
};
\end{msg}

The second way is to tag a field of the type \fkeyword{int} or any
other integral type with the \fprop{@enum} property and the name of
the enum, like so:

\begin{msg}
packet FooPacket
{
    int16_t payloadType @enum(PayloadType);
};
\end{msg}

In the generated C++ code, the field will have the original type (in this case,
\fkeyword{int16\_t}). However, additional code generated by the message compiler
will allow Qtenv to display the symbolic name of the field's value in addition
to the numeric value.


\section{Imports}
\label{sec:msg-defs:imports}

Import directives are used to make definitions in one message file available to another
one. Importing an MSG file makes the definitions in that file available to the
file that imports it, but has no further side effect (and in particular, it will
generate no C++ code).

To import a message file, use the \fkeyword{import} keyword followed by a name
that identifies the message file within its project:

\begin{msg}
import inet.linklayer.common.MacAddress;
\end{msg}

The \fkeyword{import}'s parameter is interpreted as a relative file path (by
replacing dots with slashes, and appending \ttt{.msg}), which is searched for in
folders listed in the \textit{message import path}, much like C/C++ include
files are searched for in the compiler's include path, Python modules in the
Python module search path, or NED files in the NED path.

The message import path can be be specified to the message compiler via a series
of \fopt{-I} command-line options.


\section{Namespaces}
\label{sec:msg-defs:namespaces}
\label{sec:msg-defs:declaring-a-namespace} % legacy

To place generated types into a namespace, add a \fkeyword{namespace} directive
above the types in question:

\begin{msg}
namespace inet;
\end{msg}

Hierarchical (nested) namespaces are declared using double colons in the
namespace definition, much like nested namespace definitions introduced into
C++ in version C++17.

\begin{msg}
namespace inet::ieee80211;
\end{msg}

The above code will be translated into multiple nested namespaces in the C++ code:

\begin{cpp}
namespace inet { namespace ieee80211 {
...
}}
\end{cpp}

There can be multiple \fkeyword{namespace} directives in a message file. The
effect of the \fkeyword{namespace} directive extends from the place of the
directive until the next \fkeyword{namespace} directive or the end of the
message file. Each \fkeyword{namespace} directive opens a completely new
namespace, i.e. \textit{not} a namespace within the previous one. An empty
namespace directive (\ttt{namespace;}) returns to the global namespace. For
example:

\begin{msg}
namespace foo::bar;
class A {}  // defines foo::bar::A

namespace baz;
class B {}  // defines baz::B

namespace;
class C {}  // defines ::C
\end{msg}


\section{Properties}
\label{sec:msg-defs:properties}

Properties are metadata annotations of the syntax \ttt{@name} or
\ttt{@name(...)} that may occur on file, class (packet, struct, etc.)
definition, and field level. There are many predefined properties, and a
large subset of them deal with the details of what C++ code to generate for the
item they occur with. For example, \ttt{@getter(getFoo)} on a field requests
that the generated getter function have the name \ttt{getFoo}.

Here is a syntax example. Note that class properties are placed in the fields
list (fields and properties may be mixed in arbitrary order), and field
properties are written after the field name.

\begin{msg}
@foo;
class Foo {
  @customize(true);
  string value @getter(...) @setter(...) @hint("...");
}
\end{msg}

Syntactically, the mandatory part of a property is the \ttt{@} character
followed by the property name. They are then optionally followed by an
\textit{index} and a \textit{parameter list}. The index is a name in square
brackets, and it is rarely used. The parameter list is enclosed in parentheses,
and in theory it may contain a value list and key-valuelist pairs, but
almost all properties expect to find just a single value there.

For boolean properties, the value may be \fkeyword{true} or \fkeyword{false}; if
the value is missing, \fkeyword{true} is assumed. Thus, \ttt{@customize} is
equivalent to \ttt{@customize(true)}.

As a guard against mistyping property names, properties need to be declared
before they can be used. Properties are declared using the \fprop{@property}
property, with the name of the new property in the index, and the type and other
attributes of the property in the parameter list. Examples for property
declarations, including the declaration of \fprop{@property} itself, can be seed
by listing the built-in definitions of the message compiler
(\ttt{opp\_msgtool -h builtindefs}).

The full list of properties understood by the message compiler and other {\opp}
tools can be found in Appendix \ref{cha:msg-properties}.


\subsection{Data Types}
\label{sec:msg-defs:field-data-types}

The following data types can be used for fields:

\begin{itemize}
  \item C/C++ primitive data types: \fkeyword{bool}, \fkeyword{char},
    \fkeyword{short}, \fkeyword{int}, \fkeyword{long}, \fkeyword{unsigned char},
    \fkeyword{unsigned short}, \fkeyword{unsigned int}, \fkeyword{unsigned
    long}, \fkeyword{float}, \fkeyword{double}.
  \item \fkeyword{string}. Getters and setters use the \fkeyword{const char*} data
    type; \fkeyword{nullptr} is not allowed. Setters store a copy of the string, not
    just the pointer.
  \item C99-style fixed-size integer types: \fkeyword{int8\_t},
    \fkeyword{int16\_t}, \fkeyword{int32\_t}, \fkeyword{int64\_t},
    \fkeyword{uint8\_t}, \fkeyword{uint16\_t}, \fkeyword{uint32\_t},
    \fkeyword{uint64\_t}.\footnote{These type names are accepted without the
    \ttt{\_t} suffix as well, but you are responsible to ensure that the
    generated code compiles, i.e. the shortened type names must be defined in a
    header file you include.}
\end{itemize}

In addition, {\opp} class names such as \cclass{simtime\_t} and \cclass{cMessage} are
also made available without the need to import anything. These names are
accepted both with and without spelling out the \ttt{omnetpp} namespace
name.

Numeric fields are initialized to zero, booleans to \fkeyword{false}, and string
fields to the empty string.


\section{Fields}
\label{sec:msg-defs:fields}

\subsection{Scalar fields}
\label{sec:msg-defs:scalar-fields}

A scalar field is one that holds a single value. It is defined by specifying the
data type and the field name, for example:

\begin{msg}
int timeToLive;
\end{msg}

For each field, the generated class will have a protected data member, and a
public getter and setter method. The names of the methods will begin with
\ttt{get} and \ttt{set}, followed by the field name with its first letter
converted to uppercase. Thus, the above field will generate the following
methods in the C++ class:

\begin{cpp}
int getTimeToLive() const;
void setTimeToLive(int timeToLive);
\end{cpp}

\begin{note}
All methods are generated to be virtual, but we omit the \fkeyword{virtual}
keyword here and in further examples.
\end{note}

The method names are derived from the field name, but they can be customized
with the \fprop{@getter} and \fprop{@setter} properties, as shown below:

\begin{msg}
int timeToLive @getter(getTTL) @setter(setTTL);
\end{msg}

The choice of C++ type used for the data member and the getter/setter methods
can be overridden with the help of the \fprop{@cppType} property (and on a more
fine-grained level, with \fprop{@datamemberType}, \fprop{@argType} and
\fprop{@returnType}), although this it is rarely useful.


\subsection{Initial Values}
\label{sec:msg-defs:initial-values}

Initial values for fields can be specified after an equal sign, like so:

\begin{msg}
int version = HTTP_VERSION;
string method = "GET";
string resource = "/";
bool keepAlive = true;
int timeout = 5*60;
\end{msg}

Any phrase that is a valid C++ expression can be used as initializer value. (The
message compiler does not check the syntax of the values, it merely copies them
into the generated C++ file.)

For array fields, the initializer specifies the value for individual array
elements. There is no syntax for initializing an array with a list of values.


\subsection{Overriding Initial Values from Subclasses}
\label{sec:msg-defs:assignment-of-inherited-fields}

In a subclass, it is possible to override the initial value of an inherited
field. The syntax is similar to that of a field definition with initial value,
only the data type is missing.

An example:

\begin{msg}
packet Ieee80211Frame
{
    int frameType;
    ...
};

packet Ieee80211DataFrame extends Ieee80211Frame
{
    frameType = DATA_FRAME;  // assignment of inherited field
    ...
};
\end{msg}

It may seem like the message compiler would need the definition of the base
class to check the definition of the field being assigned. However, it is
not the case. The message compiler trusts that such field exists; or
rather, it leaves the check to the C++ compiler.

What the message compiler actually does is derives a setter method name
from the field name, and generates a call to it into the constructor.
Thus, the generated constructor for the above packet type would be
something like this:

\begin{cpp}
Ieee80211DataFrame::Ieee80211DataFrame(const char *name, int kind) :
    Ieee80211Frame(name, kind)
{
    this->setFrameType(DATA_FRAME);
    ...
}
\end{cpp}

This implementation also lets one initialize \cclass{cMessage} /
\cclass{cPacket} fields such as message kind or packet length:

\begin{msg}
packet UDPPacket
{
    byteLength = 16;  // results in 'setByteLength(16);' being placed into ctor
};
\end{msg}


\subsection{Const Fields}
\label{sec:msg-defs:const-fields}

A field can be marked as const by the using \fkeyword{const} keyword. A
const field only has a (const) data member and a getter function, but no
setter. The value can be provided via an initializer. An example:

\begin{msg}
const int foo = 24;
\end{msg}

This generates a \ttt{const int} data member in the class, initialized to 24,
and a getter member function that returns its value:

\begin{cpp}
int getFoo() const;
\end{cpp}

Array fields cannot be const.

Note that a pointer field may also be marked const, but \fkeyword{const} is
interpreted differently in that case: as a mutable field that holds a pointer to
a \fkeyword{const} object.

One use of \fkeyword{const} is to implement computed fields. For that, the field
needs to be annotated with the \fprop{@custom} or \fprop{@customImpl} property
to allow for a custom implementation to be supplied for the getter. The custom
getter can then encapsulate the computation of the field value. Customization is
covered in section \ref{sec:msg-defs:customizing-generated-class}.

\begin{note}
To add actual constants (as opposed to getter-only fields) to a class, it
is better use a targeted \fkeyword{cplusplus} block to inject their definitions
into the C++ class declaration.
\end{note}


\subsection{Abstract Fields}
\label{sec:msg-defs:abstract-fields}

Abstract fields is a way to allow a custom implementation (such as storage,
getter/setter methods, etc.) to be provided for a field. For a field marked as
abstract, the message compiler does not generate a data member, and generated
getter/setter methods will be pure virtual. It is expected that the pure virtual
methods will be implemented in a subclass (possibly via \fprop{@customize}, see
section \ref{sec:msg-defs:customizing-generated-class}).

A field is declared abstract by using the \fkeyword{abstract} keyword or the
\fprop{@abstract} property (the two are equivalent).

\begin{msg}
abstract bool urgentBit; // or: bool urgentBit @abstract;
\end{msg}

The generated pure virtual methods:

\begin{cpp}
virtual bool getUrgentBit() const = 0;
virtual void setUrgentBit(bool urgentBit) = 0;
\end{cpp}

Alternatives to \fkeyword{abstract}, at least for certain use cases, are
\fprop{@custom} and \fprop{@customImpl} (see section
\ref{sec:msg-defs:customizing-generated-class}).


\subsection{Fixed-Size Arrays}
\label{sec:msg-defs:fixed-size-arrays}

Fixed-size arrays can be declared with the usual syntax of putting the
array size in square brackets after the field name:

\begin{msg}
int route[4];
\end{msg}

The generated getter and setter methods will have an extra \ttt{k} argument (the
array index), and a third method that returns the array size is also generated:

\begin{cpp}
int getRoute(size_t k) const;
void setRoute(size_t k, int route);
size_t getRouteArraySize() const;
\end{cpp}

When the getter or setter method is called with an index that is out of bounds,
an exception is thrown.

The method names can be overridden with the \fprop{@getter}, \fprop{@setter} and
\fprop{@sizeGetter} properties. To use another C++ type for array size and
indices instead of the default \fprop{size\_t}, specify the \fprop{@sizeType}
property.

\begin{note}
  Use a singular noun for field name instead of plural (\ttt{route[]} instead of
  \ttt{routes[]}), otherwise method names will look confusing (\ttt{getRoutes()},
  \ttt{appendRoutes()}, etc, for methods that deal with a single route).
\end{note}

When a default value is given, it is interpreted as a scalar for filling the
array with. There is no syntax for initializing an array with a list of values.

\begin{msg}
int route[4] = -1; // all elements set to -1
\end{msg}


\subsection{Variable-Size Arrays}
\label{sec:msg-defs:variable-size-arrays}

If the array size is not known in advance, the field can be declared
to have a variable size by using an empty pair in brackets:

\begin{msg}
int route[];
\end{msg}

In this case, the generated class will have extra methods in addition to the
getter and setter: one for resizing the array, one for getting the array size,
plus methods for inserting an element at a given position, appending an element,
and erasing an element at a given position.

\begin{cpp}
int getRoute(size_t k) const;
void setRoute(size_t k, int route);
void setRouteArraySize(size_t size);
size_t getRouteArraySize() const;
void insertRoute(size_t k, int route);
void appendRoute(int route);
void eraseRoute(size_t k);
\end{cpp}

The default array size is zero. Elements can be added by calling the inserter
or the appender method, or resizing the array and setting individual elements.

Internally, all methods that change the array size (inserter, appender, resizer)
always allocate a new array, and copy existing values over to the new array.
Therefore, when adding a large number elements, it is recommended to resize the
array first, instead of calling the appender method multiple times.

The method names can be overridden with the \fprop{@getter}, \fprop{@setter},
\fprop{@sizeGetter}, \fprop{@sizeSetter}, \fprop{@inserter}, \fprop{@appender}
and \fprop{@eraser} field properties. To use another C++ type for array size and
indices instead of the default \fprop{size\_t}, specify the \fprop{@sizeType}
property.

When a default value is given, it is used for initializing new elements
when the array is expanded.

\begin{msg}
int route[] = -1;
\end{msg}


\subsection{Classes and Structs as Fields}
\label{sec:msg-defs:other-types-as-fields}

Classes and structs may also be used as as fields, not only primitive types and
\ttt{string}. For example, given a class named \ttt{IPAddress}, one can write
the following field:

\begin{msg}
IPAddress sourceAddress;
\end{msg}

The \ttt{IPAddress} type must be known to the message compiler.

The generated class will contain an \ttt{IPAddress} data member, and the
following member functions:

\begin{cpp}
const IPAddress& getSourceAddress() const;
void setSourceAddress(const IPAddress& sourceAddress);
IPAddress& getSourceAddressForUpdate();
\end{cpp}

Note that in addition to the getter and setter, a mutable getter
(\ttt{get...ForUpdate}) is also generated, which allows the stored value (object
or struct) to be modified in place.

By default, values are passed by reference. This can be changed by specifying
the \fprop{@byValue} property:

\begin{msg}
IPAddress sourceAddress @byValue;
\end{msg}

This generates the following member functions:

\begin{cpp}
virtual IPAddress getSourceAddress() const;
virtual void setSourceAddress(IPAddress sourceAddress);
\end{cpp}

Note that both member functions use pass-by-value, and that the mutable getter
function is not generated.

Specifying \fkeyword{const} will cause only a getter function to be generated
but no setter or mutable getter, as shown before in \ref{sec:msg-defs:const-fields}.

Array fields are treated similarly, the difference being that the getter and setter
methods take an extra index argument:

\begin{msg}
IPAddress route[];
\end{msg}

The generated methods:

\begin{cpp}
void setRouteArraySize(size_t size);
size_t getRouteArraySize() const;
const IPAddress& getRoute(size_t k) const;
IPAddress& getRouteForUpdate(size_t k);
void setRoute(size_t k, const IPAddress& route);
void insertRoute(size_t k, const IPAddress& route);
void appendRoute(const IPAddress& route);
void eraseRoute(size_t k);
\end{cpp}


\subsection{Non-Owning Pointer Fields}
\label{sec:msg-defs:pointers-fields}

The field type may be a pointer, both for scalar and array fields. Pointer
fields come in two flavours: owning and non-owning. A non-owning pointer field
just stores the pointer value regardless of the ownership of the object it
points to, while an owning pointer holds the ownership of the object. This
section discusses non-owning pointer fields.

Example:

\begin{msg}
cModule *contextModule;  // missing @owner: non-owning pointer field
\end{msg}

The generated methods:

\begin{cpp}
const cModule *getContextModule() const;
void setContextModule(cModule *contextModule);
cModule *getContextModuleForUpdate();
\end{cpp}

If the field is marked \fkeyword{const}, then the setter will take a
\fkeyword{const} pointer, and the \ttt{getForUpdate()} method is not generated:

\begin{msg}
const cModule *contextModule;
\end{msg}

The output:

\begin{cpp}
const cModule *getContextModule() const;
void setContextModule(const cModule *contextModule);
\end{cpp}


\subsection{Owning Pointer Fields}
\label{sec:msg-defs:pointers-fields}

This section discusses pointer fields that own the objects they point to, that
is, are responsible for deallocating the object when the object containing the
field (let's refer to it as \textit{container} object) is deleted.

For all owning pointer fields in a class, the destructor of the class deletes the
owned objects, the \ffunc{dup()} method and the copy constructor duplicate the
owned objects for the newly created object, and the assignment operator
(\ffunc{operator=}) does both: the old objects in the destination object are
deleted, and replaced by clones of the objects in the source object.

When the owned object is a subclass of \cclass{cOwnedObject} that keeps track of
its owner, the code generated for the container class invokes the \ffunc{take()}
and \ffunc{drop()} methods at the appropriate times to manage the ownership.

Example:

\begin{msg}
cPacket *payload @owned;
\end{msg}

The generated methods:

\begin{cpp}
const cPacket *getPayload() const;
cPacket *getPayloadForUpdate();
void setPayload(cPacket *payload);
cPacket *removePayload();
\end{cpp}

The getter and mutable getter return the stored pointer (or \fkeyword{nullptr} if
there is none).

The remover method releases the ownership of the stored object, sets the field
to \fkeyword{nullptr}, and returns the object.

The setter method behavior depends on the presence of the \fprop{@allowReplace}
property. By default (when \fprop{@allowReplace} is absent), the setter does not
allow replacing the object. That is, when the setter is invoked on a field that
already contains an object (the pointer is non-null), an error is raised:
\textit{"A value is already set, remove it first with removePayload()"}.
One must call \fprop{removePayload()} before setting a new object.

When \fprop{@allowReplace} is specified for the field, there is no need to call
te remover method before setting a new value, because the setter method deletes
the old object before storing the new one.

\begin{msg}
cPacket *payload @owned @allowReplace; // allow setter to delete the old object
\end{msg}

If the field is marked \fkeyword{const}, then the \ttt{getForUpdate()} method
is not generated, and the setter takes a \fkeyword{const} pointer.

\begin{msg}
const cPacket *payload @owned;
\end{msg}

The generated methods:

\begin{cpp}
const cPacket *getPayload() const;
void setPayload(const cPacket *payload);
cPacket *removePayload();
\end{cpp}
%TODO removePayload() should return const pointer here

The name of the remover method (which is the only extra method compared to
non-pointer fields) can be customized using the \fprop{@remover} property.


\section{Literal C++ Blocks}
\label{sec:msg-defs:cplusplus-blocks}

It is possible to have C++ code fragments injected directly into the generated
code. This is done with the \fkeyword{cplusplus} keyword optionally followed by
a \textit{target} in parentheses, and the code fragment enclosed in double curly
braces.

The target specifies where to insert the code fragment in the generated header
or implementation file; we'll get to it in a minute.

As far as a the code fragment is concerned, the message compiler does not try to
make sense of it, just simply copies it into the generated source file at the
requested location. The code fragment should be formatted so that it does not
contain a double close curly brace (\ttt{\}\}}) because it would be interpreted as
end of the fragment block.\footnote{Should this ever be a problem, just insert a space
between the two braces, or use the automatic concatenation of adjacent string literals
feature of C/C++ if they occur within a string constant, i.e. break up \ttt{"foo\}\}bar"}
into \ttt{"foo\}" "\}bar"}.}

\begin{msg}
cplusplus {{
#include "FooDefs.h"
#define SOME_CONSTANT 63
}}
\end{msg}

The target can be \ttt{h} (the generated header file -- this is the default),
\ttt{cc} (the generated \ttt{.cc} file), the name of a type generated in the
same message file (content is inserted in the declaration of the type, just
before the closing curly brace), or a member function name of one such type.

\fkeyword{cplusplus} blocks with the target \ttt{h} are customarily used to
insert \ttt{\#include} directives, commonly used constants or macros (e.g.
\ttt{\#define}s), or, rarely, typedefs and other elements into the generated
header. The fragments are pasted into the namespace which is open at that point.
Note that includes should always be placed into a \fkeyword{cplusplus(h)} block
\tbf{above} the first namespace declaration in the message file.

\fkeyword{cplusplus} blocks with \ttt{cc} as target allow you to insert code
into the \ttt{.cc} file, e.g. implementations of member functions. This is
useful e.g with custom-implementation fields (\fprop{@customImpl}, see
\ref{sec:msg-defs:customimpl-fields}).

\fkeyword{cplusplus} blocks with a type name as target allow you to insert new
data members and member functions into the class. This is useful e.g with custom
fields (\fprop{@custom}, see \ref{sec:msg-defs:custom-fields}).

To inject code into the implementation of a member function of a generated
class, specify \textit{<classname>}\ttt{::}\textit{<methodname>} as target.
Supported methods include the constructor, copy constructor (use \ttt{Foo\&} as
name), destructor, \ttt{operator=}, \ttt{copy()}, \ttt{parsimPack()},
\ttt{parsimUnpack()}, etc., and the per-field generated methods (setter, getter,
etc.).



\section{Using External C++ Types}
\label{sec:msg-defs:using-cpp-types}
\label{sec:msg-defs:announcing-types} %% legacy

The message compiler only allows types it knows about to be used for fields or
base classes. If you want to use to types not generated by the message compiler,
you need to do the following:

\begin{enumerate}
   \item Let the message compiler know about the type; and
   \item Make sure its C++ declaration is available at compile time
\end{enumerate}

For the first one can be achieved with the \fprop{@existingClass} property. When
a type (class or struct) is annotated with \fprop{@existingClass}, the message
compiler remembers the definition, but assumes that the class (or struct)
already exist in C++ code, and does not generate it. (However, it will still
generate a class descriptor, see section \ref{sec:msg-defs:descriptor-classes}.)

\begin{note}
Support for C++-style type announcements is no longer part of the message
definitions syntax, they were removed in {\opp} version 6.0.
\end{note}

The second task is achieved by adding a \fkeyword{cplusplus} block with
an \ttt{\#include} directive to the message file.

For example, suppose we have a hand-written \ttt{ieee802::MACAddress} class
defined in \ttt{MACAddress.h} that we would like to use for fields in multiple
message files. One way to make this possible is to add a \ttt{MACAddress.msg}
file alongside the header with the following content:

\begin{msg}
// MACAddress.msg

cplusplus {{
#include "MACAddress.h"
}}

class ieee802::MACAddress  // a separate namespace decl would also do
{
    @existingClass;
    int8_t octet[6]; // assumes class has getOctet(k) and setOctet(k)
}
\end{msg}

As exemplified above, for existing classes it is possible to announce
them with their namespace-qualified name, there is no need for separate
\fkeyword{namespace} line.

This message file can be imported into all other message files that need
\ttt{MACAddress}, for example like this:

\begin{msg}
import MACAddress;

packet EthernetFrame {
    ieee802::MACAddress source;
    ieee802::MACAddress destination;
    ...
}
\end{msg}


\section{Customizing the Generated Class}
\label{sec:msg-defs:customizing-generated-class}

There are several possibilities for customizing a generated class:

\begin{itemize}
  \item Using custom method names and custom field types
  \item Using custom field types
  \item Injecting code into existing member functions
  \item Custom fields
  \item Fields with custom-implementation methods
  \item The Generation Gap pattern
  \item Abstract fields
  \item Special customizations, e.g. \fprop{@str}, \fprop{@nopack}, etc.
\end{itemize}

The following sections explore the above possibilities.


\subsection{Customizing Method Names}
\label{sec:msg-defs:customizing-method-names}

The names and some other properties of generated methods can be influenced
with metadata annotations (properties).

The following field properties exist for overriding method names:
\fprop{@getter}, \fprop{@setter}, \fprop{@getterForUpdate}, \fprop{@remover},
\fprop{@sizeGetter}, \fprop{@sizeSetter}, \fprop{@inserter}, \fprop{@appender}
and \fprop{@eraser}.

To override data types used by the data member and its accessor methods, use
\fprop{@cppType}, \fprop{@datamemberType}, \fprop{@argType}, or
\fprop{@returnType}.

To override the default \ttt{size\_t} type used for array size and indices, use
\fprop{@sizeType}.

Consider the following example:

\begin{msg}
packet IPPacket {
    int ttl @getter(getTTL) @setter(setTTL);
    Option options[] @sizeGetter(getNumOptions)
                     @sizeSetter(setNumOptions)
                     @sizetype(short);
}
\end{msg}

The generated class would have the following methods (note the differences
from the default names \ttt{getTtl()}, \ttt{setTtl()}, \ttt{getOptions()},
\ttt{setOptions()}, \ttt{getOptionsArraySize()}, \ttt{getOptionsArraySize()};
also note that indices and array sizes are now \fkeyword{short}):

\begin{cpp}
virtual int getTTL() const;
virtual void setTTL(int ttl);
virtual const Option& getOption(short k) const;
virtual void setOption(short k, const Option& option);
virtual short getNumOptions() const;
virtual void setNumOptions(short n);
\end{cpp}

In some older simulation models you may also see the use of the
\fprop{@omitGetVerb} class property. This property tells the message
compiler to generate getter methods without the ``get'' prefix, e.g. for a
\ttt{sourceAddress} field it would generate a \ttt{sourceAddress()} method
instead of the default \ttt{getSourceAddress()}. It is not recommended to
use \ttt{@omitGetVerb} in new models, because it is inconsistent with the
accepted naming convention.

%% TODO \fprop{nopack}  (also a field property)


\subsection{Injecting Code into Methods}
\label{sec:msg-defs:injecting-code-into-methods}

Generally, literal C++ blocks (the \fkeyword{cplusplus} keyword) are the way to
inject code into the body of individual methods, as described in
\ref{sec:msg-defs:cplusplus-blocks}.

The \fprop{@beforeChange} class property can be used to designate a member function
which is to be called before any mutator code (in setters, non-const getters,
assignment operator, etc.) executes. This can be used to implement e.g. a dirty
flag or some form of immutability (i.e. freeze the state of the object).


\subsection{Generating str()}
\label{sec:msg-defs:generating-str-method}

The \fprop{@str} class property aims at simplifying adding an \ffunc{str()}
method in the generated class. Having an \ffunc{str()} method is often useful
for debugging, and it also has a special role in class descriptors (see
\ref{sec:msg-defs:descriptor-tostring}).

When \fprop{@str} is present, an \ttt{std::string str() const} method is
generated for the class. The method's implementation will contain a single
\ttt{return} keyword, with the value of the \fprop{@str} property copied
after it.

Example:

\begin{msg}
class Location {
    double lat;
    double lon;
    @str("(" + std::to_string(getLat()) + "," + std::to_string(getLon()) + ")");
}
\end{msg}

It will result in the following \ttt{str()} method to be generated as part of
the \ttt{Location} class:

\begin{cpp}
std::string Location::str() const
{
    return "(" + std::to_string(getLat()) + "," + std::to_string(getLon()) + ")";
}
\end{cpp}


\subsection{Custom-implementation Methods}
\label{sec:msg-defs:customimpl-fields}

When member functions generated for a field need customized implementation and
method-targeted C++ blocks are not sufficient, the \fprop{customImpl} property
can be of help. When a field is marked \fprop{customImpl}, the message compiler
will skip generating the implementations of its accessor methods in the \ttt{.cc} file,
allowing the user to supply their own versions.

Here is a simple example. The methods in it do not perform anything extra
compared to the default generated versions, but they illustrate the principle.

\begin{msg}
class Packet
{
    int hopCount @customImpl;
}

cplusplus(cc) {{
int Packet::getHopCount() const
{
    return hopCount; // replace/extend with extra code
}

void Packet::setHopCount(int value)
{
    hopCount = value; // replace/extend with extra code
}
}}
\end{msg}


\subsection{Custom Fields}
\label{sec:msg-defs:custom-fields}

If a field is marked with \fprop{@custom}, the field will only appear in the
class descriptor, but no code is generated for it at all. One can inject the
code that implements the field (data member, getter, setter, etc.) via targeted
\fkeyword{cplusplus} blocks (\ref{sec:msg-defs:cplusplus-blocks}).
\fprop{@custom} is a good way to go when you want the field to have a different
underlying storage or different accessor methods than normally generated by the
message compiler. (For the latter case, however, be aware that the generated
class descriptor assumes the presence of certain accessor methods for the
field, although the set of expected methods can be customized to a degree. See
\ref{sec:msg-defs:descriptor-classes} for details.)

The following example uses \fprop{@custom} to implement a field that acts a
stack (has \ttt{push()} and \ttt{pop()} methods), and uses \ttt{std::vector} as
the underlying data structure.

\begin{msg}
cplusplus {{
#include <vector>
}}

class MPLSHeader
{
    int32_t label[] @custom @sizeGetter(getNumLabels) @sizeSetter(setNumLabels);
}

cplusplus(MPLSHeader) {{
  protected:
    std::vector<int32_t> labels;
  public:
    // expected methods:
    virtual void setNumLabels(size_t size) {labels.resize(size);}
    virtual size_t getNumLabels() const {return labels.size();}
    virtual int32_t getLabel(size_t k) const {return labels.at(k);}
    virtual void setLabel(size_t k, int32_t label) {labels.at(k) = label;}
    // new methods:
    virtual void pushLabel(int32_t label) {labels.push_back(label);}
    virtual int32_t popLabel() {auto l=labels.back();labels.pop_back();return l;}
}}

cplusplus(MPLSHeader::copy) {{
    labels = other.labels;
}}
\end{msg}

The last C++ block is needed so that the copy constructor and the
\ttt{operator=} method also copies the new field. (\ttt{copy()} is a member
function where the common part of the above two are factored out, and the C++
block injects code in there.)


\subsection{Customizing the Class via Inheritance}
\label{sec:msg-defs:customizing-via-inheritance}

Another way of customizing the generated code is by employing what is known as
the \textit{Generation Gap} design pattern, proposed by John Vlissides. The idea
is that the customization can be done while \textit{subclassing} the generated
class, overriding whichever member functions need to be different from their
generated versions.

This feature is enabled by adding the \fprop{@customize} property on the class.
Doing so will cause the message compiler to generate an intermediate class
instead of the final one, and the user will subclass the intermediate class to
obtain the real class. The name of the intermediate class is obtained by
appending \ttt{\_Base} to the class name. The subclassing code can be in an
entirely different header and \ttt{.cc} file from the generated one, so this
method does not require the use of \fkeyword{cplusplus} blocks.

Consider the following example:

\begin{msg}
packet FooPacket
{
   @customize(true);
   ...
};
\end{msg}

The message compiler will generate a \ttt{FooPacket\_Base} class instead of
\ttt{FooPacket}. It is then the user's task to subclass \ttt{FooPacket\_Base} to
derive \ttt{FooPacket}, while adding extra data members and adding/overriding
methods to achieve the goals that motivated the customization.

There is a minimum amount of code you have to write for \ttt{FooPacket}, because
not everything can be pre-generated as part of \ttt{FooPacket\_Base} (e.g.
constructors cannot be inherited). This minimum code, which usually goes into a
header file, is the following:

\begin{cpp}
class FooPacket : public FooPacket_Base
{
  private:
    void copy(const FooPacket& other) { ... }
  public:
    FooPacket(const char *s=nullptr, short kind=0) : FooPacket_Base(s,kind) {}
    FooPacket(const FooPacket& other) : FooPacket_Base(other) {copy(other);}
    FooPacket& operator=(const FooPacket& other) {if (this==&other) return *this;
        FooPacket_Base::operator=(other); copy(other); return *this;}
    virtual FooPacket *dup() const override {return new FooPacket(*this);}
};
\end{cpp}

\begin{note}
  The above boilerplate code can be copied out of the generated C++ header, which
  contains it as a comment.
\end{note}

The generated constructor, copy constructor, \ttt{operator=}, \ttt{dup()} can be
usually be copied verbatim. The only method that needs to be custom code is
\ffunc{copy()}. It is shared by the copy constructor and \ttt{operator=}, and
should take care of copying the new data members you added as part of
\ttt{FooPacket}.

In addition to the above, the implementation (\ttt{.cc}) file should contain the
registration of the new class:

\begin{cpp}
Register_Class(FooPacket);
\end{cpp}


\subsection{Using an Abstract Field}
\label{sec:msg-defs:customizing-via-abstact-field}

Abstract fields, introduced in \ref{sec:msg-defs:abstract-fields}, are an
alternative to \fprop{@custom} (see \ref{sec:msg-defs:custom-fields}) for
allowing a custom implementation (such as storage, getter/setter methods, etc.)
to be provided for a field. For a field marked \fkeyword{abstract}, the message compiler does
not generate a data member, and generated getter/setter methods will be pure
virtual.

Abstract fields are most often used together with the Generation Gap pattern
(see \ref{sec:msg-defs:customizing-via-inheritance}), so that one can
immediately supply a custom implementation.

The following example demonstrates the use of abstract fields for creating
an array field that uses \ttt{std::vector} as underlying implementation:

\begin{msg}
packet FooPacket
{
    @customize(true);
    abstract int foo[]; // impl will use std::vector<int>
}
\end{msg}

If you compile the above code, in the generated C++ code you will only find
abstract methods for \ttt{foo}, but no underlying data member or method
implementation. You can implement everything as you like. You can then write the
following C++ file to implement \ttt{foo} with \ttt{std::vector} (some
details omitted for brevity):

\begin{cpp}
#include <vector>
#include "FooPacket_m.h"

class FooPacket : public FooPacket_Base
{
  protected:
    std::vector<int> foo;

  public:
    // constructor and other methods omitted, see below
    ...
    virtual int getFoo(size_t k) {return foo[k];}
    virtual void setFoo(size_t k, int x) {foo[k]=x;}
    virtual void addFoo(int x) {foo.push_back(x);}
    virtual void setFooArraySize(size_t size) {foo.resize(size);}
    virtual size_t getFooArraySize() const {return foo.size();}
};

Register_Class(FooPacket);
\end{cpp}

Some additional boilerplate code is needed so that the class conforms
to conventions, and duplication and copying works properly:

\begin{cpp}
    FooPacket(const char *name=nullptr, int kind=0) : FooPacket_Base(name,kind) {
    }
    FooPacket(const FooPacket& other) : FooPacket_Base(other.getName()) {
        operator=(other);
    }
    FooPacket& operator=(const FooPacket& other) {
        if (&other==this) return *this;
        FooPacket_Base::operator=(other);
        foo = other.foo;
        return *this;
    }
    virtual FooPacket *dup() {
        return new FooPacket(*this);
    }
\end{cpp}


\section{Descriptor Classes}
\label{sec:msg-defs:descriptor-classes}

For each generated class and struct, the message compiler also generates an
associated descriptor class, which class carries ``reflection'' information
about the new class. The descriptor class encapsulates virtually all information
that the original message definition contains, and exposes it via member
functions. Reflection information allows inspecting object contents down to
field level in Qtenv, filtering objects by a filter expression that refers to
object fields, serializing messages-packets in a readable form for the eventlog
file, and has several further potential uses.


\subsection{cClassDescriptor}
\label{sec:msg-defs:cclassdescriptor}

The descriptor class is subclassed from \cclass{cClassDescriptor}. It has
methods for enumerating fields (\ffunc{getFieldCount()}, \ffunc{getFieldName()},
\ffunc{getFieldTypeString()}, etc.), for getting and setting a field's value in
string form (\ffunc{getFieldAsString()}, \ffunc{setFieldAsString()}) and as
\cclass{cValue} (\ffunc{getFieldValue()}, \ffunc{setFieldValue()}), for
exploring the class hierarchy (\ffunc{getBaseClassDescriptor()}, etc.), for
accessing class and field properties, and for similar tasks.

Classes derived from \cclass{cObject} have a virtual member function
\ffunc{getDescriptor} that returns their associated descriptor. For other
classes, it is possible to obtain the descriptor using
\ttt{cClassDescriptor::getDescriptorFor()} with the class name as argument.

Several properties control the creation and details of the class descriptor.


\subsection{Controlling Descriptor Generation}
\label{sec:msg-defs:descriptor-property}

The \fprop{@descriptor} class property can be used to control the generation
of the descriptor class. \ttt{@descriptor(readonly)} instructs the message
compiler not to generate field setters for the descriptor, and
\ttt{@descriptor(false)} instructs it not to generate a description class
for the class at all.


\subsection{Generating Descriptors For Existing Classes}
\label{sec:msg-defs:descriptor-existingclass-property}

It is also possible to use (or abuse) the message compiler for generating a
descriptor class for an existing class. To do that, write a message definition
for your existing class (for example, if it has \ttt{int getFoo()} and
\ttt{setFoo(int)} methods, add an \ttt{int foo} field to the message
definition), and mark it with \fprop{@existingClass}. This will tell the
message compiler that it should not generate an actual class (as it already
exists), only a descriptor class.


\subsection{Field Metadata}
\label{sec:msg-defs:descriptor-field-meadata}

When an object is shown in Qtenv's Object Inspector pane, Qtenv obtains all
information it displays from the object's descriptor. There are several
properties that can be used to customize how a field appears in the Object
Inspector:

\begin{itemize}
  \item \fprop{@icon} associates an icon with the field;
  \item \fprop{@label} overrides the text displayed as field name;
  \item \fprop{@group} is used for grouping related fields;
  \item \fprop{@hint} can be used to provide a short description of the field,
    which Qtenv displays in a tooltip
\end{itemize}


\subsection{Method Name Properties}
\label{sec:msg-defs:descriptor-method-name-properties}

Several of the properties which are for overriding field accessor method names
(\fprop{@getter}, \fprop{@setter}, \fprop{@sizeGetter}, \fprop{@sizeSetter},
etc., see \ref{sec:msg-defs:customizing-method-names}) have a secondary purpose.
When generating a descriptor for an \textit{existing} class (see
\fprop{@existingClass}), those properties specify how the descriptor can access
the field, i.e. what code to generate in the implementation of the descriptor's
various methods. In that use case, such properties may contain code fragments or
a function call template instead of a method name.


\subsection{toString/fromString}
\label{sec:msg-defs:descriptor-tostring}

To be able to generate the descriptor's \ffunc{getFieldValueAsString()} member
function, the message compiler needs to know how to convert the return type of
the getter to \ttt{std::string}. Similarly, for \ffunc{setFieldValueAsString()} it
needs to know how to convert (or parse) a string to obtain the setter's argument
type. For the built-in types (\fkeyword{int}, \fkeyword{double}, etc.) this information is
pre-configured, but for other types the user needs to supply it via two
properties:

\begin{itemize}
  \item \fprop{@toString} specifies the code to convert the return type of the setter to a string;
  \item \fprop{@fromString} specifies the code to convert a string to the setter's argument type.
\end{itemize}

These properties can be specified on the class (where it will be applied to
fields of that type), or directly on fields. Multiple syntaxes are accepted:

\begin{itemize}
  \item If the value starts with a dot, it is interpreted as a member function call.
  \item If the value contains a dollar sign, it serves as a placeholder for the
        value to be converted.
\end{itemize}

Example:

\begin{msg}
class IPAddress
{
    @existingClass;
    @opaque;
    @toString(.str());  // use IPAddress::str() to produce a string
    @fromString(IPAddress($));  // use constructor; '$' will be replaced by the string
}
\end{msg}

If the \fprop{@toString} property is missing, the message compiler generates
code which calls the \ffunc{str()} member function on the value returned by the
getter, provided that it knows for certain that the corresponding type has such
method (the type is derived from \cclass{cObject}, or has the \fprop{@str} property).

If there is no \fprop{@toString} property and no (known) \ffunc{str()} method,
the descriptor will return the empty string.


\subsection{toValue/fromValue}
\label{sec:msg-defs:descriptor-tovalue}

Similarly to \fprop{@toString}/\fprop{@fromString} described in the previous
section, the \fprop{@toValue} and \fprop{@fromValue} properties are used define
how to convert the field's value to and from \cclass{cValue} for the
descriptor's \ffunc{getFieldValue()} and \ffunc{setFieldValue()} methods.


\subsection{Field Modifiers}
\label{sec:msg-defs:descriptor-fieldmodifiers}

There are several boolean-valued properties which enable/disable various
features in the descriptor:

\begin{itemize}

  \item \fprop{@opaque}: If true: Treat the field as atomic (non-compound) type, i.e.
    having no descriptor class. When specified on a class, it determines the default
    for fields of that type.

  \item \fprop{@editable}: If set, value of the field (or value of fields that
    are instances of this type) can be set via the class descriptor's
    \ffunc{setFieldValueFromString()} and \ffunc{setFieldValue()} methods.

  \item \fprop{@replaceable}: If set, field is a pointer whose value can be set
    via the class descriptor's \ffunc{setFieldStructValuePointer()} and
    \ffunc{setFieldValue()} methods.

  \item \fprop{@resizable}: If set, field is a variable-size array whose size
    can be set via the class descriptor's \ffunc{setFieldArraySize()} method.

  \item \fprop{@readonly}: This is simply a shorthand for \ttt{@editable(false)
    @replaceable(false) @resizable(false)}.

\end{itemize}
